\documentclass[12pt]{article}

\usepackage{graphicx,../lp,amsmath}
\newcommand{\aff}{\operatorname{aff}}
\newcommand{\lin}{\operatorname{lin}}
\newcommand{\cone}{\operatorname{cone}}
\newcommand{\conv}{\operatorname{conv}}
\newcommand{\rank}{\operatorname{rank}}

% Cross-references for handout numbers.
\usepackage{amsfonts}
%\usepackage{amsthm}
\usepackage{hyperref}
\usepackage{amssymb}
%\usepackage[capitalize]{cleveref}
\usepackage{xcolor}

%\input{handouts}

\newcounter{chapnum}

\newtheorem{definition}{Definition}[chapnum]
\newtheorem{remark}{Remark}[chapnum]
\newtheorem{theorem}{Theorem}[chapnum]
\newtheorem{lemma}[theorem]{Lemma}
\newtheorem{corollary}[theorem]{Corollary}
\newtheorem{proposition}[theorem]{Proposition}
\newtheorem{claim}[theorem]{Claim}
\newtheorem{observation}{Observation}[chapnum]

\renewcommand{\thesection}{\arabic{chapnum}.\arabic{section}}
\renewcommand{\thefigure}{\arabic{chapnum}.\arabic{figure}}


\newenvironment{proof}{\noindent{\bf Proof:} \hspace*{1em}}{
        \hspace*{\fill} $\triangle$ }
\newenvironment{proof_of}[1]{\noindent {\bf Proof of #1:}
        \hspace*{1em} }{\hspace*{\fill} $\triangle$ }
\newenvironment{proof_claim}{\begin{quotation} \noindent}{
        \hspace*{\fill} $\diamond$ \end{quotation}}
\newenvironment{solution}{\noindent{\bf Solution:} \hspace*{1em}}{
        \hspace*{\fill} $\triangle$ }


\newcommand{\R}{{\mathbb R}}
\newcommand{\Z}{{\mathbb Z}}
\newcommand{\Q}{{\mathbb Q}}
\newcommand{\C}{{\mathbb C}}
\newcommand{\N}{{\mathbb N}}
\newcommand{\lin}{\operatorname{lin}}
\newcommand{\aff}{\operatorname{aff}}
\newcommand{\cone}{\operatorname{cone}}
\newcommand{\conv}{\operatorname{conv}}
\newcommand{\vol}{\operatorname{vol}}
\newcommand{\poly}{\operatorname{poly}}




\newcommand{\CF}[1]{{\color{purple}[CF: #1]}}


\newlength{\toppush}
\setlength{\toppush}{2\headheight}
\addtolength{\toppush}{\headsep}

\newcommand{\htitle}[2]{\noindent\vspace*{-\toppush}\newline\parbox{6.5in}
{Massachusetts Institute of Technology \hfill 18.453: Combinatorial Optimization 
\newline
\textbf{Instructor:} Cole Franks \quad \textbf{Notes: }Michel Goemans and Zeb Brady \hfill#2\newline
\mbox{}\hrulefill\mbox{}}\vspace*{1ex}\mbox{}\newline
\begin{center}{\Large\bf #1}\end{center}}

\newcommand{\handout}[2]{\thispagestyle{empty}
 \markboth{ #1 \hfil #2}{ #1 \hfil #2}
 \pagestyle{myheadings}\htitle{#1}{#2}}


\setlength{\oddsidemargin}{0pt}
\setlength{\evensidemargin}{0pt}
\setlength{\textwidth}{6.5in}
\setlength{\topmargin}{0in}
\setlength{\textheight}{8.5in}


\newcounter{exercisenum}
\newcounter{exercisetot}
\setcounter{exercisetot}{0}



\newenvironment{exercises}{
	\begin{list}{{\bf Exercise \arabic{chapnum}-\arabic{exercisenum}. \hspace*{0.5em}}}
	{\setlength{\leftmargin}{0em}
	 \setlength{\rightmargin}{0em}
	 \setlength{\labelwidth}{0em}
	 \setlength{\labelsep}{0em}
	\usecounter{exercisenum}
      \setcounter{exercisenum}{\theexercisetot}}}{\setcounter{exercisetot}{\theexercisenum}\end{list}}


\newenvironment{pseudocode}{
    \begin{list}{}{
        \renewcommand{\makelabel}{$\triangleright$}
        \setlength{\topsep}{0pt}
        \setlength{\leftmargin}{32pt}
        \setlength{\labelwidth}{14pt}
        \setlength{\labelsep}{0mm}
        \setlength{\itemindent}{0mm}
        \setlength{\itemsep}{-3pt}
        \setlength{\itemsep}{0mm}
        \setlength{\parsep}{0pt}%
        \setlength{\listparindent}{0pt}
    }
}
{
    \end{list}
}


\setcounter{chapnum}{3}


\begin{document}

\handout{\arabic{chapnum}. Linear Programming and Polyhedral Combinatorics}{\today}

Summary of what was seen in the introductory lectures on
linear programming and polyhedral combinatorics. The book "A first course on combinatorial optimization" by Lee is a good second supplementary source.

\begin{definition} A halfspace in $\R^n$ is a set of the form $\{x\in
  \R^n: a^Tx\leq b\}$ for some vector $a\in \R^n$ and $b\in \R$.
\end{definition}

\begin{definition} A polyhedron is the intersection of finitely many
  halfspaces: $P=\{x\in \R^n: Ax \leq b\}$. 
\end{definition}

\begin{definition} \label{def1} A polytope is a bounded polyhedron. 
\end{definition}

\begin{definition}
If $P$ is a polyhedron in $\R^n$, the projection $P_k\subseteq
\R^{n-1}$ of $P$ is defined as \\
$\{y=(x_1,x_2,\cdots,x_{k-1},x_{k+1},\cdots, x_n): x\in P$ for some
$x_k\in\R\}$.
\end{definition}
This is a special case of a projection onto a linear space (here, we
consider only coordinate projection). By repeatedly projecting, we can
eliminate any subset of coordinates.  

We claim that $P_k$ is also a polyhedron and this can be proved by
giving an explicit description of $P_k$ in terms of linear
inequalities. For this purpose, one uses {\it Fourier-Motzkin elimination}. 
Let $P=\{x: Ax\leq b\}$ and let
\begin{itemize}
\item
$S_+=\{i: a_{ik} >0\}$,
\item
$S_-=\{i: a_{ik}<0\}$,
\item
$S_0=\{i:a_{ik}=0\}$.
\end{itemize}
%
Clearly, any element in $P_k$ must satisfy the inequality $a_i^Tx\leq
b_i$ for all $i\in S_0$ (these inequalities do not involve $x_k$).
Similarly, we can take a linear combination of an inequality in $S_+$
and one in $S_-$ to eliminate the coefficient of $x_k$. This shows
that the inequalities:
%
\begin{equation} \label{proj}
a_{ik} \left(\sum_j a_{lj}x_j \right) -a_{lk} \left(\sum_j a_{ij}
x_j\right) \leq a_{ik} b_l - a_{lk} b_i 
\end{equation}
for $i\in S_+$ and $l\in S_-$ are satisfied by all elements of
$P_k$. Conversely, for any vector
$(x_1,x_2,\cdots,x_{k-1},x_{k+1},\cdots, x_n)$ satisfying (\ref{proj})
for all $i\in S_+$ and $l\in S_-$ and also 
\begin{equation} \label{proj2}
a_i^Tx\leq b_i \mbox{ for all } i\in S_0
\end{equation}
 we can find a value of $x_k$ such that the resulting $x$ belongs to
 $P$ (by looking at the bounds on $x_k$ that each constraint imposes,
 and showing that the largest lower bound is smaller than the smallest
 upper bound). This shows that $P_k$ is described by (\ref{proj}) and
 (\ref{proj2}), and therefore is a polyhedron. 

\begin{definition}
Given points $a^{(1)}, a^{(2)}, \cdots, a^{(k)}\in \R^n$, 
\begin{itemize}
\item
a {\bf linear} combination is  $\sum_i \lambda_i a^{(i)}$ where
$\lambda_i\in \R$ for all $i$,
\item
an {\bf affine} combination is  $\sum_i \lambda_i a^{(i)}$ where
$\lambda_i\in \R$ and $\sum_i \lambda_i=1$,
\item
a {\bf conical} combination is  $\sum_i \lambda_i a^{(i)}$ where
$\lambda_i\geq 0$ for all $i$,
\item
a {\bf convex} combination is  $\sum_i \lambda_i a^{(i)}$ where
$\lambda_i \geq 0$ for all $i$ and $\sum_i \lambda_i=1$.
\end{itemize}
\end{definition}

The set of all linear combinations of elements of $S$ is called the
linear hull of $S$ and denoted by $\lin(S)$. Similarly, by replacing
{\it linear} by {\it affine, conical or convex},  we define the
affine hull, $\aff(S)$, the conic hull, $\cone(S)$ and the convex hull,
$\conv(S)$.  We can give an equivalent definition of a polytope.

\begin{definition} \label{def2}
A polytope is the convex hull of a finite set of points.
\end{definition}

The fact that Definition \ref{def2} implies Definition \ref{def1} can be seen as follows. Take $P$ be the convex hull of a finite set $\{a^{(k)}\}_{k\in[m]}$ of points. To show that $P$ can be described as the intersection of a finite number of halfspaces, we can apply 
Fourier-Motzkin elimination repeatedly on
$$x-\sum_k \lambda_k a^{(k)} = 0$$
$$  \sum_k \lambda_k=1$$
$$\lambda_k\geq 0$$
to eliminate all variables $\lambda_k$ and keep only the variables
$x$. Furthermore, $P$ is bounded since for any $x\in P$, we have 
$$||x||=||\sum_k \lambda_k a^{(k)}|| \leq \sum_k \lambda_k ||a^{(k)}|| \leq \max_k ||a^{(k)}||.$$ The converse  will be proved later in these notes.

\section{Solvability of
  System of Inequalities}

In linear algebra, we saw that, for $A\in \R^{m\times n}$, $b\in \R^m$,
$Ax=b$ has no solution $x\in \R^n$ if and only if
there exists a $y\in \R^m$ with $A^Ty=0$ and $b^Ty\neq 0$ (in 18.06
notation/terminology, this is because the column
space $C(A)$ is orthogonal to the left null space $N(A^T)$).

One can state a similar {\it Theorem of the Alternatives} for systems
of linear inequalities. 

\begin{theorem}[Theorem of the Alternatives]
$Ax \leq b$ has no solution $x\in \R^n$ if and only if there exists
  $y\in \R^m$ such that $y\geq 0$, $A^Ty=0$ and $b^Ty<0$.
\end{theorem}

One can easily show that both systems indeed cannot have a solution
since otherwise $0>b^Ty=y^Tb \geq y^TAx =0^Tx=0$. For the other
direction, one takes the insolvable system $Ax\leq b$ and use 
Fourier-Motzkin elimination repeatedly to eliminate all variables and
thus obtain an inequality of the form $0^Tx\leq c$ where $c<0$. In the
process one has derived a vector $y$ with the desired properties (as
Fourier-Motzkin only performs nonnegative combinations of linear
inequalities). 

Another version of the above theorem is Farkas' lemma:
\begin{lemma}
$Ax=b$, $x\geq 0$ has no solution if and only if there exists $y$ with
  $A^Ty\geq 0$ and $b^Ty<0$.
\end{lemma}

\begin{exercises}
\item
Prove Farkas' lemma from the Theorem of the Alternatives.
\end{exercises} 
We may also interpret Farkas' lemma in the context of separating hyperplanes.
\begin{theorem}
Suppose $K, L$ are convex bodies in $\mathbb{R}^n$ and suppose one of them is compact. There is a hyperplane separating $K$ and $L$; namely there is $\alpha \in \mathbb{R}^n$ and $b \in \mathbb{R}$ such that $\alpha^T x < b$ for all $x \in K$ and $\alpha^T x > b$ for all $x \in L$. 
\end{theorem}

\noindent \textbf{Exercise 3-1.1} Prove the forward implication in Farkas' lemma using the separating hyperplane theorem. 
\section{Linear Programming Basics}

A linear program (LP) is the problem of minimizing or maximizing a linear
function over a polyhedron:
\lps
  &  &  & \mbox{Max} &  c^Tx\\
  & \lefteqn{\mbox{subject to:}} \\
  &  (P)      &   &  &   Ax \leq b,
\elps
where $A\in \R^{m\times n}$, $b\in \R^m$, $c\in \R^n$ and the
variables $x$ are in $\R^n$. 
Any $x$ satisfying $Ax\leq b$ is said to be {\it feasible}. If no $x$
satisfies $Ax\leq b$, we say that the linear program is {\it
infeasible}, and its optimum value is $-\infty$ (as we are maximizing
over an empty set). If the objective function value of the linear
program can be made arbitrarily large, we say that the linear program
is {\it unbounded} and its optimum value is $+\infty$; otherwise it is
{\it bounded}. If it is neither infeasible, nor unbounded, then its
optimum value is finite.

Other equivalent forms involve equalities as well, or nonnegative
constraints $x\geq 0$. One version that is often considered when
discussing algorithms for linear programming (especially the simplex
algorithm) is $\min\{c^Tx: Ax=b, x\geq 0\}$. 

Another linear program, {\it dual} to $(P)$, plays a crucial role:
\lps
  &  &  & \mbox{Min} &  b^Ty\\
  & \lefteqn{\mbox{subject to:}} \\
  &  (D)      &   &  &   A^Ty=c \\
 & & & & y\geq 0. 
\elps
$(D)$ is the dual and $(P)$ is the {\it primal}. The terminology for
the dual is similar. If $(D)$ has no feasible solution, it is said to
be {\it infeasible} and its optimum value is $+\infty$ (as we are
minimizing over an empty set). If $(D)$ is unbounded (i.e. its value
can be made arbitrarily negative) then its optimum value is
$-\infty$. 

The primal and dual spaces should not be confused. If $A$ is $m \times
n$ then we have $n$ primal variables and $m$ dual variables.

{\bf Weak duality} is clear: For any feasible solutions $x$ and $y$ to
$(P)$ and $(D)$, we have that $c^Tx\leq b^Ty$. Indeed,
$c^Tx=y^TAx\leq b^Ty$. The dual was precisely built to get an upper
bound on the value of any primal solution. For example, to get the inequality
$y^TAx \leq b^Ty$, we need that $y\geq 0$ since we know that $Ax\leq
b$. In particular, weak duality implies that if the primal is
unbounded then the dual must be infeasible. 

{\bf Strong duality} is the most important result in linear
programming; it says that we can prove the optimality of a primal
solution $x$ by exhibiting an optimum dual solution $y$. 
\begin{theorem}[Strong Duality]
Assume that $(P)$ and $(D)$ are feasible, and let $z^*$ be the optimum
value of the primal and $w^*$ the optimum value of the dual. Then
$z^*=w^*$. 
\end{theorem}

One proof of strong duality is obtained by writing a big system of
inequalities in $x$ and $y$ which says that (i) $x$ is primal
feasible, (ii) $y$ is dual feasible and (iii) $c^Tx\geq b^Ty$. Then use
the Theorem of the Alternatives to show that the infeasibility of this
system of inequalities would contradict the feasibility of either
$(P)$ or $(D)$. 

\begin{proof}
  Let $x^*$ be a feasible solution to the primal, and $y^*$ be a
  feasible solution to the dual. The proof is by contradiction.
  Because of weak duality, this means that there are no solution $x\in
  \R^n$ and $y\in \R^m$ such that
$$\left\{ \begin{array}{lll} Ax & & \leq b \\ & A^T y & =c \\ & -Iy &
    \leq 0 \\ -c^Tx & + b^Ty & \leq 0 \end{array} \right.$$
By a variant of the Theorem of the Alternatives or Farkas' lemma (for
the case when we have a combination of inequalities and equalities),
we derive that there must exist $s\in \R^m$, $t\in \R^n$,
$u\in \R^m$, $v\in \R$ such that:
\begin{eqnarray*}
 A^Ts-vc&=0 \\
 At -u + vb& = 0 \\
s&\geq 0 \\
 u&\geq 0 \\
 v&\geq 0 \\
 b^Ts + c^Tt &< 0.
\end{eqnarray*}
We distinguish two cases. 

\paragraph{Case 1: $v=0$.} Then $s$ satisfies $s\geq 0$ and
$A^Ts=0$. This means that, for any $\alpha\geq 0$, $y^*+\alpha s$ is
feasible for the dual. Similarly, $At=u \geq 0$ and therefore, for any
$\alpha\geq 0$, we have that $x^*-\alpha t$ is primal feasible. By
weak duality, this
means that, for any $\alpha\geq 0$, we have $$c^T(x^*-\alpha t) \leq
b^T(y^*+\alpha s)$$ or $$c^Tx^*-b^Ty^* \leq \alpha (b^Ts+c^Tt).$$
The right-hand-side tend to $-\infty$ as $\alpha$ tends to $\infty$,
and this is a contradiction as the left-hand-side is fixed. 

\paragraph{Case 2: $v>0$.} By dividing throughout by $v$ (and renaming
all the variables), we get that
there exists $s\geq 0$, $u\geq 0$ with 
\begin{eqnarray*}
A^Ts & = & c \\
At-u & = & -b \\
b^Ts + c^T t & < & 0.
\end{eqnarray*}
This means that $s$ is dual feasible and $-t$ is primal feasible, and
therefore by weak duality $c^T(-t) \leq b^Ts$ contradicting $b^Ts+c^Tt<0$. 
\end{proof}

We can also prove strong duality directly using Fourier-Motzkin elimination.

\begin{proof} Suppose that $c \ne \vec{0}$ (if $c = \vec{0}$, then the theorem follows from the Theorem of Alternatives). Then there is an invertible $n\times n$ matrix $L$ whose first row is $c^T$. Letting $z = Lx$, we see that we can rewrite our linear program in terms of $z$ as
\lps
  &  &  & \mbox{Max} &  z_1\\
  & \lefteqn{\mbox{subject to:}} \\
  &  (P)      &   &  &   AL^{-1}z \leq b.
\elps
The dual linear program is given by
\lps
  &  &  & \mbox{Min} &  b^Ty\\
  & \lefteqn{\mbox{subject to:}} \\
  &  (D)      &   &  &   (AL^{-1})^Ty=e_1 \\
 & & & & y\geq 0,
\elps
where $e_1$ is the first basis vector. Multiplying both sides of $(AL^{-1})^Ty=e_1$ with $L^T$ on the left, we see that
\[
(AL^{-1})^Ty=e_1 \;\;\; \iff \;\;\; A^Ty = c,
\]
so the dual linear program is unchanged.

Now we apply Fourier-Motzkin elimination to the system $AL^{-1}z \le b$ to eliminate the variables $z_2, ..., z_n$, getting a system of linear inequalities in $z_1$, each of which is formed by some nonnegative linear combination of the rows of the system $(AL^{-1})z \le b$. We can divide these inequalities on $z_1$ into three types: upper bounds on $z_1$, lower bounds on $z_1$, and inequalities not involving $z_1$ at all. The inequalities not involving $z_1$ at all must be \emph{true} inequalities between constants by the assumption that the primal problem was feasible, and for the same reason every upper bound on $z_1$ must be at least as large as every lower bound on $z_1$. By the assumption that the dual problem was feasible, we see that there is at least one upper bound on $z_1$. So the maximum value $z_1 = c^Tx$ can take on our polytope is equal to the minimum upper bound which can be proved on $z_1$ by adding together some nonnegative linear combination of the rows of the system $Ax = AL^{-1}z \le b$.
\end{proof}

\begin{exercises}
\item
Show that the dual of the dual is the primal.
\item 
Show that we only need either the primal or the dual to be
  feasible for strong duality to hold. More precisely, if the primal
  is feasible but the dual is infeasible, prove that the primal will
  be unbounded, implying that $z^*=w^*=+\infty$.
\end{exercises}

Looking at $c^Tx=y^TAx\leq b^Ty$, we observe that to get equality
between $c^Tx$ and $b^Ty$, we need {\it complementary slackness}:

\begin{theorem}[Complementary Slackness]
If $x$ is feasible in $(P)$ and $y$ is feasible in $(D)$ then $x$ is
optimum in $(P)$ and $y$ is optimum in $(D)$ if and only if for all
$i$ either $y_i=0$ or $\sum_j a_{ij} x_j =b_i$ (or both). 
\end{theorem}

Linear programs can be solved using the simplex method; this is not
going to be explained in these notes. No variant of the simplex method
is known to provably run in polynomial time, but there are other
polynomial-time algorithms for linear programming, namely the
ellipsoid algorithm and the class of interior-point algorithms. 

\section{Faces of Polyhedra}

\begin{definition}
$\{a^{(i)}\in \R^n: i\in K\}$ are linearly independent if $\sum_i \lambda_i
  a^{(i)}=0$ implies that $\lambda_i=0$ for all $i\in K$.
\end{definition} 

\begin{definition}
$\{a^{(i)}\in \R^n: i\in K\}$ are {\it affinely} independent if
  $\sum_i \lambda_i a^{(i)}=0$ and $\sum_i \lambda_i=0$ together imply
  that $\lambda_i=0$ for all $i\in K$.
\end{definition} 

Observe that $\{a^{(i)}\in \R^n: i\in K\}$ are {\it affinely}
independent  if and only if $$\left\{\left[\begin{array}{c} a^{(i)}\\
    1\end{array} \right]\in \R^{n+1}: i\in K\right\}$$ are {\it linearly}
independent.  

\begin{definition}
The dimension, $\dim(P)$,  of a polyhedron $P$ is the maximum number of affinely
independent points in $P$ minus 1. 
\end{definition}

(This is the same notion as the dimension of the affine hull $\aff(S)$.) The dimension can be -1 (if $P$ is empty), 0 (when $P$ consists of a
single point), 1 (when $P$ is a line segment), and up to $n$ when $P$
affinely spans $\R^n$. In the latter case, we say that $P$ is {\it
  full-dimensional}.  The dimension of a cube in $\R^3$ is 3, and so
is the dimension of $\R^3$ itself (which is a trivial polyhedron). 

\begin{definition}
$\alpha^Tx\leq \beta$ is a {\it valid inequality} for $P$ if
  $\alpha^Tx \leq \beta$ for all $x\in P$.
\end{definition}

Observe that for an inequality to be valid for $\conv(S)$ we only need
to make sure that it is satisfied by all elements of $S$, as this will
imply that the inequality is also satisfied by points in
$\conv(S)\setminus S$. This observation will be important when dealing
with convex hulls of combinatorial objects such as matchings or
spanning trees. 

\begin{definition}
A {\it face} of a polyhedron $P$ is $\{x\in P: \alpha^Tx=\beta\}$
where $\alpha^Tx\leq \beta$ is some valid inequality of $P$.
\end{definition}

By definition, all faces are polyhedra. The empty face (of dimension
-1) is {\it trivial}, and so is the entire polyhedron $P$
(which corresponds to the valid inequality $0^Tx\leq 0$). Non-trivial
are those whose dimension is between 0 and $\dim(P)-1$. Faces of
dimension 0 are called {\it extreme points} or {\it vertices}, faces
of dimension 1 are called {\it edges}, and faces of dimension
$\dim(P)-1$ are called {\it facets}. Sometimes, one uses {\it ridges}
for faces of dimension $\dim(P)-2$. 

\begin{exercises}
\item List all 28 faces of the cube $P=\{x\in \R^3: 0\leq x_i \leq 1$
  for $i=1,2,3\}$.  
\end{exercises}

Although there are infinitely many valid inequalities, there are only
finitely many faces. 

\begin{theorem} \label{thface}
Let $A\in \R^{m\times n}$. Then any non-empty face of $P=\{x\in \R^n: Ax \leq
b\}$ corresponds to the set of solutions to  
$$ \sum_j a_{ij} x_j = b_i \mbox{ for all } i\in I $$
$$ \sum_j a_{ij} x_j \leq b_i \mbox{ for all } i\notin I, $$
for some set $I\subseteq \{1,\cdots,m\}$. Therefore, the number of
non-empty faces of $P$ is at most $2^m$. 
\end{theorem}
\begin{proof}
Consider any valid inequality $\alpha^Tx\leq \beta$. Suppose the
corresponding face $F$ is non-empty. Thus $F$ are all optimum
solutions to 
\lps
  &  &  & \mbox{Max} &  \alpha^Tx \\
  & \lefteqn{\mbox{subject to:}} \\
  &  (P)      &   &  &   Ax \leq b.
\elps
Choose an optimum solution $y^*$ to the dual LP. By complementary
slackness, the face $F$ is defined by those elements $x$ of $P$ such
that $a_i^Tx=b_i$ for $i\in I=\{i: y_i^*>0\}$. Thus $F$ is defined by 
$$ \sum_j a_{ij} x_j = b_i \mbox{ for all } i\in I $$
$$ \sum_j a_{ij} x_j \leq b_i \mbox{ for all } i\notin I. $$
As there are $2^m$ possibilities for $I$, there are at most $2^m$
non-empty faces. 
\end{proof}

The number of faces given in Theorem \ref{thface} is tight for
polyhedra (see exercise below), but can be considerably improved for
polytopes in the so-called {\it upper bound theorem} (which is not given in these notes).

\begin{exercises}
\item
Let $P=\{x\in \R^n: x_i\geq 0$ for $i=1,\cdots, n\}$. Show that $P$
has $2^n+1$ faces. How many faces of dimension $k$ does $P$ have? 
\end{exercises}

For extreme points (faces of dimension 0), the characterization is
even stronger (we do not need the inequalities):

\begin{theorem}
Let $x^*$ be an extreme point for $P=\{x: Ax \leq b\}$. Then there
exists $I$ such that $x^*$ is the {\it unique} solution to 
$$ \sum_j a_{ij} x_j = b_i \mbox{ for all } i\in I. $$
Moreover, if $x^* \in P$ is such a unique solution for some $I$, then $x^*$ is extreme. 
 \end{theorem}

\begin{proof}
Given an extreme point $x^*$, define $I=\{i: \sum_j a_{ij}
x^*_j=b_i\}$. This means that for $i\notin I$, we have $\sum_j a_{ij}
x^*_j<b_i$.  

From Theorem \ref{thface}, we know that $x^*$ is uniquely
defined by 
\begin{equation} \label{r1}
 \sum_j a_{ij} x_j = b_i \mbox{ for all } i\in I
\end{equation}
\begin{equation} \label{r2}
\sum_j a_{ij} x_j \leq b_i \mbox{ for all } i\notin I. 
\end{equation}
Now suppose there exists another solution $\hat{x}$ {\it when} we
consider only the equalities for $i\in I$. Then because of $\sum_j a_{ij}
x^*_j<b_i$, we get that $(1-\epsilon)x^* + \epsilon \hat{x}$ also
satisfies (\ref{r1}) and (\ref{r2}) for $\epsilon$ sufficiently
small. A contradiction (as the face was supposed to contain a single point). 

\noindent \textbf{Exercise 3-5.5:} Prove the converse of what we just proved, namely that if $x^*$ is a unique solution for some $I$ then $x^*$ is extreme.
\end{proof}

If $P$ is given as $\{x: Ax=b, x\geq 0\}$ (as is often the case), the
theorem still applies (as we still have a system of inequalities). In
this case, the theorem says that every extreme point $x^*$ can be obtained
by setting some of the variables to 0, and solving for the unique
solution to the resulting system of equalities.

In fact, we can say more. Without loss of
generality, we can remove from $Ax=b$ equalities that are redundant;
this means that we can assume that $A$ has full row rank ($\rank(A)=m$
for $A\in \R^{m\times n}$). Let $N$ denote the
indices of the {\it non-basic} variables that we set to 0 and $B$
denote the remaining indices (of the so-called {\it basic} variables).
We can partition $x^*$ into $x^*_B$ and $x_N^*$ (corresponding to
these two sets of variables) and rewrite $Ax=b$ as $A_Bx_B+A_Nx_N=b$,
where $A_B$ and $A_N$ are the restrictions of $A$ to the indices in
$B$ and $N$ respectively. The theorem says that $x^*$ is the unique
solution to $A_Bx_B+A_Nx_N=0$ and $x_N=0$, which means $x^*_N=0$ and
$A_Bx_B^*=b$. This latter system must have a unique solution, which
means that $A_B$ must have full column rank ($\rank(A_B)=|B|$). As $A$
itself has rank $m$, we have that $|B|\leq m$ and we can augment $B$
to include indices of $N$ such that the resulting $B$ satisfies (i)
$|B|=m$ and (ii) $A_B$ is a $m\times m$ invertible matrix (and thus
there is still a unique solution to $A_Bx_B=b$). In linear programming
terminology, a {\it basic feasible solution} or {\it bfs} of $\{x:
Ax=b, x\geq 0\}$ is obtained by choosing a set $|B|=m$ of indices with
$A_B$ invertible and letting $x_B=A_B^{-1} b$ and $x_N=0$ where $N$
are the indices not in $B$. We have thus shown that all extreme points are bfs, and vice versa. Observe that 
 two different bases $B$ may lead to the same extreme point,
as there might be many ways of extending $A_B$ into a $m\times m$
invertible matrix in the discussion above.

One consequence we could derive from Theorem \ref{thface} is:
\begin{corollary} \label{corfacets}
The maximal (inclusion-wise) non-trivial faces of a non-empty
polyhedron $P$ are the facets.
\end{corollary}
For the vertices, one needs one additional condition:
\begin{corollary} \label{corvertices}
If $\rank(A)=n$ (full column rank) then the minimal (inclusion-wise)
non-trivial faces of a non-empty polyhedron $P=\{x\in \R^n: Ax\leq
b\}$ are the vertices.
\end{corollary}
Exercise 3-\ref{ex-novertex} shows that the rank condition is necessary. 

This means that, if a linear program $\max\{c^Tx: x\in P\}$ with
$P=\{x: Ax\leq b\}$ is feasible, bounded and $\rank(A)=n$, then there
exists an optimal solution which is a vertex of $P$ (indeed, the set
of all optimal solutions defines a face --- the optimal face --- and if
this face is not itself a vertex of $P$, it must contain vertices of
$P$).

We now prove Corollary \ref{corvertices}. 

\begin{proof}
Let $F$ be a minimal (inclusion-wise) non-trivial face of $P$. This
means that we have a set $I$ such that 
$$\begin{array}{lll}
F=\{x: & a_i^Tx = b_i  & \forall i\in I \\
       & a_j^Tx\leq b_j & \forall j\notin I \}
\end{array}
$$ and adding any element to $I$ makes this set empty. Consider two
cases. Either $F=\{x\in \R^n: a^T_ix=b_i$ for $i\in I\}$ or not. In
the first case, it means that for every $j\notin I$ we have $a_j\in
\lin(\{a_i: i\in I\})$ (otherwise there would be a solution $x$ to
$a_i^Tx = b_i$ for all $i\in I$ and $a_j^Tx=b_j+1$ and hence not in
$F$) and therefore since $\rank(A)=n$ we have that the
system $a_i^Tx=b_i$ for all $i\in I$ has a unique solution and thus $F$ is
a vertex. 

On the other hand, if $F\neq \{x\in \R^n: a^T_ix=b_i$ for $i\in I\}$
then let $j\notin I$ such that there exists $\tilde{x}$ with 
$$\begin{array}{ll}
a_i^T\tilde{x} = b_i  & i\in I \\
a_j^T\tilde{x} > b_j. 
\end{array}
$$ Since $F$ is not trivial, there exists $\hat{x}\in F$. In
particular, $\hat{x}$ satisfies $$\begin{array}{ll}
a_i^T\hat{x} = b_i  & i\in I \\
a_j^T\hat{x} \leq b_j. 
\end{array}$$ Consider a convex combination $x'=\lambda \tilde{x} + (1-\lambda)
\hat{x}$. Consider the largest $\lambda$ such that $x'$ is in
$P$. This is well-defined as $\lambda=0$ gives a point in $P$ while it
is not for $\lambda=1$. The corresponding $x'$ satisfies  $a_i^T x' =
b_i $ for $i\in I\cup\{k\}$ for some $k$ (possibly $j$),
contradicting the maximality of $I$.   
\end{proof}


We now go back to the equivalence between Definitions \ref{def1} and
\ref{def2} and claim that we can show that Definition \ref{def1}
implies Definition \ref{def2}. 
\begin{theorem} \label{boundedconv}
If $P=\{x: Ax \leq b\}$ is bounded then $P=\conv(X)$ where $X$ is the
set of extreme points of $P$.
\end{theorem}
This is a nice exercise using the Theorem of the Alternatives.

\begin{proof}
Since $X\subseteq P$, we have $\conv(X)\subseteq P$. Assume, by
contradiction, that we do not have equality. Then there must exist
$\tilde{x}\in P\setminus \conv(X)$. The fact that $\tilde{x}\notin
\conv(X)$ means that there is no solution to:
$$\left\{\begin{array}{ll}
\sum_{v\in X} \lambda_v v = \tilde{x} \\
\sum_{v\in X} \lambda_v = 1 \\
\lambda_v\geq 0 & v\in X.
\end{array}\right.$$
By the Theorem of the alternatives, this implies that $\exists c\in
\R^n, t\in \R$: 
$$\left\{\begin{array}{ll}
t + \sum_{j=1}^n c_j v_j \geq 0 & \forall v\in X\\
t + \sum_{j=1}^n c_j \tilde{x}_j < 0.
\end{array}\right.
$$
Since $P$ is bounded, $\min\{c^Tx: x\in P\}$ is finite (say equal to
$z^*$), and the face induced by $c^Tx \geq z^*$ is non-empty but does
not contain any vertex (as all vertices are dominated by $\tilde{x}$
by the above inequalities). This is a contradiction with Corollary
\ref{corvertices}. Observe, indeed, that Corollary \ref{corvertices} applies. If $\rank(A)<n$ there would exists $y\neq 0$ with $Ay=0$ and this would contradict the boundedness of $P$ (as we could go infinitely in the direction of $y$). 
\end{proof}




When describing a polyhedron $P$ in terms of linear inequalities, the only
inequalities that are needed are the ones that define facets of
$P$. This is stated in the next few theorems. We say that an inequality in the
system $Ax\leq b$ is {\it redundant} if the corresponding polyhedron is
unchanged by removing the inequality. For $P=\{x: Ax\leq b\}$, we let
$I_=$ denote the indices $i$ such that $a_i^Tx=b_i$ for all $x\in P$,
and $I_<$ the remaining ones (i.e. those for which there exists $x\in
P$ with $a_i^Tx<b_i$). 

This theorem shows that facets are sufficient:
\begin{theorem}
If the face associated with $a_i^Tx\leq b_i$ for $i\in I_<$ is not a facet
then the inequality is redundant. 
\end{theorem}
\begin{proof} If the inequality $a_i^Tx \le b_i$ is not redundant, then there must be a point $\hat{x}$ such that $a_i^T\hat{x} > b_i$ but $a_j^T\hat{x} \le b_j$ for all $j \ne i$. Let $F_i$ be the face of $P$ associated to the inequality $a_i^Tx \le b_i$. Then for every $x \in P$, the line segment connecting $x$ to $\hat{x}$ must contain a unique point $x_0 \ne \hat{x}$ with $a_i^Tx_0 = b_i$, and for every $j \ne i$ we have $a_j^Tx_0 \le b_j$ (since $x_0$ is a convex combination of points satisfying the $j$th inequality), so $x_0 \in F_i$. Since $x_0 \in F_i$ is on the line segment connecting $x$ to $\hat{x}$ and $x_0 \ne \hat{x}$, we see that $x$ is in the affine span of $\hat{x}$ and the face $F_i$. Since $x$ was an arbitrary point of $P$, we see that the affine spane of the face $F_i$ together with $\hat{x}$ contains the affine span of $P$, so $\dim(P) \le \dim(F_i) + 1$.

To see that $\dim(P) \ne \dim(F_i)$, we use the fact that $i \in I_<$, which means that $P$ contains a point $x_<$ with $a_i^Tx_< < b_i$, and this point $x_<$ is not in the affine span of $F_i$.
\end{proof}

And this one shows that facets are necessary:
\begin{theorem}
If $F$ is a facet of $P$ then there must exists $i\in I_<$ such that
the face induced by $a_i^T x\leq b_i$ is precisely $F$. 
\end{theorem}

In a {\it minimal} description of $P$, we must have a set of {\it
  linearly independent equalities} together with precisely one
  inequality for each facet of $P$.

Now that we understand the vertices and facets of a polyhedron, we may wish to understand the local geometry around the vertices, and some basic properties of the graph formed by the edges and vertices of the polyhedron.

\begin{theorem} Let $v_0$ be a vertex of $P$, corresponding to the valid inequality $c^Tx \le m$, and choose $\epsilon > 0$ such that $c^Tv' < m-\epsilon$ for all vertices $v'$ of $P$ other than $v_0$. Define the polyhedron $P_0$ to be $P \cap \{x \mid c^Tx = m-\epsilon\}$. Then $P_0$ is a {\it polytope} (that is, it is bounded), and for each $k$ there is a bijection between the set of faces of $P_0$ of dimension $k$ and the set of faces of $P$ which contain $v_0$ and have dimension $k + 1$.
\end{theorem}
We use the following exercise.\\
\noindent \textbf{Exercise 3-5.6:} Show that for any point $x_0$ in an unbounded polyhedron $P \subset \mathbb{R}^n$, $P$ contains a \emph{ray} from $x$, a set of the form $\{x_0 + \alpha y: \alpha \geq 0\}$ for some $y \in \mathbb{R}^n$. \textbf{Hint:} use LP duality and Farkas' lemma.\\
%cute with Farkas lemma.

\begin{proof}
%If $\rank(A) < n$, the theorem is vacuous. Thus we assume $\rank(A) = n$. 
First we show that $P_0$ is bounded. Suppose not; let $x_0 \in P_0$. Using the exercise, $P_0$ contains a \emph{ray} $\{x_0 + \alpha y: \alpha \geq 0\}$. As $P_0$ is contained within the affine space $x_0 + c^\perp$, $y$ is contained in $c^\perp$. We claim that the ray yields another minimizer; if we take the convex combination between $v_0$ and $x_0 + \alpha y$ given by 
$x(\alpha) := (1- \alpha^{-1}) v_0 + \alpha^{-1}(x_0 + \alpha y)$ we have $\lim_{\alpha \to \infty} x(\alpha) = v_0 + y \in P,$ and $c^T(v_0 + y) = c^T v_0.$ This contradicts the uniqueness of $v_0$.




%%%% old Zeb/Michel proof
%Suppose that $P = \{x \mid a_i^Tx \le b_i \;\; \forall i\}$. By LP-duality, there is a vector $y \ge 0$ such that $\sum_i y_ia_i = c$ and $\sum_i y_ib_i = m$. Moreover, we may assume that for $I := \{i \mid y_i > 0\}$, we have $\lin \{a_i: i \in I\} = \mathbb{R}^n$. We may assume this because the set of $c$'s that has $v_0$ as a unique minimizer has a positive volume ($v_0$ is still a unique minimizer on the intersection of $P$ and the unit cube, which is the convex hull of finitely many vertices), and the set of $c's$ expressible as linear combinations of non-spanning $a_i$'s is of zero measure. Thus $c$ can be perturbed so that $I$ has the desired property). For $x \in P_0$ and $i \in I$, if we set $z_i = y_i(b_i - a_i^Tx)$, then each $z_i \ge 0$ (since $x \in P$) and we have \[\sum_{i \in I} z_i = \sum_{i \in I} y_i(b_i - a_i^Tx) = \Big(\sum_i y_ib_i\Big) - \Big(\sum_i y_ia_i^T\Big)x = m - c^Tx = \epsilon.\]Thus, each $z_i$ is bounded between $0$ and $\epsilon$. Since the span of the $a_i$s with $i \in I$ is full-dimensional, we can recover the coordinates of $x$ from the $z_i$s, and since the set of possible tuples of $z_i$s is bounded, we see that $P_0$ must be bounded as well.

Next we show that every face of $P_0$ is the intersection of a face of $P$ containing $v_0$ and the hyperplane $c^Tx = m-\epsilon$. Let $F_0$ be a nonempty face of $P_0$, defined by $a_i^Tx = b_i$ for $i \in I$, $c^Tx = m-\epsilon$, and $a_j^T \le b_j$ for $j \not\in I$, and let $F$ be the face of $P$ defined by all the same equalities and inequalities other than $c^Tx = m-\epsilon$. We just need to show that $v_0 \in F$. Since $c^Tx$ is bounded above by $m$ on $F$, it reaches some maximum value $m'$ on $F$. Let $F'$ be the face of $F$ defined by $c^Tx = m'$, then $F'$ must have a vertex $v$ (since $P$ has a vertex). Since $F_0 \subseteq F$ was nonempty, we must have $m' \ge m-\epsilon$, and since $v \in F'$ we have $c^Tv = m' \ge m-\epsilon$, so by the choice of $\epsilon$ we must have $v = v_0$, so $v_0 \in F' \subseteq F$.

Finally, we show that if $F$ is a face of $P$ which contains $v_0$, and if $F_0 = F \cap \{x \mid c^Tx = m-\epsilon\}$, then $\dim(F_0) = \dim(F) - 1$. We will prove this by showing that for every $x \in F \setminus \{v_0\}$, the ray from $v_0$ to $x$ intersects $F_0$ - this will show that the affine span of $F_0$ together with the vertex $v_0$ contains the affine span of $F$. If $c^Tx \le m-\epsilon$, this is easy: the line segment connecting $v_0$ to $x$ will already intersect $F_0$. Otherwise, $x$ is in the polyhedron $F' = F \cap \{x \mid c^Tx \ge m-\epsilon\}$. By the same argument showing $P_0$ is bounded, we see that $F'$ is bounded, so $F'$ is the convex hull of its vertices. But the vertices of $F'$ are all either on the hyperplane $c^Tx = m-\epsilon$, in which case they are vertices of $F_0$, or they are vertices of $F$ which satisfy the inequality $c^Tv \ge m-\epsilon$, in which case they must be $v_0$ by the choice of $\epsilon$. Thus every point $x$ in $F'$ is a convex combination of the vertex $v_0$ and a point in $F_0$, so the ray from $v_0$ to $x$ intersects $F_0$.
\end{proof}

\begin{corollary} The graph of vertices and edges of a polyhedron $P$ is always connected. In fact, if $v_0, v^*$ are vertices of $P$ such that for some cost function $c^Tx$ we have $c^Tv^* = \max_{x \in P} c^Tx$, then there is a path $v_0, v_1, ..., v_k = v^*$ of vertices of $P$ with each pair $(v_i, v_{i+1})$ giving the two endpoints of an edge of $P$, such that $c^Tv_{i+1} \ge c^Tv_i$ for all $i$.
\end{corollary}
\begin{proof} Suppose for simplicity that $v^*$ is the unique maximizer of $c^Tx$. Suppose $v_0 \ne v^*$. We will show how to find an edge from $v_0$ to some vertex $v_1$ with $c^Tv_1 > c^Tv_0$: applying this iteratively and using the fact that $P$ has finitely many vertices will show the existence of the path. Define the polytope $P_0$ for the vertex $v_0$ as in the previous theorem, and let $x$ be the intersection point between the line segment connecting $v_0$ to $v^*$ and $P_0$. Since $v_0 \not\in P_0$ and $c^Tv^* > c^Tv_0$, we have $c^Tx > c^Tv_0$. Since $P_0$ is a polytope, it is the convex hull of its vertices, so some vertex $w$ of $P_0$ must have $c^Tw \ge c^Tx > c^Tv_0$. This vertex $w$ of $P_0$ is the intersection of an edge $e$ of $P$ containing $v_0$ with $P_0$, and this edge $e$ is contained in the ray from $v_0$ to $w$. If $e$ was unbounded, then $c^Tx$ would go to infinity along this ray (since $c^Tw > c^Tv_0$), contradicting the assumption that $c^Tx$ is bounded above by $c^Tv^*$. Thus $e$ is bounded, so it has a second endpoint $v_1$ with $c^Tv_1 \ge c^Tw > c^Tv_0$.
\end{proof}

\subsection*{Exercises}

\begin{exercises}
\item
Prove Corollary \ref{corfacets}.
\item \label{ex-novertex}
Show that if $\rank(A)<n$ then $P=\{x\in \R^n: Ax \leq b\}$ has no vertices.

\item \label{ex-face}
Suppose $P=\{x\in \R^n: Ax\leq b, Cx\leq d\}$. Show that the
set of  vertices of $Q=\{x\in \R^n: Ax \leq b, Cx = d\}$ is a subset
of the set of vertices of $P$.

(In particular, this means that if the vertices of $P$ all belong to
$\{0,1\}^n$, then so do the vertices of $Q$.)

\item
Given two extreme points $a$ and $b$ of a polyhedron $P$, we say that they are
{\it adjacent} if the line segment between them forms an edge (i.e. a face
of dimension 1) of the polyhedron $P$. This can be rephrased by saying
that $a$ and $b$ are adjacent on $P$ if and only if there exists a
cost function $c$ such that $a$ and $b$ are the only two extreme
points of $P$ minimizing $c^Tx$ over $P$.

Consider the polyhedron (polytope) $P$ defined as the convex hull of
all perfect matchings in a (not necessarily bipartite) graph $G$. Give
a necessary and sufficient condition for two matchings $M_1$ and $M_2$
to be adjacent on this polyhedron (hint: think about $M_1
\bigtriangleup M_2=(M_1\setminus M_2) \cup (M_2\setminus M_1)$) and
prove that your condition is necessary and sufficient.)

\item Show that two vertices $u$ and $v$ of a polytope $P$ are
  adjacent if and only there is a unique way to express their midpoint
  ($\frac{1}{2}(u+v)$) as a convex combination of vertices of $P$.

\end{exercises}

\section{Polyhedral Combinatorics}

In one sentence, polyhedral combinatorics deals with the study of
polyhedra or polytopes associated with discrete sets arising from
combinatorial optimization problems (such as matchings for example).
If we have a discrete set $X$ (say the incidence vectors of matchings
in a graph, or the set of incidence vectors of spanning trees of a
graph, or the set of incidence vectors of {\it stable} sets\footnote{A
set $S$ of vertices in a graph $G=(V,E)$ is stable if there are no
edges between any two vertices of $S$.} in a graph), we can consider
$\conv(X)$ and attempt to describe it in terms of linear
inequalities. This is useful in order to apply the machinery of linear
programming. However, in some (most) cases, it is actually hard to
describe the set of all inequalities defining $\conv(X)$; this occurs
whenever optimizing over $X$ is hard and this statement can be made
precise in the setting of computational complexity. For matchings, or
spanning trees, and several other structures (for which the
corresponding optimization problem is polynomially solvable), we will
be able to describe their convex hull in terms of linear inequalities.

Given a set $X$ and a proposed system of inequalities $P=\{x: Ax \leq
b\}$, it is usually easy to check whether $\conv(X) \subseteq
P$. Indeed, for this, we only need to check that every member of $X$
satisfies every  inequality in the description of
$P$. The reverse inclusion is more difficult. Here are 3 general
techniques to prove that $P \subseteq \conv(X)$ (if it is true!) (once
we know that $\conv(X) \subseteq P$).

\begin{enumerate}
\item
{\bf Algorithmically.} This involves linear programming duality. This
is what we did in the notes  about the assignment problem (minimum
weight matchings in bipartite graphs). In general, consider any cost
function $c$ and consider the combinatorial optimization problem of
maximizing $c^Tx$ over $x\in X$. We know that:
\begin{eqnarray*} 
\max\{c^T x: x\in X\} & = & \max\{c^Tx: x\in \conv(X)\}\\ &  \leq  & \max\{c^T x:
Ax \leq b\} \\
& =  &\min \{b^T y: A^Ty=c, y\geq 0\},
\end{eqnarray*} the last equality
coming from strong duality. If we can exhibit a solution $x\in X$
(say the incidence vector of a perfect matching in the assignment problem) and a dual feasible
solution $y$ (values $u_i$, $v_j$ in the assignment problem) such that
$c^Tx=b^Ty$ we will have shown that we have equality throughout, and
if this is true for {\it any} cost function $c$, this implies that  $P =
\conv(X)$. 

This is usually the most involved approach but also the one that works
most often. 

\item {\bf Focusing on extreme points.} Show first that $P=\{x: Ax
  \leq b\}$ is bounded (thus a polytope) and then study its extreme
  points. If we can show that every extreme point of $P$ is in $X$
  then we would be done since $P=\conv(ext(P))\subseteq \conv(X)$, where
  $ext(P)$ denotes the extreme points of $P$ (see Theorem
  \ref{boundedconv}). The assumption that $P$ is bounded is needed to
  show that indeed $P=\conv(ext(P))$ (not true if $P$ is unbounded).

In the case of the convex hull of bipartite matchings, this can be
done easily and this leads to the notion of {\it totally unimodular}
Matrices (TU), see the next section. 

\item
{\bf Focusing on the facets of $\conv(X)$.} This leads usually to the
shortest and cleanest proofs. Suppose that our proposed $P$ is of the
form $\{x\in \R^n: Ax\leq b, Cx=d\}$. We have already argued
that $\conv(X) \subseteq P$ and we want to show that $P\subseteq
\conv(X)$. 

First we need to show that we are not missing any equality. This can
be done for example by showing that $\dim(\conv(X))=\dim(P)$. We already know that $\dim(\conv(X))\leq \dim(P)$ (as $\conv(X)\subseteq P$), and so we need to argue that $\dim(conv(X))\geq \dim(P)$. This means showing that 
if there are $n-d$ linearly independent rows in
$C$ we can find $d+1$ affinely independent points in $X$.

Then we need to show that we are not missing a valid inequality that
induces a {\it facet} of $\conv(X)$. Consider any valid inequality
$\alpha^T x\leq \beta$ for $\conv(X)$ with $\alpha\neq 0$. We can
assume that $\alpha$ is any vector in $\R^n\setminus \{0\}$ and that
$\beta=\max\{\alpha^T x: x\in \conv(X)\}$. The face of $\conv(X)$ this
inequality defines is $F=\conv(\{x\in X: \alpha^T x=\beta\})$. Assume
that this is a non-trivial face; this will happen precisely when
$\alpha$ is not in the row space of $C$. We need to make sure that if
$F$ is a facet then we have in our description of $P$ an inequality
representing it. What we will show is that if $F$ is non-trivial then
we can find an inequality $a_i^T x\leq b_i$ in our description of $P$
such that (i) $F\subseteq \{x: a_i^T x=b_i\}$ and (ii) $a_i^Tx\leq
b_i$ defines a non-trivial face of $P$ (this second condition is not
needed if $P$ is full-dimensional), or simply that every optimum
solution to $\max\{\alpha^T x: x\in X\}$ satisfies $a_i^Tx=b_i$, and
that this equality is not satisfied by all points in $P$. This
means that if $F$ was a facet, by maximality, we have a representative
of $F$ in our description.

This is a very simple and powerful technique, and this is best
illustrated on an example. 

\paragraph{Example.} Let $X=\{(\sigma(1), \sigma(2), \cdots,
\sigma(n)): \sigma$ is a permutation of $\{1,2,\cdots, n\}\}$. 
We claim that 
\begin{eqnarray*} 
  \conv(X)=\{x\in \R^n: & \sum_{i=1}^n x_i ={n+1 \choose 2} \\
  & \sum_{i\in S} x_i \geq {|S|+1 \choose 2} & S \subset \{1,\cdots,
  n\} \}.
\end{eqnarray*}
This is known as the {\it permutahedron}.

Here $\conv(X)$ is not full-dimensional; we only need to show that we
are not missing any facets and any equality in the description of
$\conv(P)$. For the equalities, this can be seen easily as it is easy
to exhibit $n$ affinely independent permutations in $X$. For the
facets, suppose that $\alpha^T x\leq \beta$ defines a non-trivial
facet $F$ of $\conv(X)$. Consider maximizing $\alpha^T x$ over all
permutations $x$. Let $S=\arg\min\{\alpha_i\}$; by our assumption that
$F$ is non-trivial we have that $S\neq \{1,2,\cdots,n\}$ (otherwise,
we would have the equality $\sum_{i=1}^n x_i ={n+1 \choose
  2}$). Moreover, it is easy to see (by an exchange argument) that any
permutation $\sigma$ whose incidence vector $x$ maximizes $\alpha^Tx$
will need to satisfy $\sigma(i)\in\{1,2,\cdots,|S|\}$ for $i\in S$, in
other words, it will satisfy the inequality $\sum_{i \in S} x_i \geq
{|S|+1 \choose 2}$ at equality (and this is a non-trivial face as
there exist permutations that do not satisfy it at equality). Hence,
$F$ is contained in a non-trivial face corresponding to an inequality
in our description, and hence our description contains inequalities
for all facets. This is what we needed to prove.  That's it!

\end{enumerate}     

\subsection*{Exercises}
\begin{exercises}
\item Consider the set $X=\{(\sigma(1), \sigma(2), \cdots, \sigma(n)):
  \sigma$ is a permutation of $\{1,2\cdots,n\}\}$. Show that
  $\dim(\conv(X))=n-1$. (To show that $\dim(\conv(X))\geq n-1$, exhibit
  $n$ affinely independent permutations $\sigma$ (and prove that they
  are affinely independent).)

\item
A {\it stable set} $S$ (sometimes, it is called also an independent
set) in a graph $G=(V,E)$ is a set of vertices such that there are no
edges between any two vertices in $S$. If we let $P$ denote the convex
hull of all (incidence vectors of) stable sets of $G=(V,E)$, it is clear
that $x_i + x_j \leq 1$ for any edge $(i,j)\in E$ is a valid
inequality for $P$. 

\begin{enumerate}
\item
Give a graph $G$ with no isolated vertices for which $P$ is {\it not} equal to 
\begin{eqnarray*} 
\{x\in \R^{|V|}: &  x_i + x_j \leq 1 & \mbox{ for all } (i,j)\in E \\
 &  x_i \geq 0 & \mbox{ for all } i \in V\} 
\end{eqnarray*}

\item
Show that if the graph $G$ is bipartite and has no isolated vertices then $P$ equals
\begin{eqnarray*} 
\{x\in \R^{|V|}: &  x_i + x_j \leq 1 & \mbox{ for all } (i,j)\in E \\
 &  x_i \geq 0 & \mbox{ for all } i \in V\}. 
\end{eqnarray*}
\end{enumerate}

\item
Let $e_k\in \R^n$ ($k=0,\ldots,n-1$) be a vector with the first $k$
entries being $1$, and the following $n-k$ entries being  $-1$. Let
$S=\{e_0, e_1, \ldots, e_{n-1}, -e_0, -e_1,\ldots, -e_{n-1}\}$, i.e.\
$S$ consists of all vectors consisting of $+1$ followed by $-1$ or
vice versa. In this problem set, you will study $\conv(S)$. 
\begin{enumerate}
\item
Consider any vector $a\in\{-1,0,1\}^n$ such that (i) $\sum_{i=1}^n
a_i=1$ and (ii) for all $j=1,\ldots,n-1$, we have $0\leq \sum_{i=1}^j
a_i\leq 1$. (For example, for $n=5$, the vector $(1,0,-1,1,0)$
satisfies these conditions.)
Show that $\sum_{i=1}^n a_ix_i\leq 1$ and $\sum_{i=1}^n
a_ix_i\geq -1$ are valid inequalities for $\conv(S)$.
\item
How many such inequalities are there?
\item
Show that any such inequality defines a facet of $\conv(S)$. 

(This can be done in several ways. Here is one approach, but you are
welcome to use any other one as well. First show that either $e_k$ or
$-e_k$ satisfies this inequality at equality, for any $k$. Then show that the
resulting set of vectors on the hyperplane are affinely independent (or uniquely identifies it).)
\item
Show that the above inequalities define the entire convex hull of $S$. 

(Again this can be done in several ways. One possibility is to
consider the 3rd technique described above.)
\end{enumerate}

\end{exercises}


     
\section{Total unimodularity} \label{total}

\begin{definition}
A matrix $A$ is totally unimodular (TU) if every square submatrix of $A$
has determinant $-1, 0$ or $+1$.
\end{definition}

The importance of total unimodularity stems from the following
theorem. This theorem gives a subclass of integer programs which are
easily solved. A polyhedron $P$ is said to be {\it integral}
if all its vertices or extreme points are integral (belong to $\Z^n$).
\begin{theorem} \label{tum}
Let $A$ be a totally unimodular matrix. Then, for any {\it integral}
right-hand-side $b$, the polyhedron $$P=\{x: Ax\leq b, x\geq 0\}$$ is
integral.
\end{theorem}

Before we prove this result, two remarks can be made. First, the proof
below will in fact show that the same result holds for the polyhedrons
$\{x: Ax\geq b, x\geq 0\}$ or $\{x: Ax=b, x\geq 0\}$.  In the latter
case, though, a slightly weaker condition than totally unimodularity is
sufficient to prove the result. Secondly, in the above theorem, one
can prove the converse as well: If $P=\{x: Ax\leq b, x\geq 0\}$ is
integral for all integral $b$ then $A$ must be totally unimodular
(this is not true though, if we consider for example $\{x: Ax=b, x\geq 0\}$).

\begin{proof}
Adding slacks, we get the polyhedron $Q=\{(x,s): Ax+ Is=b, x\geq 0,
s\geq 0\}$. One can easily show (see exercise below) that $P$ is
integral iff $Q$ is integral. 

Consider now any bfs of $Q$. The basis
$B$ consists of some columns of $A$ as well as some columns of the
identity matrix $I$. Since the columns of $I$ have only one nonzero
entry per column, namely a one, we can expand the determinant of $A_B$
along these entries and derive that, in absolute values, the
determinant of $A_B$ is equal to the determinant of some square
submatrix of $A$. By definition of totally unimodularity, this implies
that the determinant of $A_B$ must belong to $\{-1,0,1\}$. By definition
of a basis, it cannot be equal to 0. Hence, it must be equal to $\pm
1$. 

We now prove that the bfs must be integral. The non-basic variables,
by definition, must have value zero. The vector of basic variables, on
the other hand, is equal to $A_B^{-1} b$. From linear algebra, $A_B^{-1}$
can be expressed as $$\frac{1}{\det A_B} A_B^{adj}$$ where $A_B^{adj}$ is
the adjoint (or adjugate) matrix of $A_B$ and consists of subdeterminants of $A_B$.
Hence, both $b$ and $A_B^{adj}$ are integral  which implies that
$A_B^{-1}b$ is integral since $|\det A_B|=1$. This proves the integrality
of the bfs. 
\end{proof}

\begin{exercises}
\item
Let $P=\{x: Ax\leq b, x\geq 0\}$ 
and let $Q=\{(x,s): Ax+ Is=b, x\geq 0, s\geq 0\}$. Show that $x$ is an
extreme point of $P$ iff $(x,b-Ax)$ is an extreme point of $Q$.
Conclude that whenever $A$ and $b$ have only integral entries, $P$ is
integral iff $Q$ is integral.
\end{exercises}

In the case of the bipartite matching problem, the constraint matrix
$A$ has a very special structure and we show below that it is totally
unimodular.  This together with Theorem \ref{tum} proves Theorem 1.6
from the notes on the bipartite matching problem. First, let us
restate the setting. Suppose that the
bipartition of our bipartite graph is $(U,V)$ (to avoid any confusion
with the matrix $A$ or the basis $B$). Consider 
\lps
 &     P   & =  &\{x:  &   \sum_j x_{ij} =1 & i\in U \\
  &        &   &  &   \sum_i x_{ij} =1 & j\in V \\
  &        &   &  &  x_{ij}\geq 0 & i\in U, j\in V\} \\
 & & = & \{x: & Ax=b, x\geq 0\}.
\elps
\begin{theorem}
The matrix $A$ is totally unimodular.
\end{theorem}

The way we defined the matrix $A$ corresponds to a {\it complete}
bipartite graph. If we were to consider any bipartite graph then we
would simply consider a submatrix of $A$, which is also totally
unimodular by definition.  

\begin{proof}
Consider any square submatrix $T$ of $A$. We consider three cases.
First, if $T$ has a column or a row with all
entries equal to zero then the determinant is zero. Secondly, if there exists a
column or a row of $T$ with only one $+1$ then by expanding the
determinant along that $+1$, we can consider a smaller sized matrix
$T$. The last case is when $T$ has at least two nonzero entries per
column (and per row). Given the special structure of $A$, there must
in fact be {\it exactly} two nonzero entries per column. By adding up
the rows of $T$ corresponding to the vertices of $U$ and adding up the
rows of $T$ corresponding to the vertices of $V$, one therefore
obtains the same vector which proves that the rows of $T$ are linearly
dependent, implying that its determinant is zero. This proves the
total unimodularity of $A$.
\end{proof}

%We conclude with a technical remark. One should first remove one of the rows of $A$ before applying Theorem \ref{tum} since, as such, it does not have full row rank and this fact was implicitly used in the definition of a bfs. However, deleting a row of $A$ still preservesits totally unimodularity.

\begin{exercises}
\item
If $A$ is totally unimodular then
$A^T$ is totally unimodular. 
\item
Use total unimodularity to prove
K\"onig's theorem.
\end{exercises}

The following theorem gives a necessary and sufficient condition for a
matrix to be totally unimodular. 

\begin{theorem}
Let $A$ be a $m\times n$ matrix with entries in $\{-1,0,1\}$. Then $A$
is TU if and only if for all subsets $R\subseteq \{1,2,\cdots,n\}$ of
rows, there exists a partition of $R$ into $R_1$ and $R_2$ such that
for all $j\in\{1,2,\cdots,m\}$: 
$$\sum_{i\in R_1} a_{ij} - \sum_{i\in R_2} a_{ij} \in\{0,1,-1\}.$$
\end{theorem}

We will prove only the {\it if} direction (but that is the most important as this allows to prove that a matrix is totally unimodular). 

\begin{proof}
Assume that, for every $R$, the desired partition exists. We need to prove
that the determinant of any $k\times k$ submatrix of $A$ is in
$\{-1,0,1\}$, and this must be true for any $k$. Let us prove it by
induction on $k$. It is trivially true for $k=1$. Assume it is true
for $k-1$ and we will prove it for $k$.

Let $B$ be a $k\times k$ submatrix of $A$, and we can assume that $B$
is invertible (otherwise the determinant is 0 and there is nothing to
prove). The inverse $B^{-1}$ can be written as $\frac{1}{\det(B)}
B^*$, where all entries of $B^*$ correspond to $(k-1)\times (k-1)$
submatrices of $A$. By our inductive hypothesis, all entries of $B^*$
are in $\{-1,0,1\}$. Let $b^*_1$ be the first row of $B*$ and $e_1$ be
the $k$-dimensional row vector $[1\; 0\; 0\cdots 0]$, thus $b^*_1=e_1
B^*$. By the relationship between $B$ and $B^*$, we have that 
\begin{equation} \label{eqbb}
b^*_1
B = e_1 B^* B = \det(B) e_1 B^{-1} B = \det(B) e_1.
\end{equation}

Let $R=\{i: b^*_{1i}\in \{-1,1\}\}$. By assumption, we know that there
exists a partition of $R$ into $R_1$ and $R_2$ such that for all $j$:
\begin{equation}
\label{eqbb2}
\sum_{i\in R_1} b_{ij} - \sum_{i\in R_2} b_{ij} \in\{-1,0,1\}.
\end{equation}
From (\ref{eqbb}), we have that  
\begin{equation}
\label{eqbb3}
\sum_{i\in R} b^*_{1i} b_{ij} =\left\{ \begin{array}{ll} \det(B) &
  j=1 \\ 0 & j\neq 1 \end{array} \right.
\end{equation}
Since $\sum_{i\in R_1} b_{ij} - \sum_{i\in R_2} b_{ij}$ and
$\sum_{i\in R} b^*_{1i} b_{ij}$ differ by a multiple of 2 for each $j$ (since
$b^*_{1i}\in\{-1,1\}$), this implies that  
\begin{equation}
\label{eqbb4}
\sum_{i\in R_1} b_{ij} - \sum_{i\in R_2} b_{ij} =0 \;\;\; j \neq 1.
\end{equation}
For $j=1$, we cannot get 0 since otherwise $B$ would be singular (we would get exactly the 0 vector by
adding and subtracting rows of $B$). Thus,
$$\sum_{i\in R_1} b_{i1} - \sum_{i\in R_2} b_{i1} \in\{-1,1\}.$$
If we define $y\in \R^k$ by 
$$y_i=\left\{\begin{array}{ll} 1 & i\in R_1 \\ -1 & i\in R_2 \\ 0 &
otherwise \end{array} \right.$$
we get that $yB=\pm e_1$. Thus $$y=\pm e_1 B^{-1} = \pm \frac{1}{\det
  B} e_1 B^*= \pm \frac{1}{\det B} b^*_1,$$ which implies that $\det
B$ must be either 1 or -1.
\end{proof} 

\begin{exercises}
\item Suppose we have $n$ activities to choose from. Activity $i$
  starts at time $t_i$ and ends at time $u_i$ (or more precisely just
  before $u_i$); if chosen, activity $i$ gives us a profit of $p_i$
  units. Our goal is to choose a subset of the activities which do not
  overlap (nevertheless, we can choose an activity that ends at $t$
  and one that starts at the same time $t$) and such that the total
  profit (i.e. sum of profits) of the selected activities is maximum.

\begin{enumerate}
\item
Defining $x_i$ as a variable that represents whether activity $i$ is
selected ($x_i=1$) or not ($x_i=0$), write an integer program of the
form $\max\{ p^T x: Ax \leq b, x\in\{0,1\}^n\}$ that
would solve this problem.  
\item
Show that the matrix $A$ is totally unimodular, implying that one can
solve this problem by solving the linear program 
  $\max\{ p^T x: Ax \leq b, 0\leq x_i\leq 1$ for every $i\}$.
\end{enumerate}

\item %% Not in notes posted on website in 2009 
Given a bipartite graph $G$ and given an integer $k$, let $S_k$ be the
set of all incidence vectors of  matchings with at most $k$ edges. We
are interested in finding a description of $P_k=\conv(S_k)$ as a system of
linear inequalities. More precisely, you'll show that $\conv(S_k)$ is
given by:
$$\begin{array}{lll} 
P_k=\{x: & \sum_{j} x_{ij} \leq 1 & \forall i\in A \\
  & \sum_{i} x_{ij} \leq 1 & \forall j\in B \\
& \sum_{i} \sum_j x_{ij} \leq k \\
& x_{ij} \geq 0 & i\in A, j\in B \}.
\end{array}
$$
Without the last constraint, we have shown that the
resulting matrix is totally unimodular.
\begin{enumerate}
\item
With the additional constraint, is the resulting matrix totally
unimodular? Either prove it or disprove it. 
\item
Show that $P_k$ indeed equals $\conv(S_k)$. 

\item
Suppose now that instead of a cardinality constraint on all the edges,
our edges are partitioned into $E_1$ and $E_2$ and we only impose that
our matching has at most $k$ edges from $E_1$ (and as many as we'd
like from $E_2$). Is it still true that the convex hull of all such
matchings is given by simply replacing $\sum_i \sum_j x_{ij} \leq k$
by 
$$\sum_i \sum_{j: (i,j)\in E_1} x_{ij} \leq k ?$$ 
\end{enumerate}
\end{exercises}

\section{Matching polytope}
In this section, we provide an illustration of the power of the third technique to prove a complete description of a combinatorial polytope. Consider the matching polytope, the convex hull of all incidence vectors of matchings in a graph $G=(V,E)$. If the graph is bipartite, we have seen that the degree constraints are sufficient to provide a complete description of the polytope, but this is not the case for non-bipartite graphs. We also need the blossom constraints, which for any set $S\subseteq V$ with $|S|$ odd says that
$$\sum_{e\in E(S)} x_e \leq \frac{|S|-1}{2},$$ where $E(S)$ denotes the edges of $E$ with both endpoints within $S$. These inequalities are clearly valid inequalities for all matchings as any matching can use at most $(|S|-1)/2$ edges from $E(S)$. 
But there are also sufficient:

\begin{theorem}[Edmonds] \label{thm:ed}
Let $X$ be the set of incidence vectors of matchings in $G=(V,E)$. Then $conv(X)=P$ where
\lps
&     P   & =  &\{x:  &   \sum_{e\in \delta(v)} x_e \leq 1 & v\in V \\
&        &   &  &   \sum_{e\in E(S)} x_e \leq \frac{|S|-1}{2}  & S\subseteq V, |S| \mbox{ odd} \\
&        &   &  &  x_{e}\geq 0 & e\in E\},
\elps
where $\delta(v)$ denotes the edges incident to $v$ and $E(S)$ denotes the edges with both endpoints in $S$.
\end{theorem}
  
  Here is a proof based on the third technique, showing that we are not missing any facets. 
\begin{proof}
First, it is clear that $conv(X)\subseteq P$. Also $dim(conv(X))=|E|$ since we can easily find $|E|+1$ affinely independent points in $X$ (and thus in $conv(X)$): just take the matchings consisting of a single edge and the empty matching. 
Therefore we are not missing any equality in our description. 

Now consider any valid inequality $\alpha^T x\leq \beta$ which is valid for all matchings: for any matching $M$, we have $\sum_{e\in M} \alpha_e \leq \beta$. We need to show that the face $F$ induced by this inequality is contained in the face defined by one of the inequalities in our proposed description $P$. This would mean that we have in our description $P$ an inequality for each facet of $conv(X)$. Consider the {\it extremal} matchings defined by the valid inequality $\alpha^T x\leq \beta$:
$$R=\{x\in X: \alpha^T x=\beta\}.$$ If $R$ is empty that the face is the trivial face and there is nothing to prove. So, we assume that $R\neq \emptyset$. 

Consider the following three cases. 
\begin{enumerate}
	\item[Case 1.] If there exists $e$ with $\alpha_e<0$ then for all $x\in R$, we must have $x_e=0$. Thus 
	$$\{x\in X: \alpha^T x=\beta\} \subseteq \{x\in \conv(X): \alpha^Tx=\beta\}\subseteq \{x\in P: x_e=0\},$$ and therefore this face $F$ is included in the face defined by the inequality $x_e\geq 0$ in our description $P$. Thus, in what remains, we can assume that $\alpha_e\geq 0$ for all $e\in E$. 
	\item[Case 2.] If there exists $v\in V$ such that $\forall x\in R: \sum_{e\in \delta(v)} x_e=1$ then this face $F$ is included in the face induced by $ \sum_{e\in \delta(v)} x_e \leq 1$ which is part of $P$. Thus, in what remains, we can assume that for every $v\in V$, there exists an extremal matching $M_v$ not covering $v$.
	\item[Case 3.] Let $E_+=\{e\in E: \alpha_e>0\}$. Thus (after case 1), we have $\alpha_e=0$ for all $e\in E\setminus E_+$. Let $V_+=V(E_+)$ be the vertex set of $E_+$, and consider any connected component $(V_1,E_1)$ of $V_+,E_+)$. We will show that the face $F$ is a subset of the face induced by the blossom constraint for $V_1$. 
	
	We first claim that there are no extremal matchings missing (i.e. not covering) two vertices $u,v\in V_1$. Let us assume otherwise.  Among all extremal matchings missing at least 2 vertices $u$, $v$ of $V_1$, let  $M$ and $u, v$ be such that the distance between $u$ and $v$ within $(V_1,E_1)$ is minimized. If this distance is 1 then $(u,v)\in E_1$ and $M\cup \{(u,v)\}$  would violate the inequality since $\alpha_{uv}>0$. Thus, the distance is at least 2. Let $w\notin\{u,v\}$ be on a shortest path from $u$ to $v$; thus, the distances between $w$ and both $u$ and $v$ is smaller than the distance between $u$ and $v$. Now, following case 2, we know that there exists an extremal matching $M_w$ missing $w$. Consider $M_w\triangle M$. Since this is the symmetric difference of two matchings, this contains alternating paths and cycles, and since $w$ has degree 1 in this subgraph, $M_w\triangle M$ must contain a path $Q$ with $w$ as an endpoint. Let $M_1=M\triangle Q$ and $M_2=M_w\triangle Q$. $M_1$ and $M_2$ are two matchings and therefore $\sum_{e\in M_1} \alpha_e\leq \beta$ and $\sum_{e\in M_2} \alpha_e \leq \beta$. But we also have $$
	\sum_{e\in M_1} \alpha_e + \sum_{e\in M_2} \alpha_e = \sum_{e\in M} \alpha_e + \sum_{e\in M_w} \alpha_e = 2\beta,$$ and therefore both $M_1$ and $M_2$ are also extremal. $M_1$ doesn't cover $w$, and also does not cover either $u$ or $v$ (as $Q$ only has two endpoints, one of which is $w$). This is a contradiction with our choice of $M$, $u$ and $v$ to minimize the distance between $u$ and $v$. 
	
	By the claim, no extremal matching misses more than one vertex of $V_1$. Moreover, any extremal matching that misses one vertex of $V_1$ (and these exist) cannot use any edge of $\delta(V_1)$ since these edges have $\alpha_e=0$ and thus the removal of such an $e$ would give another extremal matching which would then miss more than one vertex of $V_1$, a contradiction. Thus the existence of  an extremal matching like $M_v$ that misses $v\in V_1$ implies that $|V_1|-1$ is even. We claim that every extremal matching $M$ has precisely $(|V_1|-1)/2$ edges from $E_1$. If this was not the case, removing the edges from $M\setminus E_+$ would give another extremal matching that misses more than one vertex in $V_1$, a contradiction. Thus we have shown that every extremal matching $M$ satisfies $|M\cap E_1|=(|V_1|-1)/2$ and therefore all extremal matchings belong to the face induced by
\[
\sum_{e\in E(V_1)} x_e \leq \frac{|V_1|-1}{2}.
\]
	
 \end{enumerate}
\end{proof}

Since the vertices of a face of a polyhedron are vertices of the original polyhedron, we can deduce from Theorem \ref{thm:ed} that a complete description of the {\it perfect matching} polytope is obtained by simply replacing the degree inequalities $\sum_{e\in \delta(v)} x_e \leq 1$ by equalities: $\sum_{e\in \delta(v)} x_e = 1$. 

\end{document}

