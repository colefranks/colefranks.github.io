\documentclass[12pt]{article}
%\usepackage{psfig}
\usepackage{../lp}
\usepackage{amsmath}
% Cross-references for handout numbers.
\usepackage{amsfonts}
%\usepackage{amsthm}
\usepackage{hyperref}
\usepackage{amssymb}
%\usepackage[capitalize]{cleveref}
\usepackage{xcolor}

%\input{handouts}

\newcounter{chapnum}

\newtheorem{definition}{Definition}[chapnum]
\newtheorem{remark}{Remark}[chapnum]
\newtheorem{theorem}{Theorem}[chapnum]
\newtheorem{lemma}[theorem]{Lemma}
\newtheorem{corollary}[theorem]{Corollary}
\newtheorem{proposition}[theorem]{Proposition}
\newtheorem{claim}[theorem]{Claim}
\newtheorem{observation}{Observation}[chapnum]

\renewcommand{\thesection}{\arabic{chapnum}.\arabic{section}}
\renewcommand{\thefigure}{\arabic{chapnum}.\arabic{figure}}


\newenvironment{proof}{\noindent{\bf Proof:} \hspace*{1em}}{
        \hspace*{\fill} $\triangle$ }
\newenvironment{proof_of}[1]{\noindent {\bf Proof of #1:}
        \hspace*{1em} }{\hspace*{\fill} $\triangle$ }
\newenvironment{proof_claim}{\begin{quotation} \noindent}{
        \hspace*{\fill} $\diamond$ \end{quotation}}
\newenvironment{solution}{\noindent{\bf Solution:} \hspace*{1em}}{
        \hspace*{\fill} $\triangle$ }


\newcommand{\R}{{\mathbb R}}
\newcommand{\Z}{{\mathbb Z}}
\newcommand{\Q}{{\mathbb Q}}
\newcommand{\C}{{\mathbb C}}
\newcommand{\N}{{\mathbb N}}
\newcommand{\lin}{\operatorname{lin}}
\newcommand{\aff}{\operatorname{aff}}
\newcommand{\cone}{\operatorname{cone}}
\newcommand{\conv}{\operatorname{conv}}
\newcommand{\vol}{\operatorname{vol}}
\newcommand{\poly}{\operatorname{poly}}




\newcommand{\CF}[1]{{\color{purple}[CF: #1]}}


\newlength{\toppush}
\setlength{\toppush}{2\headheight}
\addtolength{\toppush}{\headsep}

\newcommand{\htitle}[2]{\noindent\vspace*{-\toppush}\newline\parbox{6.5in}
{Massachusetts Institute of Technology \hfill 18.453: Combinatorial Optimization 
\newline
\textbf{Instructor:} Cole Franks \quad \textbf{Notes: }Michel Goemans and Zeb Brady \hfill#2\newline
\mbox{}\hrulefill\mbox{}}\vspace*{1ex}\mbox{}\newline
\begin{center}{\Large\bf #1}\end{center}}

\newcommand{\handout}[2]{\thispagestyle{empty}
 \markboth{ #1 \hfil #2}{ #1 \hfil #2}
 \pagestyle{myheadings}\htitle{#1}{#2}}


\setlength{\oddsidemargin}{0pt}
\setlength{\evensidemargin}{0pt}
\setlength{\textwidth}{6.5in}
\setlength{\topmargin}{0in}
\setlength{\textheight}{8.5in}


\newcounter{exercisenum}
\newcounter{exercisetot}
\setcounter{exercisetot}{0}



\newenvironment{exercises}{
	\begin{list}{{\bf Exercise \arabic{chapnum}-\arabic{exercisenum}. \hspace*{0.5em}}}
	{\setlength{\leftmargin}{0em}
	 \setlength{\rightmargin}{0em}
	 \setlength{\labelwidth}{0em}
	 \setlength{\labelsep}{0em}
	\usecounter{exercisenum}
      \setcounter{exercisenum}{\theexercisetot}}}{\setcounter{exercisetot}{\theexercisenum}\end{list}}


\newenvironment{pseudocode}{
    \begin{list}{}{
        \renewcommand{\makelabel}{$\triangleright$}
        \setlength{\topsep}{0pt}
        \setlength{\leftmargin}{32pt}
        \setlength{\labelwidth}{14pt}
        \setlength{\labelsep}{0mm}
        \setlength{\itemindent}{0mm}
        \setlength{\itemsep}{-3pt}
        \setlength{\itemsep}{0mm}
        \setlength{\parsep}{0pt}%
        \setlength{\listparindent}{0pt}
    }
}
{
    \end{list}
}

\newcommand{\rank}{\operatorname{rank}}
\newcommand{\spa}{\operatorname{span}}
%\newcommand{\conv}{\operatorname{conv}}

\setcounter{chapnum}{5}
\newcommand{\F}{{\mathbb F}}


\begin{document}
\handout{\arabic{chapnum}. Matroid optimization}{\today}

\section{Definition of a Matroid}
{\bf Matroids} are combinatorial structures that generalize the notion
of linear independence in matrices. There are many equivalent
definitions of matroids, we will use one that focus on its {\it
independent sets}. A matroid $M$ is defined on a finite ground set $E$
(or $E(M)$ if we want to emphasize the matroid $M$) and a collection
of subsets of $E$ are said to be {\it independent}. The family of
independent sets is denoted by ${\cal I}$ or ${\cal I}(M)$, and we
typically refer to a matroid $M$ by listing its ground set and its
family of independent sets: $M=(E,{\cal I})$. For $M$ to be a matroid,
${\cal I}$ must satisfy two main axioms:
\begin{enumerate}
\item[$(P_1)$] if $X\subseteq Y$ and $Y\in {\cal I}$ then $X\in {\cal
  I}$,
\item[$(P_2)$] if $X\in {\cal I}$ and $Y \in {\cal I}$ and $|Y|>|X|$
  then $\exists e\in Y\setminus X:$ $X\cup\{e\}\in {\cal I}$.
\end{enumerate}
In words, the second axiom says that if $X$ is independent and there
exists a larger independent set $Y$ then $X$ can be extended to a larger
independent by adding an element of $Y\setminus X$. Axiom $(P_2)$
implies that every {\it maximal} (inclusion-wise) independent set is
maximum; in other words, all  maximal independent sets have the same
cardinality. A maximal independent set is called a {\it base} of the matroid. 
  

\paragraph{Examples.} 
\begin{itemize}
\item One trivial example of a matroid $M=(E,{\cal I})$ is a {\bf
    uniform} matroid in which
$${\cal I}=\{X\subseteq E: |X|\leq k\},$$ for a given $k$. It is
usually denoted as $U_{k,n}$ where $|E|=n$. A base is any set of
cardinality $k$ (unless $k>|E|$ in which case the only base is $|E|$).

A {\bf free} matroid is
one in which all sets are independent; it is $U_{n,n}$. 
\item
Another is a {\bf partition} matroid in which $E$ is partitioned into
(disjoint) sets $E_1, E_2, \cdots, E_l$ and $${\cal I}=\{X\subseteq E:
|X\cap E_i| \leq k_i \mbox{ for all } i=1,\cdots,l\},$$ for some given
parameters $k_1, \cdots, k_l$. As an exercise, let us check that
$(P_2)$ is satisfied. If $X, Y\in {\cal I}$ and $|Y|>|X|$, there must
exist $i$ such that $|Y\cap E_i|>|X\cap E_i|$ and this means that
adding any element $e$ in $E_i\cap (Y\setminus X)$ to $X$ will
maintain independence.

Observe that $M$ would {\it not} be a matroid if the sets $E_i$ were {\it
  not} disjoint. For example, if $E_1=\{1,2\}$ and $E_2=\{2,3\}$ with
  $k_1=1$ and $k_2=1$ then both $Y=\{1,3\}$ and $X=\{2\}$ have at most
  one element of each $E_i$, but one can't find an element of $Y$ to
  add to $X$. 

\item
{\bf Linear} matroids (or representable matroids) are defined from a matrix $A$, and this is
where the term {\it matroid} comes from. Let $E$ denote the index set
of the columns of $A$. For a subset $X$ of $E$, let $A_X$ denote the
submatrix of $A$ consisting only of those columns indexed by $X$. Now,
define $${\cal I}=\{X\subseteq E: \rank(A_X)=|X|\},$$
i.e.\ a set $X$ is independent if the corresponding columns are
linearly independent. A base $B$ corresponds to a linearly independent
set of columns of cardinality $\rank(A)$.  

Observe that $(P_1)$ is trivially satisfied, as
if columns are linearly independent, so is a subset of them. $(P_2)$
is less trivial, but corresponds to a fundamental linear
algebra property. If $A_X$ has full column rank, its columns span a
space of dimension $|X|$, and similarly for $Y$, and therefore if
$|Y|>|X|$, there must exist a column of $A_Y$ that is not in the span
of the columns of $A_X$; adding this column to $A_X$ increases the rank by 1. 

A linear matroid can be defined over any field $\F$ (not just the
reals); we say that the matroid is {\bf representable over $\F$}. If
the field is $\F_2$ (field of 2 elements with operations $\pmod 2$)
then the matroid is said to be {\bf binary}. If the field is $\F_3$
then the matroid is said to be {\bf ternary}.

For example,
the binary matroid corresponding to the matrix $$A=\left[
\begin{array}{ccc}
1 & 1 & 0 \\
1 & 0 & 1 \\
0 & 1 & 1 
\end{array}\right]$$ corresponds to $U_{2,3}$ since the sum of the 3
columns is the 0 vector when taking components modulo 2. If $A$ is viewed
over the reals or over $\F_3$ then the matroid is the free matroid on 3
elements. 

Not every matroid is linear. Among those that are linear, some can be
represented over some fields $\F$ but not all. For example, there are
binary matroids which are not ternary and vice versa (for example,
$U_{2,4}$ is ternary but not binary). Matroids which
can be represented over {\it any} field are called {\bf regular}. One
can show that regular matroids are precisely those linear matroids that can
be represented over the reals by a totally unimodular marix. (Because
of this connection, a deep
result of Seymour provides a polynomial-time algorithm for deciding
whether a matrix is TU.)

\item Here is an example of something that is not a matroid. Take a
  graph $G=(V,E)$, and let ${\cal I}=\{F\subseteq E: F $ is a
  matching$\}$. This is not a matroid since $(P_2)$ is not necessarily
  satisfied ($(P_1)$ is satisfied\footnote{When $(P_1)$ alone is
    satisfied, $(E,{\cal I})$ is called an {\it independence system.}},
    however). Consider, for example, a graph on 4 vertices and let
    $X=\{(2,3)\}$ and $Y=\{(1,2),(3,4)\}$. Both $X$ and $Y$ are
    matchings, but one cannot add an edge of $Y$ to $X$ and still have
    a matching.
\item
There is, however, another matroid associated with matchings in a
(general, not necessarily bipartite) graph $G=(V,E)$, but this time
the ground set of $M$ corresponds to $V$. In the {\bf matching
matroid}, ${\cal I}=\{S \subseteq V: S$ is covered by some matching
$M\}$. In this definition, the matching does not need to cover
precisely $S$; other vertices can be covered as well.
\item
A very important class of matroids in combinatorial optimization is
the class of {\bf graphic} matroids (also called cycle
matroids). Given a graph $G=(V,E)$, we define independent sets to be
those subsets of edges which are forests, i.e.\ do not contain any
cycles. This is called the graphic matroid $M=(E,\cal I)$, or $M(G)$. 

$(P_1)$ is clearly satisfied. To check $(P_2)$, first notice that if
$F$ is a forest then the number of connected components of the graph
$(V,F)$ is given by $\kappa(V,F)=|V|-|F|$. Therefore, if $X$ and $Y$ are 2
forests and $|Y|>|X|$ then $\kappa(V,Y)<\kappa(V,X)$ and therefore there must
exist an edge of $Y\setminus X$ which connects two different
connected components of $X$; adding this edge to $X$ results in a larger
forest. This shows $(P_2)$. 

If the graph $G$ is connected, any base will correspond to a spanning
tree $T$ of the graph. If the original graph is disconnected then a
base corresponds to taking a spanning tree in each connected component
of $G$.   

A graphic matroid is a linear matroid. We first show that the field
$\F$ can be chosen to be the reals. Consider the matrix $A$ with a row
for each vertex $i\in V$ and a column for each edge $e=(i,j)\in E$. In
the column corresponding to $(i,j)$, all entries are 0, except for a
$1$ in $i$ or $j$ (arbitrarily) and a $-1$ in the other. To show
equivalence between the original matroid $M$ and this newly
constructed linear matroid $M'$, we need to show that any independent
set for $M$ is independent in $M'$ and vice versa.  
This is left as an exercise.

In fact, a graphic matroid is {\it regular}; it can be represented
over any field $\F$. In fact the above matrix $A$ can be shown to be
TU.  To obtain a representation for a field $\F$, one simply needs to
take the representation given above for $\R$ and simply view/replace
all $-1$ by the additive inverse of $1$.

\end{itemize}

\subsection{Circuits}
A minimal (inclusionwise) dependent set in a matroid is called a {\it
  circuit}. In a graphic matroid $M(G)$, a circuit will be the usual notion
of a {\it cycle} in the graph $G$; to be dependent in the graphic matroid,
one needs to contain a cycle and the minimal sets of edges containing
a cycle are the cycles themselves. In a partition matroid, a circuit
will be a set $C\subseteq E_i$ for some $i$ with $|C|=k_i+1$. 

By definition of a circuit $C$, we have that if we remove any element
of a circuit then we get an independent set. A crucial property of
circuit is given by the following property,

\begin{theorem}[Unique Circuit Property] \label{uniquecir}
  Let $M=(E, {\cal I})$ be a matroid. Let $S\in {\cal I}$ and $e$ such
  that\footnote{For a set $S$ and an element $e$, we often write $S+e$
    for $S\cup\{e\}$ and $S-e$ for $S\setminus \{e\}$.} $S+e\notin
  {\cal I}$. Then there exists a {\it unique} circuit $C\subseteq
  S+e$.
\end{theorem}

The uniqueness is very important. Indeed, if we consider any $f\in C$
where $C$ is this unique circuit then we have that $S+e-f\in {\cal
  I}$. Indeed, if $S+e-f$ was dependent, it would contain a circuit
$C'$ which is distinct from $C$ since $f\notin C'$, a contradiction.

As a special case of the theorem, consider a graphic matroid. If we
add an edge to a forest and the resulting graph has a cycle then it
has a unique cycle.

\begin{proof}

Suppose $S+e$ contains more than one circuit, say $C_1$ and $C_2$ with
$C_1\neq C_2$. By minimality of $C_1$ and $C_2$, we have that there
exists $f\in C_1\setminus C_2$. Since
$C_1-f\in {\cal I}$ (by minimality of the circuit $C_1$), we can
extend it to a maximal independent set $X$ of $S+e$. Since $S$ is
also independent, we must have that $|X|=|S|$ and since $e\in C_1-f$, we
must have that $X=S+e-f\in {\cal I}$. But this means that
$C_2\subseteq S+e-f=X$ which is a contradiction since $C_2$ is
dependent. 
\end{proof}


\begin{exercises}
\item
Show that any partition matroid is also a linear matroid over $\F=\R$. (No need to
give a precise matrix $A$ representing it; just argue its existence.) 
\item
Prove that a matching matroid is indeed a matroid. 
\item
Show that $U_{2,4}$ is representable over $\F_3$.
\item
Consider the linear matroid (over the reals) defined by the $3\times
5$ matrix:
$$A=\left(\begin{array}{ccccc}
1 & 2 & 1 & 0 & 1 \\
1 & 2 & 0 & 1 & -1 \\
1 & 2 & 0 & 1 & -1 
\end{array}\right).$$
The ground set $E=\{1,2,3,4,5\}$ has cardinality 5, corresponds to the
columns of $A$, and the independent sets are the set of columns which
are linearly independent (over the reals). 
\begin{enumerate}
\item
Give all bases of this matroid.
\item
Give all circuits of this matroid. 
\item 
Choose a base $B$ and an element $e$ not in $B$, and verify the unique
circuit property for $B+e$.
\end{enumerate}

\item
Given a family $A_1, A_2, \cdots, A_n$ of sets (they are not
necessarily disjoint), a {\it transversal} is a set $T$ such that
$T=\{a_1, a_2, \cdots, a_n\}$, the $a_i$'s are distinct, and $a_i\in
A_i$ for all $i$. A partial transversal is a transversal for $A_{i_1},
A_{i_2}, \cdots, A_{i_k}$ for some subfamily of the $A_i$'s. 

Show that the family of all partial transversals forms a matroid (on
the ground set $E=\cup A_i$). (Hint: Think of bipartite matchings.)  

\item \label{ex:truncated}
Let $M=(E,{\cal I})$ be a matroid. Let $k\in \N$ and define $${\cal
  I}_k=\{X\in {\cal I}: |X|\leq k \}.$$ Show that $M_k=(E,{\cal I}_k)$
  is also a matroid. This is known as a truncated matroid. 

\item \label{ex:laminar}
A family ${\cal F}$ of sets is said to be {\it laminar} if, for any
two sets $A, B\in {\cal F}$, we have that either (i) $A\subseteq B$,
or (ii) $B\subseteq A$ or (iii) $A\cap B=\emptyset$. Suppose that we
have a laminar family ${\cal F}$ of subsets of a finite set $E$ and an integer $k(A)$ for
every set $A\in {\cal F}$. Show that $(E, {\cal I})$ defines a matroid
(a {\it laminar} matroid) where:
$${\cal I} =\{X\subseteq E: |X\cap A|\leq k(A) \mbox{ for all } A \in
{\cal F}\}.$$

\end{exercises}

\section{Matroid Optimization}

Given a matroid $M=(E, {\cal I})$ and a cost function $c:E\rightarrow
\R$, we are interested in finding an independent set $S$ of $M$ of
maximum total cost $c(S)=\sum_{e\in S} c(e)$. This is a fundamental
problem. 

If all $c(e)\geq 0$, the problem is equivalent to finding a
maximum cost {\it base} in the matroid. If $c(e)<0$ for some element
$e$ then, because of $(P_1)$, $e$ will not be contained in any optimum
solution, and thus we could eliminate such an element from the ground
set.  In the special case of a graphic matroid $M(G)$ defined on a
connected graph $G$, the problem is thus equivalent to the maximum
spanning tree problem which can be solved by a simple greedy
algorithm. This is actually the case for any matroid and this is the
topic of this section. 

The greedy algorithm we describe actually returns, for every $k$, a
set $S_k$ which maximizes $c(S)$ over all independent sets of size
$k$. The overall optimum can thus simply be obtained by outputting the
best of these. The greedy algorithm is the following: 

\begin{pseudocode}
\item Sort the elements (and renumber them) such that $c(e_1)\geq
  c(e_2)\geq \cdots \geq c(e_{|E|})$
\item $S_0=\emptyset$, k=0
\item For $j=1$ to $|E|$
\begin{pseudocode}
\item
 if $S_{k}+e_j \in {\cal I}$ then
\begin{pseudocode}
\item
$k \leftarrow k+1$
\item
$S_k \leftarrow S_{k-1} + e_j$ 
\item $s_k \leftarrow e_j $
\end{pseudocode}
\end{pseudocode}
\item Output $S_1$, $S_2, \cdots, S_k$
\end{pseudocode}

\begin{theorem}
For any matroid $M=(E,{\cal I})$, the greedy algorithm above finds,
for every $k$,  an independent set $S_k$ of maximum cost among all
independent sets of size $k$.
\end{theorem}


\begin{proof}
Suppose not. Let $S_k=\{s_1, s_2, \cdots, s_k\}$ with $c(s_1)\geq
c(s_2)\geq \cdots \geq c(s_k)$, and suppose $T_k$ has greater cost
($c(T_k)>c(S_k)$) where $T_k=\{t_1, t_2, \cdots, t_k\}$ with
$c(t_1)\geq c(t_2)\geq \cdots \geq c(t_k)$. Let $p$ be the first index
such that $c(t_p)>c(s_p)$. Let $A=\{t_1, t_2, \cdots, t_p\}$ and
$B=\{s_1, s_2, \cdots, s_{p-1}\}$. Since $|A|>|B|$, there exists
$t_i\notin B$ such that $B+t_i\in {\cal I}$. Since $c(t_i)\geq
c(t_p)>c(s_p)$, $t_i$ should have been selected when it was
considered. To be more precise and detailed, when $t_i$ was
considered, the greedy algorithm checked whether $t_i$ could be added
to the current set at the time, say $S$. But since $S\subseteq B$,
adding $t_i$ to $S$ should have resulted in an independent set (by
$(P_1)$) since its addition to $B$ results in an independent set. This
gives the contradiction and completes the proof.
\end{proof}

Observe that, as long as $c(s_k)\geq 0$, we have that $c(S_k)\geq
c(S_{k-1})$. Therefore, to find a maximum cost set over all
independent sets, we can simply replace the loop
\begin{pseudocode}
\item For $j=1$ to $|E|$
\end{pseudocode}
by  
\begin{pseudocode}
\item For $j=1$ to $q$
\end{pseudocode}
where $q$ is such that $c(e_q)\geq 0 >c(e_{q+1})$, and output
the last $S_k$. 

For the maximum cost spanning tree problem, the greedy algorithm
reduces to Kruskal's algorithm which considers the edges in
non-increasing cost and add an edge to the previously selected edges
if it does not form a cycle. 

One can show that the greedy algorithm actually characterizes
matroids. If $M$ is an independence system, i.e. it satisfies $(P_1)$,
then $M$ is a matroid if and only if the greedy algorithm finds a
maximum cost set of size $k$ for every $k$ and every cost function. 

\begin{exercises}
\item
We are given $n$ jobs that each take one unit of
processing time. All jobs are available at time 0, and job $j$ has a
profit of $c_j$ and a deadline $d_j$. The profit for job $j$ will only
be earned if the job completes by time $d_j$. The problem is to find
an ordering of the jobs that maximizes the total profit. First, prove
that if a subset of the jobs can be completed on time, then they can
also be completed on time if they are scheduled in the order of their
deadlines. Now, let $E(M)=\{1,2,\cdots,n\}$ and let ${\cal
I}(M)=\{J\subseteq E(M): J$ can be completed on time $\}$. Prove that
$M$ is a matroid and describe how to find the most profitable subset of jobs that can be completed.
\end{exercises}


\section{Rank Function of a Matroid}

Similarly to the notion of rank for matrices, one can define a rank
function for any matroid. The rank function of $M$, denoted by either
$r(\cdot)$ or $r_M(\cdot)$, is defined by:
$$r_M:2^E \rightarrow \N: r_M(X)=\max\{|Y|: Y\subseteq X, Y\in {\cal
  I}\}.$$

Here are a few specific rank functions:
\begin{itemize}
\item
For a linear matroid, the rank of $X$ is precisely the rank in the
  linear algebra sense of the matrix $A_X$ corresponding to the
  columns of $A$ in $X$. 
\item
For a partition matroid $M=(E,{\cal I})$ where $${\cal I}=\{X\subseteq
  E: |X\cap E_i|\leq k_i \mbox{ for } i=1,\cdots, l\}$$ (the $E_i$'s
  forming a partition of $E$) its rank
  function is given by:
$$ r(X)=\sum_{i=1}^l \min(|E_i\cap X|, k_i).$$
\item
For a graphic matroid $M(G)$ defined on graph $G=(V,E)$, the rank
function is equal to: 
$$r_{M(G)}(F)= n - \kappa(V,F),$$ where $n=|V|$ and $\kappa(V,F)$ denotes the
number of connected components (including isolated vertices) of the
graph with edges $F$.
\end{itemize}

The rank function of any matroid $M=(E,{\cal I})$ has the following properties:
\begin{enumerate}
\item[$(R_1)$] $0\leq r(X)\leq |X|$ and is integer valued for all
  $X\subseteq E$
\item[$(R_2)$] $X\subseteq Y \Rightarrow r(X) \leq r(Y)$,
\item[$(R_3)$] $r(X)+r(Y) \geq r(X\cap Y) + r(X \cup Y)$.
\end{enumerate}
The last property is called {\it submodularity} and is a key concept in
combinatorial optimization. It is clear that, as defined, any rank
function satisfies $(R_1)$ and $(R_2)$. Showing that the rank function
satisfies submodularity needs a proof. 

\begin{lemma}
The rank function of any matroid is submodular.
\end{lemma}

\begin{proof}
Consider any two sets $X, Y\subseteq E$. Let $J$ be a maximal
independent subset of $X\cap Y$; thus, $|J|=r(X\cap Y)$. By $(P_2)$,
$J$ can be extended to a maximal (thus maximum) independent subset of
$X$, call it $J_X$. We have that $J\subseteq J_X \subseteq X$ and
$|J_X|=r(X)$. Furthermore, by maximality of $J$ within $X\cap Y$, we
know \begin{equation} \label{ma:eq1}
J_X \setminus Y = J_X \setminus J.
\end{equation}
 Now extend $J_X$ to a maximal independent set $J_{XY}$
of $X\cup Y$. Thus, $|J_{XY}|=r(X \cup Y)$. 

In order to be able to
prove that 
$$r(X)+r(Y)\geq r(X\cap Y) + r(X \cup Y)$$
or equivalently
$$|J_X|+r(Y) \geq |J|+|J_{XY}|,$$
we need to show that $r(Y)\geq |J|+|J_{XY}|-|J_X|$. Observe that
$J_{XY}\cap Y$ is independent (by $(P_1)$) and a subset of $Y$, and
thus $r(Y)\geq |J_{XY}\cap Y|$. Observe now 
that $$J_{XY}\cap Y = J_{XY}\setminus (J_X\setminus Y) =J_{XY}\setminus
(J_X\setminus J),$$
the first equality following from the fact that $J_X$ is a maximal
independent subset of $X$ and the second equality by
(\ref{ma:eq1}). Therefore, $$   r(Y)\geq |J_{XY}\cap Y| = |J_{XY}\setminus
(J_X\setminus J)|= |J_{XY}|-|J_X|+|J|,$$ proving the lemma.
\end{proof}

\subsection{Span}
The following definition is also motivated by the linear algebra
setting. 

\begin{definition} \label{def:span}
Given a matroid $M=(E, {\cal I})$ and given $S \subseteq E$,
let $$\spa(S)=\{e\in E: r(S\cup\{e\})=r(S)\}.$$
\end{definition}
The span of a set is also called the closure. Observe that $S\subseteq
\spa(S)$. We claim that $r(S)=r(\spa(S))$; in other words, if adding
an element to $S$ does not increase the rank, adding many such
elements also does not increase the rank. Indeed, take a maximal
independent subset of $S$, say $J$. If $r(\spa(S))>|J|$ then there
exists $e\in \spa(S)\setminus J$ such that $J+e\in {\cal I}$. Thus
$r(S+e)\geq r(J+e)=|J|+1>|J|=r(S)$ contradicting the fact that $e\in
\spa(S)$.

\begin{definition} \label{def:closed}
A set $S$ is said to be closed if $S=\spa(S)$. (Some authors use the word ``flat'' or ``subspace'' instead of ``closed''.)
\end{definition}

\begin{exercises}
\item
Given a matroid $M$ with rank function $r$ and given an integer
$k\in\N$, what is the rank function of the truncated matroid $M_k$
(see Exercise \arabic{chapnum}-\ref{ex:truncated} for a definition)?
\item
What is the rank function of a laminar matroid, see exercise \arabic{chapnum}-\ref{ex:laminar}?
\end{exercises}

\section{Matroid Polytope}

Let $$X=\{\chi(S)\in\{0,1\}^{|E|}: S\in {\cal I}\}$$ denote the
incidence (or characteristic) vectors of all independent sets of a
matroid $M=(E,{\cal I})$, and let the {\it matroid polytope} be
defined as $\conv(X)$. In this section, we provide a complete
characterization of $\conv(X)$ in terms of linear inequalities. In
addition, we illustrate the different techniques proposed in the
polyhedral chapter for proving a complete description of a polytope. 

\begin{theorem} \label{matpol}
Let $$\begin{array}{rll} 
P = \{x\in \R^{|E|}: & x(S) \leq r(S) & \forall S\subseteq E \\
& x_e \geq 0 & \forall e\in E\}
\end{array}$$
where $x(S):=\sum_{e\in S} x_e$.
Then $\conv(X)=P$. 
\end{theorem}

It is clear that $\conv(X) \subseteq P$ since $X\subseteq P$. The
harder part is to show that $P\subseteq \conv(X)$. In the next three
subsections, we provide three different proofs based on the three
techniques to prove complete polyhedral descriptions. 

\subsection{Algorithmic Proof}

Here we provide an algorithmic proof based on the greedy
algorithm. From $\conv(X) \subseteq P$, we know that 
$$\begin{array}{lcll}
\max\{c^T x: x\in X\} = \max\{c^Tx: x\in \conv(X)\} & \leq 
\max\{c^Tx: & 
x(S) \leq r(S) & S\subseteq E \\
& & x_e \geq 0 & e\in E\}.
\end{array}
$$
Using LP duality, we get that this last expression equals:
$$\begin{array}{lcll}
\min\{\sum_S r(S) y_S: & \sum_{S:
  e\in S} y_S \geq c(e) & \forall e\in E \\
& y_S\geq 0 & S\subseteq E\}.
\end{array}
$$
Our goal now is, for any cost function $c$, to get an independent set
$S$ and a dual feasible solution $y$ such that $c^T \chi(S) = \sum_S
r(S) y_S$ which proves that $\conv(X)=P$. 

Consider any cost function $c$. We know that the maximum cost
independent set can be obtained by the greedy algorithm. More
precisely, it is the last set $S_k$ returned by the greedy algorithm
when we consider only those elements up to $e_q$ where $c(e_q)\geq
0\geq c(e_{q+1})$. We need now to exhibit a dual solution of the same
value as $S_k$. There are exponentially many variables in
the dual, but this is not a problem. In fact, we will set most of
them to 0. 

For any index $j\leq k$, we have $S_j=\{s_1, s_2, \cdots, s_j\}$, and
we define $U_j$ to be all elements in our ordering up to and excluding
$s_{j+1}$, i.e.\ $U_j=\{e_1, e_2, \cdots, e_l\}$ where
$e_{l+1}=s_{j+1}$.  In other words, $U_j$ is all the elements in the
ordering just before $s_{j+1}$. One important property
of $U_j$ is that $$r(U_j)=r(S_j)=j.$$ Indeed, by independence
$r(S_j)=|S_j|=j$, and by $(R_2)$, $r(U_j)\geq r(S_j)$. If
$r(U_j)>r(S_j)$, there would be an element say $e_p\in U_j\setminus
S_j$ such that $S_j\cup\{e_p\}\in {\cal I}$. But the greedy algorithm
would have selected that element (by $(P_1)$) contradicting the fact
that $e_p\in U_j\setminus S_j$.

Set the non-zero entries of $y_S$ in the following way. For
$j=1,\cdots, k$, let 
$$y_{U_j}=c(s_j)-c(s_{j+1}),$$where it is understood that
$c(s_{k+1})=0$. By the ordering of the $c(\cdot)$, we have that
$y_S\geq 0$ for all $S$. In addition, for any $e\in E$, we have that 
$$\sum_{S: e\in S} y_S = \sum_{j=t}^k y_{U_j} = c(s_t) \geq c(e),$$
where $t$ is the least index such that $e\in U_t$ (implying that $e$
does not come before $s_t$ in the ordering). This shows that $y$ is a
feasible solution to the dual. Moreover, its dual value is:
$$\sum_S r(S) y_S = \sum_{j=1}^k r(U_j) y_{U_j} = \sum_{j=1}^k j
(c(s_j)-c(s_{j+1})) = \sum_{j=1}^k (j-(j-1)) c(s_j)= \sum_{j=1}^k
c(s_j) = c(S_k).$$ This shows that the dual solution has the same
value as the independent set output by the greedy algorithm, and this
is true for all cost functions. This completes the algorithmic proof.

\subsection{Vertex Proof}
Here we will focus on any vertex $x$ of
$$\begin{array}{rll} 
P = \{x\in \R^{|E|}: & x(S) \leq r(S) & \forall S\subseteq E \\
& x_e \geq 0 & \forall e\in E\}
\end{array}$$
and show that $x$ is an integral vector. Since $x(\{e\})\leq
r(\{e\})\leq 1$, we get that $x\in \{0,1\}^{|E|}$ and thus it is the
incidence vector of an independent set. 

Given any $x\in P$, consider the {\it tight} sets $S$, i.e. those sets
for which $x(S)=r(S)$. The next lemma shows that these tight sets are
closed under taking intersections or unions. This lemma is really
central, and follows from submodularity.

\begin{lemma} \label{uncross}
Let $x\in P$. Let $${\cal F}=\{S\subseteq E: x(S)=r(S)\}.$$
Then $$S\in {\cal F}, T\in {\cal F} \Rightarrow S\cap T\in {\cal F},
S\cup T\in {\cal F}.$$
\end{lemma}
Observe that the lemma applies even if $S$ and $T$ are disjoint. In
that case, it says that $\emptyset\in {\cal F}$ (which is always the
case as $x(\emptyset)=0=r(\emptyset)$) and $S\cup T\in {\cal F}$. 

\begin{proof}
The fact that $S, T\in {\cal F}$ means that:
\begin{equation} \label{ma:e1}
r(S) + r(T)= x(S) + x(T).
\end{equation}
Since $x(S)=\sum_{e\in S} x_e$, we have that 
\begin{equation} \label{ma:e2}
x(S)+x(T) = x(S\cap T) + x(S \cup T),
\end{equation}
i.e.\ that the function
$x(\cdot)$ is modular (both $x$ and $-x$ are submodular).
Since $x\in P$, we know that $x(S\cap T) \leq r(S \cap T)$ (this is
true even if $S\cap T=\emptyset$) and
similarly $x(S \cup T) \leq r(S\cup T)$; this implies that 
\begin{equation} \label{ma:e3}
x(S\cap T) + x(S\cup T) \leq r(S\cap T) + r(S\cup T).
\end{equation}
By submodularity, we have that   
\begin{equation} \label{ma:e4}
r(S\cap T) + r(S\cup T) \leq r(S) + r(T).
\end{equation}
Combining (\ref{ma:e1})--(\ref{ma:e4}), we get 
$$r(S) + r(T)= x(S) + x(T) = x(S\cap T) + x(S \cup T)  \leq r(S\cap T)
+ r(S\cup T)  \leq r(S) + r(T),$$
and therefore we have equality throughout. This implies that $x(S\cap
T)=r(S\cap T)$ and $x(S\cup T) =r(S\cup T)$, i.e. $S\cap T$ and $S\cup
T$ in ${\cal F}$.  
\end{proof}

To prove that any vertex or extreme point of $P$ is integral, we first
characterize any face of $P$. A {\it chain} ${\cal C}$ is a
family of sets such that for all $S, T\in {\cal C}$ we have that
either $S\subseteq T$ or $T\subseteq S$ (or both if $S=T$). 
\begin{theorem}
Consider any face $F$ of $P$. Then there exists a chain ${\cal C}$ and
a subset $J\subseteq E$ such that:
$$\begin{array}{rll} 
F = \{x\in \R^{|E|}: & x(S) \leq r(S) & \forall S\subseteq E \\
& x(C) =r(C) & \forall C\in {\cal C} \\
& x_e \geq 0 & \forall e\in E\setminus J \\
& x_e = 0 & \forall e\in J.\}
\end{array}$$
\end{theorem}

\begin{proof}
By Theorem 3.5 of the polyhedral notes, we know that any face is
characterized by setting some of the inequalities of $P$ by
equalities. In particular, $F$ can be expressed as 
$$\begin{array}{rll} 
F = \{x\in \R^{|E|}: & x(S) \leq r(S) & \forall S\subseteq E \\
& x(C) =r(C) & \forall C\in {\bf {\cal F}} \\
& x_e \geq 0 & \forall e\in E\setminus J \\
& x_e = 0 & \forall e\in J.\}
\end{array}$$
where $J=\{e: x_e=0$ for all $x\in F\}$ and ${\cal F}=\{S: x(S)=r(S)$
for all $x\in F\}$. To prove the theorem, we need to argue that the
system of equations:
$$\begin{array}{rll} 
& x(C) =r(C) & \forall C\in {\cal F}
\end{array}$$ 
can be replaced by an equivalent (sub)system in which ${\cal F}$ is
replaced by a chain ${\cal C}$. To be equivalent, we need that
$$\spa({\cal F})=\spa({\cal C})$$
where by $\spa({\cal L})$ we mean 
$$\spa({\cal L}):=\spa\{\chi(C): C\in {\cal L}\}.$$

Let ${\cal C}$ be a maximal subchain of ${\cal F}$, i.e. ${\cal
  C}\subseteq {\cal F}$, ${\cal C}$ is a chain and for all $S\in {\cal
  F}\setminus {\cal C}$, there exists $C\in {\cal C}$ such that
$S\not\subseteq C$ and $C\not\subseteq S$. We claim that $\spa({\cal
  C})=\spa({\cal F})$.

Suppose not, i.e. $H\neq \spa({\cal F})$ where
$H:=\spa({\cal C})$. This means that there exists $S\in {\cal
  F}\setminus {\cal C}$ such that $\chi(S)\notin H$ but $S$ cannot be
added to ${\cal C}$ without destroying the chain structure. In other
words, for any such $S$, the set of 'chain violations'
$$V(S):= \{C\in {\cal C}: C\not\subseteq S \mbox{ and } S\not\subseteq
C\}$$ is non-empty. Among all such sets $S$, choose one for which
$|V(S)|$ is as small as possible ($|V(S)|$ cannot be 0 since we are
assuming that $V(S)\neq \emptyset$ for all possible $S$). Now fix some
set $C\in V(S)$. By Lemma \ref{uncross}, we know that both $C\cap S\in
{\cal F}$ and $C\cup S\in {\cal F}$. Observe that there is a linear
dependence between $\chi(C), \chi(S), \chi(C\cup T), \chi(C\cap T)$:
$$ \chi(C)+ \chi(S) = \chi(C\cup S)+ \chi(C\cap S).$$
This means that, since $\chi(C)\in H$ and $\chi(S)\notin H$, we must
have that either $\chi(C\cup S)\notin H$ or $\chi(C\cap S)\notin H$
(otherwise $\chi(S)$ would be in $H$). Say that $\chi(B)\notin H$
where $B$ is either $C\cup S$ or $C\cap S$. This is a contradiction
since $|V(B)| < |V(S)|$, contradicting our choice of $S$. Indeed, one
can see that $V(B) \subset V(S)$ and $C\in V(S)\setminus V(B)$. 
\end{proof}

As a corollary, we can also obtain a similar property for an extreme
point, starting from Theorem 3.6. 

\begin{corollary}
Let $x$ be any extreme point of $P$. Then there exists a chain ${\cal C}$ and
a subset $J\subseteq E$ such that $x$ is the unique solution to:
$$\begin{array}{rll} 
& x(C) =r(C) & \forall C\in {\cal C} \\
& x_e = 0 & \forall e\in J.
\end{array}$$
\end{corollary}

From this corollary, the integrality of every extreme point follows
easily. Indeed, if the chain given in the corollary consists of
$C_1\subset C_2 \subset C_p$ the the system reduces to 
$$\begin{array}{rll} 
& x(C_{i}\setminus C_{i-1}) =r(C_{i})-r(C_{i-1}) & i=1,\cdots,p\\
& x_e = 0 & \forall e\in J,
\end{array}$$
where $C_0=\emptyset$. For this to have a unique solution, we'd better
have $|C_i\setminus C_{i-1}\setminus J|\leq 1$ for all $i$ and the
values for the resulting $x_e$'s will be integral. Since $0\leq x_e
\leq r(\{e\}) \leq 1$, we have that $x$ is a $0-1$ vector and thus
$x=\chi(S)$ for some set $S$. As $|S|\leq r(S) \leq |S|$, we have
$|S|=r(S)$ and thus $S\in {\cal I}$ and therefore $x$ is the incidence
vector of an independent set. This completes the proof. 


\subsection{Facet Proof}
Our last proof of Theorem \ref{matpol} focuses on the facets of
$\conv(X)$. 

First we need to argue that we are missing any equalities. Let's focus
on the (interesting) case in which any singleton set is independent:
$\{e\}\in {\cal I}$ for every $e\in E$. In that case
$dim(\conv(X))=|E|$ since we can exhibit $|E|+1$ affinely independent
points in $X$: the 0 vector and all unit vectors $\chi(\{e\})$ for
$e\in E$. Thus we do not need any equalities. See exercise
\arabic{chapnum}-\ref{ex:fulldim} if we are not assuming that every singleton set is
independent. 

Now consider any facet $F$ of $\conv(X)$. This facet is induced by a
valid inequality $\alpha^T x\leq \beta$ where $\beta=\max\{\sum_{e\in
  I} \alpha_e: I\in {\cal I}\}.$ Let $${\cal O}=\{I\in {\cal I}:  \sum_{e\in
  I} \alpha_e=\beta\},$$ i.e. ${\cal O}$ is the set of all independent sets
whose incidence vectors belong to the face. We'll show that there
exists an inequality in our description of $P$ which is satisfied at
equality by the incidence vectors of all sets $I\in {\cal O}$.

We consider two cases. If there exists $e\in M$ such that $\alpha_e<
0$ then $I\in {\cal O}$ implies that $e\notin I$, implying that our
face $F$ is included in the face induced by $x_e\geq 0$ (which is in
our description of $P$). 

For the other case, we assume that for all $e\in E$, we have
$\alpha_e\geq 0$. We can further assume that
$\alpha_{\max}:=\max_{e\in E}\alpha_e>0$ since otherwise $F$ is
trivial. Now, define $S$ as
$$S=\{e\in E: \alpha_e =\alpha_{\max}\}.$$ 
\begin{claim}
For any $I\in {\cal O}$, we have $|I\cap S|=r(S)$.
\end{claim}
This means that the face $F$ is contained in the face induced by the
inequality $x(S)\leq r(S)$ and therefore we have in our description of
$P$ one inequality
inducing each facet of $\conv(X)$. Thus we have a complete description
of $\conv(X)$. 

To prove the claim, suppose that $|I\cap S|<r(S)$. Thus $I\cap S$ can be extended to an
independent set $X\in {\cal I}$ where $X\subseteq S$ and $|X|>|I\cap
S|$. Let $e\in X\setminus (I\cap S)$; observe that $e\in S$ by our
choice of $X$. Since $\alpha_e>0$ we have that $I+e\notin {\cal I}$,
thus there is a circuit $C\subseteq I+e$. By the unique circuit
property (see Theorem \ref{uniquecir}), for any $f\in C$ we have
$I+e-f\in {\cal I}$. But $C\setminus S\neq \emptyset$ since $(I\cap S)
+e\in {\cal I}$, and thus we can choose $f\in C\setminus S$. The cost
of $I+e-f$ satisfies:
$$c(I+e-f)=c(I)+c(e)-c(f)>c(I),$$ contradicting the definition of
${\cal O}$.   

\section{Facets?}
Now that we have a description of the matroid polytope in terms of
linear inequalities, one may wonder which of these (exponentially
many) inequalities define facets of $\conv(X)$. 

For simplicity, let's assume that $r(\{e\})=1$ for all $e\in E$ ($e$
belongs to some independent set). Then, every nonnegativity constraint
defines a facet of $P=\conv(X)$. Indeed, the 0 vector and all unit
vectors except $\chi(\{e\})$ constitute $|E|$ affinely independent
points satisfying $x_e=0$. This mean that the corresponding face has
dimension at least $|E|-1$  and since the dimension of $P$ itself is
$|E|$, the face is a facet. 

We now consider the constraint $x(S)\leq r(S)$ for some set
$S\subseteq E$. If $S$ is not closed (see Definition \ref{def:closed})
then $x(S)\leq r(S)$ definitely does not define a facet of $P=\conv(X)$
since it is implied by the constraints $x(\spa(S))\leq r(S)$ and
$x_e\geq 0$ for $e\in \spa(S)\setminus S$. 

Another situation in which $x(S)\leq r(S)$ does not define a facet is
if $S$ can be expressed as the disjoint union of $U\neq \emptyset$ and
$S\setminus U \neq \emptyset$ and $r(U)+r(S\setminus U)=r(S)$. In
this case, the inequality for $S$ is implied by those for $U$ and for
$S\setminus U$. 
\begin{definition}
$S$ is said to be {\it inseparable} if there is no $U$ with $\emptyset
  \neq U \subset S$ such that $r(S)=r(U)+r(S\setminus U)$. 
\end{definition} 

From what we have just argued, a necessary condition for $x(S)\leq
r(S)$ to define a facet of $P=\conv(X)$ is that $S$ is {\it closed and
inseparable}. This can be shown to be sufficient as well, although the
proof is omitted. 

As an example, consider a partition matroid with $M=(E, {\cal I})$
where $${\cal I}=\{X\subseteq E: |X\cap E_i| \leq k_i \mbox{ for all }
i=1,\cdots,l\},$$ for disjoint $E_i$'s. Assume that $k_i\geq 1$ for
all $i$. The rank function for this matroid is:
$$r(S)=\sum_{i=1}^l \min(k_i, |S\cap E_i|).$$ For a set $S$ to be
inseparable, there must exist (i) $i\in\{1,\cdots, l\}$ with $S\subseteq
E_i$, and (ii) $|S\cap E_i|$ is either $\leq 1$ or $>k_i$ for every
$i$. Furthermore, for $S\subseteq E_i$ to be closed, we must have that
if $|S\cap E_i|\geq k_i$ then $S\cap E_i=E_i$. Thus the only sets we need
for the description of a partition matroid polytope are (i) sets
$S=E_i$ for $i$ with $|E_i|>k_i$ and (ii) singleton sets $\{e\}$ for
$e\in E$. The partition matroid polytope is thus given by:
$$\begin{array}{rll} 
P = \{x\in \R^{|E|}: & x(E_i) \leq k_i & i\in\{1,\cdots,l\}: |E_i|>k_i
\\
& 0\leq x_e \leq 1 & e\in E\}.
\end{array}$$

As another example, take $M$ to be the graphic matroid $M(G)$. For a
set of edges $F\subseteq E$ to be inseparable, we need that the
subgraph $(V,F)$ either is a single edge or has only one non-trivial
(i.e.\ with more than 1 vertex) 2-connected component\footnote{For example, if the graph consists of 3 cycles that are vertex-disjoint except for one common (to all three) vertex $v$ then each of these cycles would form a 2-connected component. In general, there is a forest structure on the set of 2-connected components.} (this is a
maximal set of vertices such that for any two vertices in it, there
exists two (internally) vertex-disjoint paths between them).   Indeed,
if we partition $F$ into the edge sets $F_1, \cdots, F_c$ of the ($c$
non-trivial) 2-connected components and single edges (for those edges
not in any 2-connected components), we have that $r(F)=\sum_{i=1}^c
r(F_i)$ and thus $c$ must be 1 for $F$ to be inseparable. Given a set
$F$ of edges, its span (with respect to the graphic matroid) consists
of all the edges with both endpoints within any connected
component of $F$; these are the edges whose addition does not increase
the size of the largest forest. Thus, for $F$ to be inseparable and
closed, either it must be a single edge or there exists a vertex set $S\subseteq V$
such that $F=E(S)$ ($E(S)$ denotes all the edges with both endpoints
in $S$) and $(S,E(S))$ is 2-connected. Thus (the minimal description of) the forest polytope
(convex hull of all forests in a graph $G=(V,E)$) is given by:
$$\begin{array}{rll} P = \{x\in \R^{|E|}: & x(E(S)) \leq |S|-1 &
S\subseteq V, E(S) \mbox{
  2-connected or } |S|=2\\
\\
& 0\leq x_e\leq 1 & e\in E\}.
\end{array}$$
(As usual, $x(E(S))$ denotes $\sum_{e\in E(S)} x_e$.) We could also include all vertex sets $S$; this would of course also give the spanning tree polytope, albeit not the minimal description. 
Observe that this polyhedral description still has a very large number
of inequalities, wven if we include only the facet-defining inequalities. 

From this, we can also easily derive the {\it spanning tree polytope}
of a graph, namely the convex hull of incidence vectors of all
spanning trees in a graph. Indeed, this is a face of the forest
polytope obtained by replacing the inequality for $S=V$ ($x(E) \leq
|V|-1$) by an equality:
$$\begin{array}{rll} 
P = \{x\in \R^{|E|}: & x(E) = |V|-1 \\
 & x(E(S)) \leq |S|-1 & S\subset V, E(S) \mbox{
  2-connected or }|S|=2\\
\\
& 0\leq x_e\leq 1  & e\in E\}.
\end{array}$$



\begin{exercises}
\item \label{ex:fulldim}
Let $M=(E,{\cal I})$ be a matroid and let $S=\{e\in E: \{e\}\in {\cal
  I}\}$. Show that $\dim(\conv(X))=|S|$ (where $X$ is the set of
incidence vectors of independent sets)
and show that the description for $P$ has the required number of
linearly independent equalities. 

\item
Let $M=(E, {\cal I})$ be a matroid and let $P$ be the corresponding
matroid  polytope, i.e. the convex hull of characteristic vectors of
independent sets. Show that two independent sets $P_1$ and $P_2$ are
adjacent on $P$ if and only if either (i) $P_1\subseteq P_2$ and
$|P_1|+1=|P_2|$, or (ii) $P_2\subseteq P_1$ and $|P_2|+1=|P_1|$, or
(iii)   
$|P_1\setminus P_2|=|P_2\setminus P_1|=1$ and $P_1\cup P_2\notin
{\cal I}$. 

\end{exercises}

\end{document}
