\documentclass[12pt]{article}

\usepackage{../lp,graphicx}
\usepackage{amsmath}
\usepackage{amssymb}
% Cross-references for handout numbers.
\usepackage{amsfonts}
%\usepackage{amsthm}
\usepackage{hyperref}
\usepackage{amssymb}
%\usepackage[capitalize]{cleveref}
\usepackage{xcolor}

%\input{handouts}

\newcounter{chapnum}

\newtheorem{definition}{Definition}[chapnum]
\newtheorem{remark}{Remark}[chapnum]
\newtheorem{theorem}{Theorem}[chapnum]
\newtheorem{lemma}[theorem]{Lemma}
\newtheorem{corollary}[theorem]{Corollary}
\newtheorem{proposition}[theorem]{Proposition}
\newtheorem{claim}[theorem]{Claim}
\newtheorem{observation}{Observation}[chapnum]

\renewcommand{\thesection}{\arabic{chapnum}.\arabic{section}}
\renewcommand{\thefigure}{\arabic{chapnum}.\arabic{figure}}


\newenvironment{proof}{\noindent{\bf Proof:} \hspace*{1em}}{
        \hspace*{\fill} $\triangle$ }
\newenvironment{proof_of}[1]{\noindent {\bf Proof of #1:}
        \hspace*{1em} }{\hspace*{\fill} $\triangle$ }
\newenvironment{proof_claim}{\begin{quotation} \noindent}{
        \hspace*{\fill} $\diamond$ \end{quotation}}
\newenvironment{solution}{\noindent{\bf Solution:} \hspace*{1em}}{
        \hspace*{\fill} $\triangle$ }


\newcommand{\R}{{\mathbb R}}
\newcommand{\Z}{{\mathbb Z}}
\newcommand{\Q}{{\mathbb Q}}
\newcommand{\C}{{\mathbb C}}
\newcommand{\N}{{\mathbb N}}
\newcommand{\lin}{\operatorname{lin}}
\newcommand{\aff}{\operatorname{aff}}
\newcommand{\cone}{\operatorname{cone}}
\newcommand{\conv}{\operatorname{conv}}
\newcommand{\vol}{\operatorname{vol}}
\newcommand{\poly}{\operatorname{poly}}




\newcommand{\CF}[1]{{\color{purple}[CF: #1]}}


\newlength{\toppush}
\setlength{\toppush}{2\headheight}
\addtolength{\toppush}{\headsep}

\newcommand{\htitle}[2]{\noindent\vspace*{-\toppush}\newline\parbox{6.5in}
{Massachusetts Institute of Technology \hfill 18.453: Combinatorial Optimization 
\newline
\textbf{Instructor:} Cole Franks \quad \textbf{Notes: }Michel Goemans and Zeb Brady \hfill#2\newline
\mbox{}\hrulefill\mbox{}}\vspace*{1ex}\mbox{}\newline
\begin{center}{\Large\bf #1}\end{center}}

\newcommand{\handout}[2]{\thispagestyle{empty}
 \markboth{ #1 \hfil #2}{ #1 \hfil #2}
 \pagestyle{myheadings}\htitle{#1}{#2}}


\setlength{\oddsidemargin}{0pt}
\setlength{\evensidemargin}{0pt}
\setlength{\textwidth}{6.5in}
\setlength{\topmargin}{0in}
\setlength{\textheight}{8.5in}


\newcounter{exercisenum}
\newcounter{exercisetot}
\setcounter{exercisetot}{0}



\newenvironment{exercises}{
	\begin{list}{{\bf Exercise \arabic{chapnum}-\arabic{exercisenum}. \hspace*{0.5em}}}
	{\setlength{\leftmargin}{0em}
	 \setlength{\rightmargin}{0em}
	 \setlength{\labelwidth}{0em}
	 \setlength{\labelsep}{0em}
	\usecounter{exercisenum}
      \setcounter{exercisenum}{\theexercisetot}}}{\setcounter{exercisetot}{\theexercisenum}\end{list}}


\newenvironment{pseudocode}{
    \begin{list}{}{
        \renewcommand{\makelabel}{$\triangleright$}
        \setlength{\topsep}{0pt}
        \setlength{\leftmargin}{32pt}
        \setlength{\labelwidth}{14pt}
        \setlength{\labelsep}{0mm}
        \setlength{\itemindent}{0mm}
        \setlength{\itemsep}{-3pt}
        \setlength{\itemsep}{0mm}
        \setlength{\parsep}{0pt}%
        \setlength{\listparindent}{0pt}
    }
}
{
    \end{list}
}


\setcounter{chapnum}{4}
\newcommand{\F}{{\mathbb F}}


\begin{document}
\handout{\arabic{chapnum}. Lecture notes on flows and cuts}{\today}

\section{Maximum Flows}

Network flows deals with modeling the flow of a commodity (water,
electricity, packets, gas, cars, trains, money, or any abstract object) in
a network. The links in the network are capacitated
and the commodity does not vanish in the network except at specified
locations where we can either inject or extract some amount of
commodity. The main question is how much can be sent in this network. 

Here is a more formal definition of the maximum flow problem. We have
a digraph (directed graph) $G=(V,E)$ and two special vertices $s$ and
$t$; $s$ is called the source and $t$ the sink.  We have an upper
capacity function $u: E \rightarrow \R$ and also a lower capacity
function $l: E \rightarrow \R$ (sometimes chosen to be 0
everywhere). A flow $x$ will be an assignment of values to the arcs
(directed edges) so that:
\begin{enumerate}
\item
for every $e\in E$: $l(e) \leq x_e \leq u(e)$,
\item
for every $v\in V\setminus \{s,t\}$:
\begin{equation} \label{eq:flowconv}
\sum_{e\in \delta^+(v)} x_e - \sum_{e\in \delta^-(v)} x_e =0.
\end{equation}
\end{enumerate}
The notation $\delta^+(v)$ represents the set of arcs {\it leaving}
$v$, while $\delta^-(v)$ represents the set of arcs {\it entering}
$v$. 

Equations (\ref{eq:flowconv}) are called {\it flow conservation} 
constraints. Given a flow $x$, its {\it flow value} $|x|$ is the net
flow out of $s$:
\begin{equation} \label{eq:flowvalue}
|x|:= \sum_{e\in \delta^+(s)} x_e - \sum_{e\in \delta^-(s)} x_e.
\end{equation}
One important observation is that $|x|$ is also equal to the net flow
into $t$, or minus the net flow out of $t$. Indeed, summing
(\ref{eq:flowconv}) over $u\in V\setminus\{s,t\}$ together with 
(\ref{eq:flowvalue}), we get:
\begin{eqnarray*}
|x|& = & \sum_{v\in V\setminus\{t\}} \left(\sum_{e\in \delta^+(v)} x_e - \sum_{e\in \delta^-(v)} x_e \right) \\
& = & \sum_{e\in \delta^-(t)} x_e - \sum_{e\in \delta^+(t)} x_e
\end{eqnarray*}
by looking at the contribution of every arc in the first summation. 

The {\it maximum flow problem} is the problem of finding a flow $x$ of
maximum value $|x|$. This is a linear program:
\lps & & & \mbox{Max} & \sum_{e\in \delta^+(s)} x_e - \sum_{e\in
  \delta^-(s)} x_e
\\ & \lefteqn{\mbox{subject to:}} 
\\ & & & & \sum_{e\in \delta^+(v)} x_e - \sum_{e\in \delta^-(v)} x_e
=0 & v\in V\setminus\{s,t\} \\
\\ & & & & l(e) \leq x_e \leq u(e) & e\in E.
\elps
We could therefore use algorithms for linear programming to find the
maximum flow and duality to derive optimality properties, but we will
show that more combinatorial algorithms can be developed and duality
translates into statements about {\it cuts}. 

In matrix form, the linear program can be written as:
$$\begin{array}{lrcl}
\max\{c^Tx: & Nx& = & 0, \\ & Ix& \leq & u, \\ & x & \geq & l\} 
\end{array}
$$
where $N$ is the (vertex-arc incidence\footnote{More precisely, part
  of it as we are not considering vertices $s$ and $t$}) 
matrix with rows indexed by
$u\in V\setminus \{s,t\}$ and columns indexed by arcs $e=(i,j)\in E$;
the entry $N_{ue}$ is:
$$N_{ue}=\left\{ \begin{array}{ll} 1 & u=i \\ -1 & u=j \\ 0 & u\notin
  \{i,j\}. \end{array}\right.$$
The constraints of the linear program are thus: $Ax \Delta b$ where
$$A = \left(\begin{array}{c} N \\ --- \\ I 
\end{array} \right),$$  and  some of the constraints in $\Delta$ are equalities
and some are inequalities. 

\begin{lemma}
$A$ is totally unimodular.
\end{lemma}






\begin{proof} We could use Theorem 3.14 from the polyhedral chapter,
but proving it directly is as easy. Consider any square submatrix of
$A$, and we would like to compute its determinant up to its sign. If
there is a row with a single $+1$ or a single $-1$ in it (in
particular, a row coming from either the identity submatrix $I$ or
$-I$), we can expand the determinant and compute the determinant (up
to its sign) of a smaller submatrix of $A$. Repeating this, we now
have a square submatrix of $N$. If there is a column with a single
$+1$ or a single $-1$ then we can expand the determinant along this
column and get a smaller submatrix. We are thus left either with an
empty submatrix in which case the determinant of the original matrix
was $+1$ or $-1$, or with a square submatrix of $N$ with precisely one
$+1$ and one $-1$ in {\it
  every} column. The rows of this submatrix are linearly dependent
since their sum is the 0 vector. Thus the determinant is 0. This
proves total unimodularity. 
\end{proof}

As a corollary, this means that if the right-hand-side (i.e. the upper
and lower capacities) are integer-valued then there always exists a
maximum flow which takes only integer values. 
\begin{corollary} \label{cor:flowint}
If $l: E\rightarrow \Z$ and $u:E \rightarrow \Z$ then there exists a
maximum flow $x$ such that $x_e\in \Z$ for all $e\in E$. 
\end{corollary}
Actually the following modification to what we know about total unimodularity is needed for the above corollary.
\paragraph{Exercise 4-0.5: } Show that if $A$ is totally unimodular then the polyhedron $\{x: Ax \Delta b, x \geq l\}$ is integral for $\ell \in \Z$, even when $\Delta$ has $\leq, \geq, =$. 


\subsection{Special cases} \label{sub:specialcases}

\paragraph{Arc-disjoint paths.} 
If $l(e)=0$ for all $e\in E$ and $u(e)=1$ for all $e\in E$, any
integer flow $x$ will only take values in $\{0,1\}$. We claim that for
an integer flow $x$, there exist $|x|$ arc-disjoint (i.e. not having
any arcs in common) paths from $s$ to $t$. Indeed, such paths can be
obtained by {\it flow decomposition}. As long as $|x|>0$, take an arc
out of $s$ with $x_e=1$. Now follow this arc and whenever we reach a
vertex $u\neq t$, by flow conservation we know that there exists an
arc leaving $u$ that we haven't traversed yet (this is true even if we
reach $s$ again). This process stops when we reach $t$ and we have
therefore identified one path from $s$ to $t$. Removing this path
gives us a new flow $x'$ (indeed flow conservation at vertices $\neq
s,t$ is maintained) with $|x'|=|x|-1$. Repeating this process gives us
$|x|$ paths from $s$ to $t$ and, by construction, they are
arc-disjoint. The paths we get might not be {\it simple}\footnote{A
  simple path is one in which no vertex is repeated.}; one can however
make them simple by removing the part of the walk between repeated
occurrences of the same vertex.  Summarizing, if $l(e)=0$ for all $e\in
E$ and $u(e)=1$ for all $e\in E$, then from a maximum flow of value
$k$, we can extract $k$ arc-disjoint (simple) paths from $s$
to $t$. Conversely, if the digraph contains $k$ arc-disjoint paths
from $s$ to $t$, it is easy to construct a flow of value $k$. This
means that the maximum flow value from $s$ to $t$ represents the
maximum number of arc-disjoint paths between $s$ and $t$. 

\paragraph{Bipartite matchings.} One can formulate the maximum
 matching problem in a bipartite graph as a maximum flow
 problem. Indeed, consider a bipartite graph $G=(V,E)$ with
 bipartition $V=A\cup B$. Consider now a directed graph $D$ with vertex
 set $V\cup \{s,t\}$. In $D$, there is an arc from $s$ to every vertex
 of $A$ with $l(e)=0$ and $u(e)=1$. There is also an arc from every
 vertex in $B$ to $t$ with capacities $l(e)=0$ and $u(e)=1$. Every
 edge $(a,b)\in E$ is oriented from $a\in A$ to $b\in B$ and gets a
 lower capacity of 0 and an upper capacity equal to $+\infty$ (or just
 1). One can easily see that from any matching of size $k$ one can
 construct a flow of value $k$; similarly to any {\it integer
 valued} flow of value $k$ corresponds a matching of size $k$. Since
 the capacities are in $\Z$, by Corollary \ref{cor:flowint}, this
 means that a maximum flow in $D$ has the same value as the maximum
 size of any matching in $G$. Observe that the upper capacities for
 the arcs between $A$ and $B$ do not matter, provided they are $\geq
 1$. 

\paragraph{Orientations.}
%In the chapter on matroid intersection, we considered 
Consider the problem of
orienting the edges of an undirected graph $G=(V,E)$ so that the
indegree of any vertex $v$ in the resulting digraph is at most
$k(v)$. This can be formulated as a maximum flow problem in which we
have (i) a vertex for every vertex of $G$, (ii) a vertex for every
edge of $G$ and (iii) 2 additional vertices $s$ and $t$. Details are
left as an exercise.

\begin{exercises}
\item
Suppose you are given an $m\times n$ matrix $A\in \R^{m\times n}$ with
row sums $r_1, \cdots, r_m\in \Z$ and column sums $c_1, \cdots, c_n\in
\Z$. Some of the entries might not be integral but the row sums and
column sums are. Show that there exists a rounded matrix $A'$ with the
following properties:
\begin{itemize}
\item
row sums and column sums of $A$ and $A'$ are identical,

\item
$a'_{ij}=\lceil a_{ij}\rceil$ or $a'_{ij}=\lfloor a_{ij} \rfloor$
  (i.e. $a'_{ij}$ is $a_{ij}$ either rounded up or down.).
\end{itemize}

By the way, this rounding is useful to the census bureau as they do
not want to publish statistics that would give too much information on
specific individuals. They want to be able to modify the entries
without modifying row and column sums. 
\end{exercises}

\section{$s$-$t$ Cuts}
In this section, we derive an important duality result for the maximum
flow problem, and as usual, this takes the form of a minmax relation. 

In a digraph $G=(V,A)$, we define a {\it cutset} or more simply a {\it
  cut} as the set of arcs $\delta^+(S)=\{(u,v)\in A: u\in S, v\in
  V\setminus S\}$. Observe that our earlier notation $\delta^+(v)$ for
  $v\in V$ rather than $\delta^+(\{v\})$ is
  a slight abuse of notation. Similarly, we define $\delta^-(S)$ as
  $\delta^+(V\setminus S)$, i.e. the arcs entering the vertex set
  $S$. We will typically identify a cutset $\delta^+(S)$ with the
  corresponding vertex set $S$. 
We say that a cut $\delta^+(S)$ is {\it an $s-t$ cut} (where $s$ and $t$ are
vertices) if $s\in S$ and $t\notin S$. 

For an undirected graph $G=(V,E)$, $\delta^+(S)$ and $\delta^-(S)$ are
identical and will be denoted by $\delta(S)=\{(u,v)\in E: |\{u,v\}\cap
S|=1\}$. Observe that now $\delta(S)=\delta(V\setminus S)$.

For a maximum flow instance on a digraph $G=(V,E)$ and upper and lower
capacity functions $u$ and $l$, we define the capacity $C(S)$ of the
cut induced by $S$ as 
$$C(S)=\sum_{e\in \delta^+(S)} u(e) -\sum_{e\in \delta^-(S)} l(e) =
u(\delta^+(S)) - l(\delta^-(S)).$$
By definition of a flow $x$, we have that $$C(S) \geq \sum_{e\in
  \delta^+(S)} x_e -\sum_{e\in \delta^-(S)} x_e. $$
We have shown earlier that the net flow out of $s$ is equal to the net
flow into $t$. Similarly, we can show that for any $S$ with $s\in S$
and $t\notin S$ (i.e. the cut induced is an $s-t$ cut), we have that the flow value $|x|$ equals:
$$|x|=\sum_{e\in \delta^+(S)} x_e -\sum_{e\in \delta^-(S)} x_e.$$ This
is shown by summing (\ref{eq:flowconv}) over $u\in S\setminus\{s\}$
together with (\ref{eq:flowvalue}). Thus, we get that for any $S$ with
$s\in S$ and $t\notin S$ and any $s-t$ flow $x$, we have:
$$|x|\leq C(S).$$ Therefore, maximizing over the $s-t$ flows and
minimizing over the $s-t$ cuts, we get 
$$\max |x| \leq \min_{S: s\in S, t\notin S} C(S).$$
This is weak duality, but in fact, one always has equality as stated
in the following theorem. Of course, we need the assumption that the
maximum flow problem is feasible. For example if there is an edge with
$l(e)>u(e)$ then no flow exists (we will show later that a necessary
and sufficient condition for the existence of a flow is that (i)
$l(e)\leq u(e)$ for every $e\in E$ and (ii) for any $S\subset V$ with
$|S\cap\{s,t\}| \neq 1$, we have $u(\delta^+(S))\geq l(\delta^-(S))$). 

\begin{theorem}[max $s-t$ flow-min $s-t$ cut] \label{thm:maxflowmincut}
  For any maximum flow problem for which a feasible flow exists, we
  have that that the maximum $s-t$ flow value is equal to the minimum
  capacity of any $s-t$ cut:
$$\max_{\text{flow } x} |x|=\min_{S: s\in S, t\notin S} C(S).$$
\end{theorem} 

One way to prove this theorem is by using strong duality of linear
programming and show that from any optimum dual solution one can
derive an $s-t$ cut of that capacity. (If we use this route, we can exploit the fact that the dual linear program will also have an integer optimum solution since $A^T$ is T.U. whenever $A$ is T.U.) 
Another way, and this is the
way we pursue, is to develop an algorithm to find a maximum
flow and show that when it terminates we have also a cut whose
capacity is equal to the flow we have constructed, therefore proving
optimality of the flow and equality in the minmax relation. 

Here is an algorithm for finding a maximum flow. Let us assume that we
are given a feasible flow $x$ (if $u(e)\geq 0$ and $l(e)\leq 0$ for
all $e$, we could start with $x=0$). Given a flow $x$, we define a
{\it residual graph} $G_x$ on the same vertex set $V$. In $G_x$, we
have an arc $(i,j)$ if (i) $(i,j)\in E$ and $x_{ij}<u((i,j))$ or if
(ii) $(j,i)\in E$ and $x_{ji}> l((j,i))$. In case (i), we say that
$(i,j)$ is a {\it forward} arc and in case (ii) it is a {\it backward} arc. If
both (i) and (ii) happen, we introduce two arcs $(i,j)$, one forward
and one backward; to be precise, $G_x$ is thus a multigraph. Consider
now any directed path $P$ from $s$ to $t$ in the residual graph; such
a path is called an {\it augmenting path}. Let $P^+$ denote the
forward arcs in $P$, and $P^-$ the backward arcs. We can modify the flow
$x$ in the following way:
$$x'_e=\left\{\begin{array}{ll} x_e + \epsilon & e\in P^+ \\ x_e
    -\epsilon & e\in P^- \\  x_e & e\notin P \end{array}\right.$$ This
is known as  pushing $\epsilon$ units of flow along $P$, or simply
augmenting along $P$. 
Observe that flow conservation at any vertex $u$ still holds when
pushing flow along a path. This is trivial if $u$ is not on the path,
and if $u$ is on the path, the contributions of the two arcs incident
to $u$ on $P$ cancel each other. To make sure the resulting  $x'$
is feasible (satisfies the capacity constraints), we choose 
$$\epsilon = \min\left(\min_{e\in P^+} (u(e)-x_e),\min_{e\in P^-}
  x_e-l(e)\right).$$ By construction of the residual graph we have
that $\epsilon>0$. Thus, pushing $\epsilon$ units of flow along an
augmenting path provides a new flow $x'$ whose value $|x'|$ satisfy
$|x'|=|x|+\epsilon$. Thus the flow $x$ was not maximum. 

Conversely, assume that the residual graph $G_x$ does not contain any
directed path from $s$ to $t$. Let $S=\{u\in V:$ there exists a
directed path in $G_x$ from $s$ to $u\}$. By definition, $s\in S$ and
$t\notin S$ (otherwise there would be an augmenting path). Also, by
definition, there is no arc in $G_x$ from $S$ to $V\setminus S$. This
means that, for $e\in E$, if $e\in \delta^+(S)$ then $x_e=u(e)$ and if
$e\in \delta^-(S)$ then $x_e=l(e)$. This implies that 
$$C(S)=\sum_{e\in \delta^+(S)} u(e) -\sum_{e\in \delta^-(S)} l(e) =
u(\delta^+(S)) - l(\delta^-(S)) = \sum_{e\in \delta^+(S)} x_e
-\sum_{e\in \delta^-(S)} x_e = |x|.$$ 
This shows that the flow $x$ is maximum and there exists an $s-t$ cut
of the same capacity as $|x|$. 

This almost proves Theorem \ref{thm:maxflowmincut}. Indeed, as long as
there exists an augmenting path, we can push flow along it, update the
residual graph and continue. Whenever this algorithm stops, {\it if
  it stops}, we have a maximum flow and a corresponding minimum
cut. But maybe this algorithm never stops; this can actually happen if
the capacities might be irrational and the ``wrong'' augmenting paths
are chosen at every iteration. For such an instance on just 6 vertices, the reader is referred to \cite{Zwick95}. 

To complete the proof of the max flow
min cut theorem, we can simply use the linear programming
formulation of the maximum flow problem and this shows that a maximum
flow exists (in a linear program, the $\max$ is a real maximum (as it
is achieved by a vertex) and not
just a supremum which may not be attained). Starting from that flow $x$
and constructing its residual graph $G_x$, we get that there exists a
corresponding minimum $s-t$ cut of the same value. 

       
 
\subsection{Interpretation of max flow min cut}  
The max $s-t$ flow min $s-t$ cut theorem together with integrality of
the maximum flow allows to derive several combinatorial min-max
relations. 

\paragraph{Bipartite matchings.} Consider for example the maximum
bipartite matching problem and its formulation as a maximum flow
problem given in section \ref{sub:specialcases}. We said that for the
arcs between $A$ and $B$ we had flexibility on how we choose $u(e)$;
here, let us assume we have set them to be equal to $+\infty$ (or any
sufficiently large integer). Consider any set $S\subseteq (\{s\}\cup
A\cup B)$ with $s\in S$ (and
$t\notin S$). For $C(S)$ to be finite there cannot be any edge
$(i,j)\in E$ between
$i\in A\cap S$ and $j\in B\setminus S$. In other words, $N(A\cap S)
\subseteq B\cap S$, i.e.\ if we set $C=(A\setminus S)\cup (B\cap S)$
we have that $C$ is a vertex cover. What is the capacity $C(S)$ of the
corresponding cut? It is precisely $C(S)=|A\setminus S| + |B\cap S|$,
the first term corresponding to the arcs from $s$ to $A\setminus S$
and the second term corresponding to the arcs between $B\cap S$ and
$t$. The max $s-t$ flow min $s-t$ cut theorem therefore implies that
there exists a vertex cover $C$ whose cardinality equals the size of
the maximum matching. We have thus rederived K\"onig's theorem. We
could also  derive Hall's theorem about the existence of a perfect
matching.

\paragraph{Arc-disjoint paths.} For the problem of the maximum number
of arc-disjoint paths between $s$ and $t$, the max $s-t$ flow min
$s-t$ cut theorem can be interpreted as Menger's theorem:

\begin{theorem}   
In a directed graph $G=(V,A)$, there are $k$ arc-disjoint paths
between $s$ and $t$ if and only if for all $S\subseteq V\setminus
\{t\}$ with $s\in S$, we have $|\delta^+(S)|\geq k$. 
\end{theorem}

\begin{exercises}
\item
At some point during baseball season, each of $n$ teams of the
American League has already played several games. Suppose team $i$ has
won $w_i$ games so far, and $g_{ij}=g_{ji}$ is the number of games
that teams $i$ and $j$ have yet to play. No game ends in a tie, so
each game gives one point to either team and 0 to the other. You would
like to decide if your favorite team, say team $n$, can
still win. In other words, you would like to determine whether there
exists an outcome to the games to be played (remember, with no ties)
such that team $n$ has at least as many victories as all the other
teams (we allow team $n$ to be tied for first place with other teams).

Show that this problem can be solved as a maximum flow problem. Give a necessary and sufficient condition on the $g_{ij}$'s so that team $n$ can still win. 

\item
Consider the following orientation problem. We
  are given an undirected graph $G=(V,E)$ and integer values $p(v)$
  for every vertex $v\in V$. we would like to know if we can orient
  the edges of $G$ such that the directed graph we obtain has at most
  $p(v)$ arcs incoming to $v$ (the ``indegree requirements''). In
  other words, for each edge $\{u,v\}$, we have to decide whether to
  orient it as $(u,v)$ or as $(v,u)$, and we would like at most $p(v)$
  arcs oriented towards $v$.
\begin{enumerate}
\item
Show that the problem can be formulated as a maximum flow
problem. That is, show how to create a maximum flow problem such that,
from its solution, you can decide whether or not the graph can be
oriented and if so, it also gives the orientation. 
\item
Consider the case that the graph cannot be oriented and meet the
indegree requirements. Prove from the max-flow min-cut theorem that
there must exist a set $S\subseteq V$ such that $|E(S)| > \sum_{v\in
  S} p(v)$, where as usual $E(S)$ denotes the set of edges with both
endpoints within $S$.
\end{enumerate} 

\end{exercises}

\section{Efficiency of Maximum Flow Algorithm}

The proof of the max $s-t$ flow min $s-t$ cut theorem suggests a
simple augmenting path algorithm for finding the maximum flow. Start
from any feasible flow and keep pushing flow along an augmenting in
the residual graph until no such augmenting path exists. The main
question we address now is how many iterations does this algorithm need
before terminating. 


As mentioned earlier, if the capacities are irrational, this algorithm
may never terminate. In the case of integral capacities, if we start
from an integral flow, it is easy to see that we always maintain an
integral flow and we will always be pushing an integral amount of
flow. Therefore, the number of iterations is bounded by the maximum
difference between the values of two flows, which is at most
$\sum_{e\in \delta(s)} (u(e)-l(e))$. This is finite, but not
polynomial in the size of the input (which depends only
logarithmically on the capacities $u$ and $l$).

%\paragraph{Fattest augmenting path variant.}

\paragraph{Shortest augmenting path variant.} Edmonds and Karp
proposed a variant of the augmenting flow algorithm which is
guaranteed to terminate in a polynomial number of iterations depending
only on $n=|V|$ and $m=|E|$. No assumptions on the capacities are
made, and the algorithm is even correct and terminates for irrational
capacities. 

The idea of Edmonds and Karp is to always find in the residual graph
a {\it shortest} augmenting path, i.e.\ one with the fewer number of
arcs. Given a flow $x$, consider the residual graph $G_x$. For any
vertex $v$, let $d(v)$ denote the distance (number of arcs) from $s$
to $v$ in $G_x$. The shortest augmenting path algorithm is to select a
path $v_0-v_1-\cdots-v_k$ in the residual graph where $v_0=s$, $v_k=t$
and $d(v_i)=i$.

The analysis of the algorithm proceeds as follows. Let $P$ be a
shortest augmenting path from $s$ to $t$ in $G_x$ and let $x'$ be the
resulting flow after pushing as much flow as possible along $P$. Let
$d'$ be the distance labels corresponding to $G_{x'}$. Observe that only reverse arcs $(i,j)$ along $P$
(thus satisfying $d(i)=d(j)+1$) may get
introduced in $G_{x'}$. Therefore, after augmentation, we have that
$d(j)-d(i)\leq 1$ for every arc $(i,j)\in E_{x'}$. Summing these
inequalities along the edges of any path $P'$ in $G_{x'}$ from $s$ to
$j\in V$, we get that
$d(j)\leq d'(j)$ for any $j\in V$. In particular, we have that
$d(t)\leq d'(t)$. As distance labels can never become greater than
$n-1$, we have that the distance to $t$  can only increase at most $n-1$ times. 
But $d'(t)$ can also be equal to $d(t)$ - we need to bound the number of iterations in which this can happen. The fact
that an arc of $P$ is saturated means that there is one fewer arc
$(i,j)$ with $d(j)=d(i)+1$ in
$G_{x'}$ than in $G_x$. Thus after at most $m$ such iterations, we
must have a strict increase in the distance label of $t$. Summarizing,
this means that the number of augmentations is at most $m(n-1)$. The
time it takes to build the residual graph and to find an augmenting
path in it is at most $O(m)$ time. This means that the total running
time of the shortest augmenting path algorithm is at most $O(m^2n)$.
This can be further improved but this is not the focus of these
notes. 

\subsection{Finding the initial feasible flow}
This augmenting path algorithm for the maximum flow problem needs an initial flow to start with. When the capacities are such that $l(e)\leq 0\leq u(e)$ for every $e\in E$, we can simply choose the initial flow to be everywhere $x_e=0$. In general, when 0 is not a feasible flow, we can find a feasible flow if one exists by solving an auxiliary maximum flow problem (for which an initial flow is easy to find). 

For this, we first get rid of $s$ and $t$ by simply adding an arc from $t$ to $s$ with $l(ts)=-\infty$ and $u(ts)=+\infty$. Define a {\it circulation} $x$ to be an assignment of values to the arcs so that (i) $l(e)\leq x_e \leq u(e)$ for every arc and (ii) flow conservation is satisfied at all vertices in $V$ (including $s$ and $t$). There is an easy bijection between feasible flows $x$ in $G$ and circulations in the digraph in which we add $(t,s)$: simply set $x_{ts}=|x|$ to be the net flow out of $s$. So this means we have reduced the problem of finding a feasible flow to the problem of finding a feasible circulation in an augmented graph (with one additional arc). 

To find a circulation $x$ in a directed graph $G=(V,E)$ with capacities $l$ and $u$, consider the following construction. First arbitrarily choose $y_e$ for $e\in E$ so that $l(e)\leq y_e\leq u(e)$ for all $e\in E$. This may violate the flow conservation constraints as the excess at vertex $v$ may not be 0: define the excess $ex(v):=\sum_{e\in \delta^+(v)} y_e-\sum_{e\in \delta^-(v)} y_e$. Now add two new vertices say $s'$ and $t'$, and add arcs between any vertex $v\in V$ and either $s'$ or $t'$ in the following way. Let $V^+=\{v: ex(v)>0\}$ and $V^-=\{v: ex(v)<0\}$. For $v\in V^+$ we add an arc $a=(t',v)$ with $l(a)=0$ and $u(a)=ex(v)$, while for $v\in V^-$, we add an arc $a=(v,s')$ with $l(a)=0$ and $u(a)=-ex(v)$. This gives a maximum flow instance on an augmented digraph $G'=(V',E')$ (where $V'=V\cup\{s',t'\}$). By setting the flow on these arcs to be equal to their upper bounds, we get a feasible flow $y$ from $s'$ to $t'$ defined on this augmented arc set (and satisfying flow conservation at all vertices of $V$, but not at $s'$ and $t'$). This initial flow has a net value equal to $-\sum_{v\in V^+} e(v) =\sum_{v\in V^-} e(v) <0$. Since we have an initial ($s'$-$t'$) flow in $G'$, we can use the augmenting path algorithm to find a maximum flow from $s'$ to $t'$ in $G'$. If the net flow out of $s'$ is 0 then all arcs incoming to $s'$ must be at their lower bound 0, and so will be all the arcs outgoing from $t'$ (as net flow out of $s'$ equals the net flow into $t'$). Thus the restriction of $y$ to the arc set $E$ gives a circulation in $G=(V,E)$. On the other hand, if the maximum flow value in this augmented instance is not 0 (it is negative) then no feasible circulation exists in $G$ since such a circulation $x$ corresponds to a flow in the augmented graph of value 0. This shows how to decide if a feasible circulation in $G$ exists, and if so, to find one. 

Can we provide a good characterization for when a circulation in $G$ exists? We can use the max flow min cut theorem in this augmented graph, with source and sink $s'$ and $t'$. Any cut in $G'$ corresponds to $S\cup\{s'\}$ where $S\subseteq V$. The capacity of this cut in $G'$ is given by $$C_{G'}(S\cup \{s'\})=C_G(S)=\sum_{e\in \delta^+(S)} u(e) - \sum_{e\in \delta^-(S)} l(e),$$ since none of the new arcs we have added into $s'$ or from $t'$ will be involved in these cuts. Thus the maximum flow in this augmented graph will be 0 (which is equivalent to the existence of a circulation) if and only $C_G(S) \geq 0$ for all $S\cup V$. This is summarized here. 

\begin{theorem}
Consider a circulation problem in a digraph $G=(V,E)$ with upper and lower capacities $u: E\rightarrow \R$ and $l: E \rightarrow R$. Then a circulation exists if and only if for all $S\subseteq V$:
$$\sum_{e\in \delta^+(S)} u(e) - \sum_{e\in \delta^-(S)} l(e) \geq 0.$$
\end{theorem}



\section{Minimum cuts}

From now on, we assume that we have only upper capacities $u$ and no
lower capacities $l$ ($l(e)=0$ for all $e$). The minimum $s-t$ cut
problem that we have solved so far corresponds to:
$$\min_{S: s\in S, t\notin S} u(\delta^+(S)).$$

If our graph $G=(V,E)$ is undirected and we would like to find the
minimum $s-t$ cut, i.e.
$$\min_{S: s\in S, t\notin S} u(\delta(S)),$$
we can simply replace every edge $e$ by two opposite arcs of the same
capacity and reduce the problem to finding a minimum $s-t$ cut in a
directed graph. As we have just shown, this can be done by a maximum
flow computation. 

Now, consider the problem of finding the {\it global} minimum cut in a
graph. Let us first consider the directed case. Finding the global
mincut (or just the mincut) means finding $S$ minimizing:
$$\min_{S: \emptyset \neq S \neq V} u(\delta^+(S)).$$ This problem can
be reduced to $2(n-1)$ maximum flow computations (where $n=|V|$) in
the following way. First we can arbitrarily choose a vertex $s\in V$ and $s$
will either be in $S$ or in $V\setminus S$. Thus, for any $t\in
V\setminus \{s\}$, we solve two maximum flow problems, one giving us
the minimum $s-t$ cut, the other giving us the minimum $t-s$
cut. Taking the minimum over all such cuts, we get the global mincut
in a directed graph. 

To find the minimum cut problem in an undirected graph, we do not even
need to solve two maximum flow problems for each $t\in V\setminus
\{s\}$, only one of them is enough. Thus the global minimum cut
problem in an undirected graph can be solved by computing $n-1$
maximum flow problems. 
The fastest maximum flow algorithms currently take around $O(mn)$ time (for example, Goldberg and Tarjan's algorithm
\cite{GoldbergT88} take $O(mn\log(n^2/m))$ time). Since we need to use
it $n-1$ times, we can find a mincut in $O(mn^2\log(n^2/m))$ time.
However, these $n-1$ maxflow problem are related, and Hao and Orlin
\cite{HaoO92} have shown that it is possible to solve {\em all} of
them in $O(mn\log(n^2/m))$ by modifying Goldberg and Tarjan's
algorithm. Thus the minimum cut problem can be solved within this time
bound.

We will now derive an algorithm for the (global) mincut problem (in an undirected graph)
which is not based on network flows, and which has a running time
slightly better than Hao and Orlin's. The algorithm is due to Stoer
and Wagner \cite{StoerW94}, and is a simplification of an earlier
result of Nagamochi and Ibaraki \cite{NagamochiI92}. We should also
point out that there is a randomized algorithm due to Karger and Stein
\cite{KargerS93} whose running time is $O(n^2\log^3 n)$, and a
subsequent one due to Karger \cite{Karger96} that runs in $O(m\log^3 n)$.

We first need a definition.
Define, for any two sets $A,B\subseteq V $ of vertices,
\[
u(A:B) := \sum_{i\in A, j\in B} u((i,j)).
\]

The algorithm is described below. In words, the algorithm starts with
any vertex, and build an ordering of the vertices by always adding to
the selected vertices the vertex whose total cost to the previous
vertices is maximized; this is called the {\it maximum adjacency ordering}. The cut induced by the last vertex in this maximum adjacency
ordering is considered, as well as the cuts obtained by recursively
applying the procedure to the graph obtained by shrinking the last two
vertices. (If there are edges from a vertex $v$ to these last two
vertices then we substitute those two edges with only one edge having
capacity equal to the sum of the capacities of the two edges.) The
claim is that the best cut among the cuts considered is the overall
mincut. The formal description is given below.

\begin{center}
\begin{minipage}{11cm}
\medskip
{\sc mincut}$(G)$
\begin{pseudocode}
\item Let $v_1$ be any vertex of $G$
\item $n=|V(G)|$
\item $S=\{v_1\}$
\item {\bf for} $i=2$ {\bf to} $n$
\begin{pseudocode}
\item let $v_i$ the vertex of $V\setminus S$ s.t.
\item $u(S:\{v\})$ is maximized (over all $v\in V\setminus S$)
\item $S:=S\cup\{v_i\}$
\end{pseudocode}
\item {\bf endfor}
\item {\bf if} $n=2$ {\bf then} return the cut $\delta(\{v_n\})$ 
\item {\bf else} 
\begin{pseudocode}
\item Let $G'$ be obtained from $G$ by shrinking $v_{n-1}$ and $v_n$
\item Let $C$ be the cut returned by {\sc mincut}$(G')$
\item Among $C$ and $\delta(\{v_n\})$ return the smaller cut (in terms
of cost)
\end{pseudocode}
\item {\bf endif}
\end{pseudocode}
\end{minipage}
\end{center}

%\begin{figure}[htb]
%\includefigure{cutinst}
%\caption{Illustration of the mincut algorithm.}
%\label{fig:cut}
%\end{figure}

%Figure \ref{fig:cut} illustrates how the algorithm works on an example.
The analysis is based on the following crucial claim. 
\begin{claim}
$\{v_n\}$ (or $\{v_1,v_2,...,v_{n-1}\}$) induces a
min $(v_{n-1},v_n)$-cut in $G$. (Notice that we do not know in advance
$v_{n-1}$ and $v_n$.)
\end{claim}

From this, the correctness of the algorithm follows easily. Indeed,
the mincut is either a $(v_{n-1},v_n)$-cut or not. If it is, we are
fine thanks to the above claim. If it is not, we
can assume by induction on the size of the vertex set that it will be
returned by the call {\sc mincut}$(G')$.


\begin{proof}
Let $v_1,v_2,...,v_i,...,v_j,...,v_{n-1},v_n$ be the sequence of vertices
chosen by the algorithm and let us denote by $A_i$ the sequence
$v_1,v_2,...,v_{i-1}$. We are interested in the cuts that separate
$v_{n-1}$ and $v_n$. Let $C$ be any set such that $v_{n-1}\in C$ and
$v_n\not\in C$. Then we want to prove that the cut induced by $C$
satisfies
\[
u(\delta(C))\ge u(\delta(A_n)).
\]

Let us define vertex $v_i$ to be critical with respect to $C$  if either
$v_i$ or $v_{i-1}$ belongs to $C$ but not both.
We claim that if $v_i$ is critical then
\[
u(A_i:\{v_i\})\le u(C_i:A_i\cup\{v_i\}\setminus C_i)
\]
where $C_i= (A_i\cup\{v_i\})\cap C$.

Notice that this implies that $u(\delta(C))\ge
u(\delta(A_n))$ because $v_n$ is critical.  Now let us prove the claim
 by induction on the sequence of critical vertices.

Let $v_i$ be the first critical vertex. Then
\[
u(A_i:\{v_{i}\}) = u(C_{i} : A_{i}\cup\{v_{i}\}\setminus C_i)
\]
Thus the base of the induction is true.

For the inductive step, let the assertion be true for critical vertex
$v_i$ and let $v_j$ be the next (after $v_i$) critical vertex. Without loss of generality, we may assume $v_i \in C$ and $v_j \not\in C$ because the right-hand side $u(C_j:A_j\cup\{v_j\}\setminus C_j)$ is preserved under the complementation of $C$.
Then
\begin{eqnarray*}
u(A_j:\{v_j\}) &=& u(A_i:\{v_j\})+u(A_j\setminus A_i:\{v_j\})\\
&\le& u(A_i:\{v_i\})+u(A_j\setminus A_i:\{v_j\})\\
&\le& u(C_i:A_i\cup\{v_i\}\setminus C_i)+u(A_j\setminus A_i:\{v_j\})\\
&\le& u(C_j:A_j\cup\{v_j\}\setminus C_j),
\end{eqnarray*}
the first inequality following from the definition of $v_i$, the
second inequality from the inductive hypothesis, and the last from the
fact that $v_j$ is the next critical vertex and that $v_i \in C, v_j \not\in C$.
The proof is concluded observing that $A_n$ induces the cut
$\{v_1,v_2,\cdots,v_{n-1}\} : \{v_n\}$.
\end{proof}

The running time depends on the particular implementation of each iteration. We need to maintain a list of the vertices together with ``keys" equal to their cost to the current set $S$ of vertices, and be able to quickly find the minimum vertex among these and update the keys after adding the minimum one to $S$. Priority queue data structures such as Fibonacci heaps (\url{https://en.wikipedia.org/wiki/Fibonacci_heap}) are designed precisely for this purpose. Using Fibonacci heaps we can implement each iteration in $O(m+n\log n)$ time. 
and this yields a total running time of $O(mn+n^2\log n)$.

\begin{exercises}
\item
Let $G$ be an undirected graph in which the degree of every vertex is
at least $k$. Show that there exist two vertices $s$ and $t$ with at
least $k$ edge-disjoint paths between them.   
\end{exercises}


There is one fundamental property of cut values that was hidden in the algorithm above, and that is {\it submodularity}. Let $f(S)=u(\delta(S))$ where $u$ are (nonnegative) capacities in an undirected graph. A function $f:2^V \rightarrow \R$ is said to be {\it submodular} if for all $A, B \subseteq V$, we have $$f(A)+f(B) \geq f(A\cup B) + f(A\cap B).$$ By considering the contribution of any edge on both sides of the above inequality, one can verify that the cut function of undirected graph is indeed submodular. This is also the case for the cut function $f(S)=u(\delta^+(S))$ in a directed graph, but the cut function in an undirected graph satisfies one more property that was implicitly used in the algorithm above. It is symmetric, i.e. $f(S)=f(V\setminus S)$ for any set $S$. The algorithm we have described can be modified and generalized to work for solving $$\min_{S: \emptyset \neq S \neq V} f(S),$$ where $f$ is a symmetric, submodular function. We will not describe this extension.   % TODO: cite Queyranne

\begin{exercises}
	\item Prove that the cut function of an undirected graph is submodular.
	\item Show that the definition of submodularity is equivalent to the property of {\it having diminishing returns}. A function $f: 2^V \rightarrow$ has diminishing returns if for all $S\subseteq T$ and $e\notin T$ we have $$f(T\cup\{e\})-f(T) \leq f(S\cup\{e\})-f(S).$$
\end{exercises}	 


\section{Minimum $T$-odd cut problem}
Given a graph $G=(V,E)$ with nonnegative edge capacities given by $u$
and an even set $T$ of vertices, the minimum $T$-odd cut problem is to
find $S$ minimizing:
$$\min_{S\subset V: |S\cap T| \mbox{ odd}} u(\delta(S)).$$
We'll say that $S$ is $T$-odd if $|S\cap T|$ is odd. Observe that if
$S$ is $T$-odd, so is $V\setminus S$ and vice versa.  

We  give a polynomial-time algorithm for this problem, and submodularity is also an essential property here. We won't
present the most efficient one, but one of the easiest ones. Let
$ALG(G,T)$ denote this algorithm. The first step of $ALG(G,T)$ is to
find a minimum cut having at least one vertex of $T$ on each side:
$$\min_{S\subset V: \emptyset \neq S\cap T \neq T} u(\delta(S)).$$ This can be done by doing $|T|-1$ minimum $s-t$ cut
computations, by fixing one vertex $s$ in $T$ and then trying all
vertices $t\in T\setminus\{s\}$, and then returning the smallest cut
$S$ obtained in this way.

Now, two things can happen. Either $S$ is a $T$-odd cut in which case
it must be minimum and we are done, or $S$ is $T$-even (i.e. $T\cap S$
has even cardinality). If $S$ is $T$-even, we show in the lemma below
that we can assume that the minimum $T$-odd cut $A$ is either a
subset of $S$ or a subset of $V\setminus S$. Thus we can find by
recursively solving 2 smaller minimum $T$-odd cut problems, one in the
graph $G_1=G/S$ obtained by shrinking $S$ into a single vertex and
letting $T_1=T\setminus S$ and the other in the graph
$G_2=G/(V\setminus S)$ obtained by shrinking $V\setminus S$ and
letting $T_2=T\setminus (V\setminus S)=T\cap S$. Thus the algorithm
makes two calls, $ALG(G_1,T_1)$ and $ALG(G_2,T_2)$ and returns the
smallest (in terms of capacity) $T$-odd cut returned. 

At first glance, it is not obvious that this algorithm is polynomial
as every call may generate two recursive calls. However, letting
$R(k)$ denote an upper bound on the running time of $ALG(G,T)$ for
instances with $|T|=k$ (and say $|V|\leq n$), we can see that 
\begin{enumerate}
\item
$R(2)=\tau$, where $\tau$ is the time needed for a minimum $s-t$ cut
computation,
\item
$R(k) \leq \max_{k_1\geq 2, k_2\geq 2, k=k_1+k_2} \left((k-1)\tau + R(k_1) +
R(k_2)\right).$ 
\end{enumerate}
By induction, we can see that $R(k) \leq k^2 \tau$, as this is true for
$k=2$ and the inductive step is also satisfied: 
\begin{eqnarray*}
R(k) & \leq & \max_{k_1\geq 2, k_2\geq 2, k=k_1+k_2} \left((k-1)\tau +
  k_1^2 \tau+ k_2^2 \tau\right)\\
 & \leq & (k-1)\tau+ 4 \tau + (k-2)^2 \tau \\
& = & (k^2-3k+7) \tau \\
& \leq & k^2 \tau,
\end{eqnarray*} for $k\geq 4$. Thus, this algorithm is polynomial. 

We are left with stating and proving the following lemma, which crucially uses submodularity. 
\begin{lemma}
If $S$ is a minimum cut among those having at least one vertex of $T$
on each side, and $|S\cap T|$ is even then there exists a minimum
$T$-odd cut $A$ with $A\subseteq S$ or $A\subseteq V\setminus S$. 
\end{lemma}

\begin{proof}
Let $B$ be any minimum $T$-odd cut. Partition $T$ into $T_1$, $T_2$,
$T_3$ and $T_4$ as follows: $T_1=T\setminus (B \cup S)$, $T_2=(T\cap S)
\setminus B$, $T_3=T \cap B \cap S$, and $T_4=(T\cap B) \setminus S$. 
Since by definition of $B$ and $S$ we have that $T_1\cup T_2\neq
\emptyset$, $T_2\cup T_3\neq \emptyset$, $T_3\cup T_4\neq \emptyset$
and $T_4\cup T_1\neq \emptyset$, we must have that either $T_1$ and
$T_3$ are non-empty, or $T_2$ and $T_4$ are non-empty. Possibly
replacing $B$ by $V\setminus B$, we can assume that $T_1$ and $T_3$
are non-empty. 

By submodularity of the cut function, we know that 
\begin{equation}\label{cutsubm}
\sum_{e\in \delta(S)} u(e) + \sum_{e\in \delta(B)} u(e) \geq  \sum_{e\in
  \delta(S\cup B)} u(e) + \sum_{e\in \delta(S\cap B)} u(e).
\end{equation} Since
$T_1\neq \emptyset$ and $T_3\neq \emptyset$, both $S\cup B$ and $S\cap
B$ separate vertices of $T$. Furthermore, one of them has to be
$T$-even and the other $T$-odd, as $|(S\cap B)\cap T| + |(S\cup B)
\cap T| = |T_2| + 2 |T_3| + |T_4|=|S\cap T| + |B\cap T|$ is odd. Thus,
one of $S\cup B$ and $S\cap B$ has to have a cut value no greater than
the one of $B$ while the other has a cut value no greater than the one
of $S$. This means that either $S\cap B$ or $S \cup B$ is a minimum
$T$-odd cut. 
\end{proof} 
    


\begin{thebibliography}{99}

\bibitem{GoldbergT88} A.V.\ Goldberg and R.E.\ Tarjan, ``A new
approach to the maximum flow problem'', {\it Journal of the ACM}, {\bf
35}, 921--940, 1988.

\bibitem{HaoO92} X.\ Hao and J.B.\ Orlin, ``A faster algorithm for
finding the minimum cut in a graph'', {\it Proc. of the 3rd ACM-SIAM
Symposium on Discrete Algorithms}, 165--174, 1992. 

\bibitem{Karger96} D.\ Karger, ``Minimum cuts in near-linear time'',
{\it Proc.\ of the 28th STOC}, 56--63, 1996. 

\bibitem{KargerS93} D.\ Karger and C.\ Stein, ``An $\tilde{O}(n^2)$
algorithm for minimum cuts'', {\it Proc. of the 25th STOC}, 757--765,
1993. 

\bibitem{NagamochiI92} H.\ Nagamochi and T.\ Ibaraki, ``Computing
edge-connectivity in multigraphs and capacitated graphs'', {\it SIAM
Journal on Discrete Mathematics}, {\bf 5}, 54--66, 1992.

\bibitem{StoerW94} M. Stoer and F. Wagner, ``A simple mincut
algorithm'', {\it Proc. of ESA94}, Lecture Notes in Computer Science,
{\bf 855}, 141--147, 1994. 

\bibitem{Zwick95} U. Zwick, ``The smallest networks on which the Ford-Fulkerson maximum flow procedure may fail to terminate'', {\it Theoretical Computer Science}, {\bf 148}, 165--170, 1995. 

\end{thebibliography}

  
\end{document}
