\documentclass[12pt]{article}

\usepackage{../lp,amsmath}
% Cross-references for handout numbers.
\usepackage{amsfonts}
%\usepackage{amsthm}
\usepackage{hyperref}
\usepackage{amssymb}
%\usepackage[capitalize]{cleveref}
\usepackage{xcolor}

%\input{handouts}

\newcounter{chapnum}

\newtheorem{definition}{Definition}[chapnum]
\newtheorem{remark}{Remark}[chapnum]
\newtheorem{theorem}{Theorem}[chapnum]
\newtheorem{lemma}[theorem]{Lemma}
\newtheorem{corollary}[theorem]{Corollary}
\newtheorem{proposition}[theorem]{Proposition}
\newtheorem{claim}[theorem]{Claim}
\newtheorem{observation}{Observation}[chapnum]

\renewcommand{\thesection}{\arabic{chapnum}.\arabic{section}}
\renewcommand{\thefigure}{\arabic{chapnum}.\arabic{figure}}


\newenvironment{proof}{\noindent{\bf Proof:} \hspace*{1em}}{
        \hspace*{\fill} $\triangle$ }
\newenvironment{proof_of}[1]{\noindent {\bf Proof of #1:}
        \hspace*{1em} }{\hspace*{\fill} $\triangle$ }
\newenvironment{proof_claim}{\begin{quotation} \noindent}{
        \hspace*{\fill} $\diamond$ \end{quotation}}
\newenvironment{solution}{\noindent{\bf Solution:} \hspace*{1em}}{
        \hspace*{\fill} $\triangle$ }


\newcommand{\R}{{\mathbb R}}
\newcommand{\Z}{{\mathbb Z}}
\newcommand{\Q}{{\mathbb Q}}
\newcommand{\C}{{\mathbb C}}
\newcommand{\N}{{\mathbb N}}
\newcommand{\lin}{\operatorname{lin}}
\newcommand{\aff}{\operatorname{aff}}
\newcommand{\cone}{\operatorname{cone}}
\newcommand{\conv}{\operatorname{conv}}
\newcommand{\vol}{\operatorname{vol}}
\newcommand{\poly}{\operatorname{poly}}




\newcommand{\CF}[1]{{\color{purple}[CF: #1]}}


\newlength{\toppush}
\setlength{\toppush}{2\headheight}
\addtolength{\toppush}{\headsep}

\newcommand{\htitle}[2]{\noindent\vspace*{-\toppush}\newline\parbox{6.5in}
{Massachusetts Institute of Technology \hfill 18.453: Combinatorial Optimization 
\newline
\textbf{Instructor:} Cole Franks \quad \textbf{Notes: }Michel Goemans and Zeb Brady \hfill#2\newline
\mbox{}\hrulefill\mbox{}}\vspace*{1ex}\mbox{}\newline
\begin{center}{\Large\bf #1}\end{center}}

\newcommand{\handout}[2]{\thispagestyle{empty}
 \markboth{ #1 \hfil #2}{ #1 \hfil #2}
 \pagestyle{myheadings}\htitle{#1}{#2}}


\setlength{\oddsidemargin}{0pt}
\setlength{\evensidemargin}{0pt}
\setlength{\textwidth}{6.5in}
\setlength{\topmargin}{0in}
\setlength{\textheight}{8.5in}


\newcounter{exercisenum}
\newcounter{exercisetot}
\setcounter{exercisetot}{0}



\newenvironment{exercises}{
	\begin{list}{{\bf Exercise \arabic{chapnum}-\arabic{exercisenum}. \hspace*{0.5em}}}
	{\setlength{\leftmargin}{0em}
	 \setlength{\rightmargin}{0em}
	 \setlength{\labelwidth}{0em}
	 \setlength{\labelsep}{0em}
	\usecounter{exercisenum}
      \setcounter{exercisenum}{\theexercisetot}}}{\setcounter{exercisetot}{\theexercisenum}\end{list}}


\newenvironment{pseudocode}{
    \begin{list}{}{
        \renewcommand{\makelabel}{$\triangleright$}
        \setlength{\topsep}{0pt}
        \setlength{\leftmargin}{32pt}
        \setlength{\labelwidth}{14pt}
        \setlength{\labelsep}{0mm}
        \setlength{\itemindent}{0mm}
        \setlength{\itemsep}{-3pt}
        \setlength{\itemsep}{0mm}
        \setlength{\parsep}{0pt}%
        \setlength{\listparindent}{0pt}
    }
}
{
    \end{list}
}

\usepackage{graphicx}
\setlength{\topmargin}{-1.0in}
\setlength{\textheight}{9.5in}
\begin{document}

% \handout{Final}{May 18th, 2015}

\paragraph{18.453 final.} This exam is closed book.  Be neat!  {\bf In any problem, you can refer to results we have covered in class, but you need to state them precisely. } If you can't solve one part of a problem, but need the result for a later part of the problem, you can assume that the earlier part of the problem has been proved for the sake of the later part.
% If you need more space for a question, you can continue on one of the extra pages at the end, but write a pointer to it. 
%Don't worry; the exam is probably pretty challenging but that way, the learning process continues... 
Have a great summer! 
 \vspace*{0.1in}

%\vspace*{0.1in}

%{\Large {\bf Your Name:}}



\begin{enumerate}
\item

% optimal permutation with restrictions

For $k \le n$ an integer, define a $k$-\emph{bounded permutation} on $\{1, ..., n\}$ to be a permutation $\sigma$ such that $|\sigma(i) - i| \le k$ for all $i \in \{1, ..., n\}$.

Suppose we are given an integer $k \le n$ and costs $c(i)$ for $i\in\{1,...,n\}$, and our goal is to find a $k$-bounded permutation $\sigma$ on $\{1, ..., n\}$ minimizing $\sum_{i=1}^n c(i) \sigma(i)$. Give a polynomial-time  algorithm for this problem (there is no need to give the most efficient algorithm, but the algorithm should be polynomial in $n$ and $k$). (You can refer to any algorithm we have seen in class.)  

\bigskip

%\newpage 
%~
%%%%%%%%%%%%%%%%
%\newpage
\item

% greedy

Let $M=(E,{\cal I})$ be a matroid with rank function $r$ and suppose we have a cost function $c: E \rightarrow {\mathbb R}_{\geq 0}$ (for simplicity we are assuming that all the costs are nonnegative). We are interested in finding a base $B$ of maximum total cost, i.e.~maximizing $\sum_{e \in B} c(e)$.%, and we hope to derive an algorithm different from the one seen in lecture.

Consider the following greedy algorithm, different from the one covered in lecture, where instead of starting from an empty set and repeatedly adding elements of highest cost, we start from a full set and repeatedly remove elements of lowest cost.
\begin{pseudocode}
\item Sort the elements (from smallest to largest) such that $c(e_1)\leq
  c(e_2)\leq \cdots \leq c(e_{m})$ where $m=|E|$
\item $S=E$
\item For $j=1$ to $m$
\begin{pseudocode}
\item
 if $r(S \setminus \{e_j\})=r(E)$ then $S\leftarrow S\setminus \{e_j\}$
 \end{pseudocode}
 \item Output $S$
 \end{pseudocode}
Prove that this algorithm returns a maximum cost basis in the matroid. 


\bigskip

%%%%%%%%%%%%%%%%%%%
%\newpage
\item

% network flows

\begin{enumerate}
	\item
Consider a directed graph $G=(V,E)$ with nonnegative (upper) capacities $u: E \rightarrow {\mathbb R}$ (and no lower capacities). For any two vertices $s, t\in V$, define $\lambda_{st}\in {\mathbb R}$ to be the maximum flow value from $s$ to $t$. Given any 3 vertices $s, t, u\in V$, show that $\lambda_{su} \geq \min(\lambda_{st},\lambda_{tu})$. 

%\newpage
\item If the graph is undirected, the previous result still holds: $\lambda_{su} \geq \min(\lambda_{st},\lambda_{tu})$ for all $s, t, u$. Furthermore, $\lambda_{st}=\lambda_{ts}$. Now, consider the complete graph $K_V$ on the vertex set $V$ with weight $\lambda_{uv}$ on edge $(u,v)$ for all $u,v$. Let $T$ be a {\it maximum weight} spanning tree on $K_V$ with respect to these weights $\lambda_{uv}$. Argue that for every $(s,t)\notin T$, we have
\[
\lambda_{st}=\min_{(u,v)\in P_{st}} \lambda_{uv}
\]
where $P_{st}$ denotes the edges (of $K_V$) of the unique path in $T$ between $s$ and $t$. (This implies the somewhat surprising result that, over all pairs $(s,t)$, $\lambda_{st}$ can take at most $|V|-1$ values (those along the edges of $T$).)
\end{enumerate}


\bigskip

%%%%%%%%%%%%%%%
%\newpage
\item 

% matching matroid

Consider a bipartite graph $G = (A,B,E)$ with parts $A,B$ and edges $E \subseteq A\times B$. Suppose we have a matroid $M_A = (A,\mathcal{I}_A)$ on $A$ with rank function $r_A$. Define a family of sets $\mathcal{I}_B$ to be the collection of sets $T \subseteq B$ such that there exists a matching $M$ of $G$ with vertex set $V(M) = S\cup T$, such that $S \subseteq A$ and $S \in \mathcal{I}_A$.

\begin{enumerate}
\item Prove that $M_B = (B,\mathcal{I}_B)$ is a matroid. (For {\bf partial credit}, you can do this in the special case where every vertex of $A$ has degree $1$, so that $G$ is the graph of a function from $A$ to $B$.)

\item Prove the following generalization of K\"onig's Theorem, which gives a formula for the rank of $M_B = (B,\mathcal{I}_B)$:
\[
\max_{T \in \mathcal{I}_B} |T| = \min_{C \text{ a vertex cover of } G} r_A(C \cap A) + |C \cap B|.
\]
\end{enumerate}

%%%%%%%%%%%%%%%%

\bigskip

%\newpage
\item ({\bf Extra Credit})

% Commoner's criterion

Given a matrix $A \in \mathbb{R}^{m\times n}$ with entries in $\{-1, 0, 1\}$, we can associate a directed bipartite graph $D_A$ with parts $\{r_1, ..., r_m\}$ and $\{c_1, ..., c_n\}$, which has an edge directed from $r_i$ to $c_j$ when $A_{ij} = +1$, has an edge directed from $c_j$ to $r_i$ when $A_{ij} = -1$, and has no edges between $r_i,c_j$ when $A_{ij} = 0$.

Define a \emph{circuit} of $D_A$ to be a connected subgraph $C \subseteq D_A$ such that every vertex of $C$ has degree $2$. We say that a circuit $C$ of $D_A$ is \emph{odd} if we can flip the directions of an odd number of edges of $C$ to make it into a directed cycle, and otherwise we say that $C$ is \emph{even}.

\begin{enumerate}
\item Suppose that $D_A$ has an \emph{odd} circuit $C$. Show that it is possible to replace some of the entries of $A$ by $0$s to get a matrix $A'$ which is \emph{not} totally unimodular. (Hint: consider the case where $D_A$ consists of just a single circuit.)

\item Suppose that every circuit of $D_A$ is \emph{even}. Show that it is possible to negate some of the rows of $A$ to get a matrix $A''$ with the property that for \emph{every} $k \le m$, the sum of the first $k$ rows of $A''$ is a row vector with all entries in $\{-1,0,1\}$. (Hint: pair up the nonzero entries of each column of $A$ into groups of two, and try to arrange for the corresponding pairs of entries of $A''$ to have opposite signs.)

\item Prove Commoner's sufficient condition for total unimodularity: if every circuit of $D_A$ is \emph{even}, then $A$ is totally unimodular.

%\item Suppose that every circuit of $D_A$ is \emph{even}, and that $D_A$ has at least two perfect matchings (so $A$ is square). Let $\mathcal{G}$ be the graph whose vertices correspond to perfect matchings $M$ of $D_A$, with an edge connecting two matchings $M_1, M_2$ of $D_A$ when the symmetric difference $M_1 \Delta M_2$ is a circuit of $D_A$. Show that in this case, $\mathcal{G}$ is itself a bipartite graph with a perfect matching, and use this to prove $\det(A) = 0$. (Hint: show that for any perfect matching $M$ of $D_A$ and any edge $e$ of $M$, there is at most one alternating cycle for $M$ which contains the edge $e$.)
\end{enumerate}


For part (b), you can use the following fact without proof: if you have a system of equations in variables $x_i \in \{-1,+1\}$, each of the form $x_i = x_j$ or of the form $x_i = -x_j$, then the system has a solution if and only if there are no ``bad cycles'' of the form $x_i = \pm x_j = \pm x_k = \cdots = -x_i$, with each equality coming from one of the equations of your system (possibly with both sides negated).



\end{enumerate}

%%%%%%%%%%%%%%%%%%%%%%%%%%%%%%

\end{document}
