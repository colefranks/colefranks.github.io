\documentclass[12pt]{article}

\usepackage{../lp,amsmath}
% Cross-references for handout numbers.
\usepackage{amsfonts}
%\usepackage{amsthm}
\usepackage{hyperref}
\usepackage{amssymb}
%\usepackage[capitalize]{cleveref}
\usepackage{xcolor}

%\input{handouts}

\newcounter{chapnum}

\newtheorem{definition}{Definition}[chapnum]
\newtheorem{remark}{Remark}[chapnum]
\newtheorem{theorem}{Theorem}[chapnum]
\newtheorem{lemma}[theorem]{Lemma}
\newtheorem{corollary}[theorem]{Corollary}
\newtheorem{proposition}[theorem]{Proposition}
\newtheorem{claim}[theorem]{Claim}
\newtheorem{observation}{Observation}[chapnum]

\renewcommand{\thesection}{\arabic{chapnum}.\arabic{section}}
\renewcommand{\thefigure}{\arabic{chapnum}.\arabic{figure}}


\newenvironment{proof}{\noindent{\bf Proof:} \hspace*{1em}}{
        \hspace*{\fill} $\triangle$ }
\newenvironment{proof_of}[1]{\noindent {\bf Proof of #1:}
        \hspace*{1em} }{\hspace*{\fill} $\triangle$ }
\newenvironment{proof_claim}{\begin{quotation} \noindent}{
        \hspace*{\fill} $\diamond$ \end{quotation}}
\newenvironment{solution}{\noindent{\bf Solution:} \hspace*{1em}}{
        \hspace*{\fill} $\triangle$ }


\newcommand{\R}{{\mathbb R}}
\newcommand{\Z}{{\mathbb Z}}
\newcommand{\Q}{{\mathbb Q}}
\newcommand{\C}{{\mathbb C}}
\newcommand{\N}{{\mathbb N}}
\newcommand{\lin}{\operatorname{lin}}
\newcommand{\aff}{\operatorname{aff}}
\newcommand{\cone}{\operatorname{cone}}
\newcommand{\conv}{\operatorname{conv}}
\newcommand{\vol}{\operatorname{vol}}
\newcommand{\poly}{\operatorname{poly}}




\newcommand{\CF}[1]{{\color{purple}[CF: #1]}}


\newlength{\toppush}
\setlength{\toppush}{2\headheight}
\addtolength{\toppush}{\headsep}

\newcommand{\htitle}[2]{\noindent\vspace*{-\toppush}\newline\parbox{6.5in}
{Massachusetts Institute of Technology \hfill 18.453: Combinatorial Optimization 
\newline
\textbf{Instructor:} Cole Franks \quad \textbf{Notes: }Michel Goemans and Zeb Brady \hfill#2\newline
\mbox{}\hrulefill\mbox{}}\vspace*{1ex}\mbox{}\newline
\begin{center}{\Large\bf #1}\end{center}}

\newcommand{\handout}[2]{\thispagestyle{empty}
 \markboth{ #1 \hfil #2}{ #1 \hfil #2}
 \pagestyle{myheadings}\htitle{#1}{#2}}


\setlength{\oddsidemargin}{0pt}
\setlength{\evensidemargin}{0pt}
\setlength{\textwidth}{6.5in}
\setlength{\topmargin}{0in}
\setlength{\textheight}{8.5in}


\newcounter{exercisenum}
\newcounter{exercisetot}
\setcounter{exercisetot}{0}



\newenvironment{exercises}{
	\begin{list}{{\bf Exercise \arabic{chapnum}-\arabic{exercisenum}. \hspace*{0.5em}}}
	{\setlength{\leftmargin}{0em}
	 \setlength{\rightmargin}{0em}
	 \setlength{\labelwidth}{0em}
	 \setlength{\labelsep}{0em}
	\usecounter{exercisenum}
      \setcounter{exercisenum}{\theexercisetot}}}{\setcounter{exercisetot}{\theexercisenum}\end{list}}


\newenvironment{pseudocode}{
    \begin{list}{}{
        \renewcommand{\makelabel}{$\triangleright$}
        \setlength{\topsep}{0pt}
        \setlength{\leftmargin}{32pt}
        \setlength{\labelwidth}{14pt}
        \setlength{\labelsep}{0mm}
        \setlength{\itemindent}{0mm}
        \setlength{\itemsep}{-3pt}
        \setlength{\itemsep}{0mm}
        \setlength{\parsep}{0pt}%
        \setlength{\listparindent}{0pt}
    }
}
{
    \end{list}
}

\usepackage{graphicx}
\setlength{\topmargin}{-1.0in}
\setlength{\textheight}{9.5in}
\begin{document}

% \handout{Final}{May 18th, 2015}

\paragraph{18.453 final.} This exam is closed book.  Be neat!  {\bf In any problem, you can refer to results we have covered in class, but you need to state them precisely. } 
% If you need more space for a question, you can continue on one of the extra pages at the end, but write a pointer to it. 
Don't worry; the exam is probably pretty challenging but that way, the learning process continues... Have a great summer! 
 \vspace*{0.1in}

%\vspace*{0.1in}

%{\Large {\bf Your Name:}}



\begin{enumerate}
\item
\begin{enumerate}
\item
Define when a matrix $A$ is totally unimodular. 

\item
State precisely the main property of a system of linear inequalities whose underlying matrix is totally unimodular.

\item
Let $A$ be a $0-1$ matrix. We say that $A$ has the consecutive-one property if for all rows of A, the value 1 appears consecutively (and the remaining entries are 0). Show that any consecutive-ones matrix is totally unimodular. 
\end{enumerate}
%\newpage 
%~
%%%%%%%%%%%%%%%%
%\newpage
\item
\begin{enumerate}
	\item
Give a complete description in terms of linear inequalities of the matching polytope for an arbitrary graph $G=(V,E)$. Argue that your stated inequalities are valid for the matching polytope, but you do {\it not} need to prove that they form a complete description of it. 
\end{enumerate}


%%%%%%%%%%%%%%%%%%%
%\newpage
\item
\begin{enumerate}
	\item
Consider a directed graph $G=(V,E)$ with nonnegative (upper) capacities $u: E \rightarrow {\mathbb R}$ (and no lower capacities). For any two vertices $s, t\in V$, define $\lambda_{st}\in {\mathbb R}$ to be the maximum flow value from $s$ to $t$. Given any 3 vertices $s, t, u\in V$, show that $\lambda_{su} \geq \min(\lambda_{st},\lambda_{tu})$. 

%\newpage
\item If the graph is undirected, the previous result still holds: $\lambda_{su} \geq \min(\lambda_{st},\lambda_{tu})$ for all $s, t, u$. Furthermore, $\lambda_{st}=\lambda_{ts}$. Now, consider the complete graph $K_V$ on the vertex set $V$ with weight $\lambda_{uv}$ on edge $(u,v)$ for all $u,v$. Let $T$ be a {\it maximum weight} spanning tree on $K_V$ with respect to these weights $\lambda_{uv}$. Argue that for every $(s,t)\notin T$, we have $$\lambda_{s,t}=\min_{(u,v)\in P_{st}} \lambda_{uv}$$ where $P_{st}$ denotes the (edges of $K_V$) of the unique path in $T$ between $s$ and $t$. (This implies the somewhat surprising result that, over all pairs $(s,t)$, $\lambda_{st}$ can take at most $|V|-1$ values (those along the edges of $T$).)
\end{enumerate}



%%%%%%%%%%%%%%%
%\newpage
\item For a matching $M$ in a graph $G=(V,E)$, let $V(M)$ denote the vertices matched in $M$. 
\begin{enumerate}
\item
Suppose that we are given a set $S\subseteq V$ and a matching $M$ covering, i.e. such that $S\subseteq V(M)$. Given $v\notin S$, how would you decide (efficiently) 	whether there exists a matching $M'$ such that $S\cup\{v\} \subseteq V(M')$. You can use building blocks we have seen in lectures (state them, but no need to reprove them), but you should justify any additional statements. 


%\newpage
\item

 Consider a (not necessarily bipartite) graph $G=(V,E)$ and a
  profit function $p: V\rightarrow \R_+$. (The profit function is
  defined on the vertices of $G$.) Our goal is to find a matching $M$
  which maximizes $\sum_{v\in V(M)} p(v)$. How would you solve this problem? Justify
  your solution.
  
\end{enumerate}

%%%%%%%%%%%%%%%%

%\newpage
\item Consider a bipartite graph $G=(V,E)$ with bipartition $(A,B)$
  (so $V=A \cup B$). Suppose we are also given a matroid $M=(A,{\cal
    I})$ defined on $A$ (one of the sides of the bipartition). We would
  like to restrict our attention to {\it independent matchings} $M$
  which are those matchings $M$ such that $\{a\in A: \exists b\in B$
  with $(a,b)\in M\}\in {\cal I}$. The maximum independent matching
  problem is the problem of finding an independent matching of maximum
  cardinality. How can this problem be solved efficiently?  Justify your
  answer. (You can use as building blocks things we have seen in
  lectures (state them precisely though); anything else needs to be justified.)




\end{enumerate}

%%%%%%%%%%%%%%%%%%%%%%%%%%%%%%

\end{document}