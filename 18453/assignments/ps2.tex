\documentclass[12pt]{article} % Cross-references for handout numbers.
\usepackage{amsfonts}
%\usepackage{amsthm}
\usepackage{hyperref}
\usepackage{amssymb}
%\usepackage[capitalize]{cleveref}
\usepackage{xcolor}

%\input{handouts}

\newcounter{chapnum}

\newtheorem{definition}{Definition}[chapnum]
\newtheorem{remark}{Remark}[chapnum]
\newtheorem{theorem}{Theorem}[chapnum]
\newtheorem{lemma}[theorem]{Lemma}
\newtheorem{corollary}[theorem]{Corollary}
\newtheorem{proposition}[theorem]{Proposition}
\newtheorem{claim}[theorem]{Claim}
\newtheorem{observation}{Observation}[chapnum]

\renewcommand{\thesection}{\arabic{chapnum}.\arabic{section}}
\renewcommand{\thefigure}{\arabic{chapnum}.\arabic{figure}}


\newenvironment{proof}{\noindent{\bf Proof:} \hspace*{1em}}{
        \hspace*{\fill} $\triangle$ }
\newenvironment{proof_of}[1]{\noindent {\bf Proof of #1:}
        \hspace*{1em} }{\hspace*{\fill} $\triangle$ }
\newenvironment{proof_claim}{\begin{quotation} \noindent}{
        \hspace*{\fill} $\diamond$ \end{quotation}}
\newenvironment{solution}{\noindent{\bf Solution:} \hspace*{1em}}{
        \hspace*{\fill} $\triangle$ }


\newcommand{\R}{{\mathbb R}}
\newcommand{\Z}{{\mathbb Z}}
\newcommand{\Q}{{\mathbb Q}}
\newcommand{\C}{{\mathbb C}}
\newcommand{\N}{{\mathbb N}}
\newcommand{\lin}{\operatorname{lin}}
\newcommand{\aff}{\operatorname{aff}}
\newcommand{\cone}{\operatorname{cone}}
\newcommand{\conv}{\operatorname{conv}}
\newcommand{\vol}{\operatorname{vol}}
\newcommand{\poly}{\operatorname{poly}}




\newcommand{\CF}[1]{{\color{purple}[CF: #1]}}


\newlength{\toppush}
\setlength{\toppush}{2\headheight}
\addtolength{\toppush}{\headsep}

\newcommand{\htitle}[2]{\noindent\vspace*{-\toppush}\newline\parbox{6.5in}
{Massachusetts Institute of Technology \hfill 18.453: Combinatorial Optimization 
\newline
\textbf{Instructor:} Cole Franks \quad \textbf{Notes: }Michel Goemans and Zeb Brady \hfill#2\newline
\mbox{}\hrulefill\mbox{}}\vspace*{1ex}\mbox{}\newline
\begin{center}{\Large\bf #1}\end{center}}

\newcommand{\handout}[2]{\thispagestyle{empty}
 \markboth{ #1 \hfil #2}{ #1 \hfil #2}
 \pagestyle{myheadings}\htitle{#1}{#2}}


\setlength{\oddsidemargin}{0pt}
\setlength{\evensidemargin}{0pt}
\setlength{\textwidth}{6.5in}
\setlength{\topmargin}{0in}
\setlength{\textheight}{8.5in}


\newcounter{exercisenum}
\newcounter{exercisetot}
\setcounter{exercisetot}{0}



\newenvironment{exercises}{
	\begin{list}{{\bf Exercise \arabic{chapnum}-\arabic{exercisenum}. \hspace*{0.5em}}}
	{\setlength{\leftmargin}{0em}
	 \setlength{\rightmargin}{0em}
	 \setlength{\labelwidth}{0em}
	 \setlength{\labelsep}{0em}
	\usecounter{exercisenum}
      \setcounter{exercisenum}{\theexercisetot}}}{\setcounter{exercisetot}{\theexercisenum}\end{list}}


\newenvironment{pseudocode}{
    \begin{list}{}{
        \renewcommand{\makelabel}{$\triangleright$}
        \setlength{\topsep}{0pt}
        \setlength{\leftmargin}{32pt}
        \setlength{\labelwidth}{14pt}
        \setlength{\labelsep}{0mm}
        \setlength{\itemindent}{0mm}
        \setlength{\itemsep}{-3pt}
        \setlength{\itemsep}{0mm}
        \setlength{\parsep}{0pt}%
        \setlength{\listparindent}{0pt}
    }
}
{
    \end{list}
}

\usepackage{graphicx,../lp,amsmath} 


\begin{document}



\handout{Problem set 2}{Feb 21, 2019}
\medskip
This problem set is due at 11:00 pm on Mar 18, 2021. Instructions same as the first pset; some key points: collaboration is encouraged but you {\bf must} write up
your answers in your own words. You are required to list and identify clearly all sources and collaborators except instructors, TA or lecture notes.  %Your grade will not count unless you submit this information.

To submit your homework, upload it in PDF format using the Gradescope tool in Canvas before the deadline.
%1-4/1-5, 1-6, 1-7?, 1-8?, 1-10, 1-11, 1-15, 1-18, 1-20*
Unless otherwise indicated, the problems are graded out of 4 points. The graduate questions are worth 2 bonus points each for undergraduates.


\begin{enumerate}
\item
Exercise 2-2 from the notes on (non-bipartite) matchings. 
\item
Exercise 2-6 from the notes on (non-bipartite) matchings.

\item (For undergrads; optional for grads) Exercise 2-7, Parts 1 and 2 (Part 3 optional). You may want to look at the new and improved pdf file on non-bipartite matching. \textbf{Hint:}\footnote{Try to show that if $B$ is a blossom with respect to a maximum matching in a factor critical graph $G$, $G/B$ is also factor critical.}

\item
Consider $S=\{(1,0,1),(0,1,1),(1,1,2),(0,2,2)\}\subseteq \R^3\}.$ Describe lin$(S)$, aff$(S)$, cone$(S)$ and conv$(S)$ (as a polyhedron, in terms of the linear equalities/inequalities). \textbf{Hint: }\footnote{ For example, conv$(\{(0,1), (1,0)\}) = \{x: x_1 + x_2 = 1, x_1 \geq 0, x_2 \geq 0\}$. It's ok if you do this in an ad-hoc manner using pictures etc, as long as your final inequalities are correct.}

\item
%Let $G=(V,E)$ be a bipartite graph having a perfect matching. Consider the set ${\cal M}\subseteq \R^E$ of the incidence vectors of all perfect matchings of $G$. We have seen a description of conv$({\cal M})$ as a system of linear inequalities/equalities. Give a description (and a proof) of the conic hull, cone$({\cal M})$, as the solution set of a system of linear inequalities and equalities.
Suppose you are given a description of a polyhedron $P$ as the solution set to a system of linear inequalities/equalities. Describe a procedure for finding a description of the conic hull, cone$(P)$, as the solution set of a system of linear inequalities and equalities. \textbf{Hint:} \footnote{ You may use that for a polyhedron $P$,  cone$(P) = \{x: \exists \lambda > 0\text{ such that } \lambda x \in P\} \cup \{0\}$. Introduce a new variable and use Fourier-Motzkin elimination to get rid of it.}

\item
For graduate students, exercise 2-5. 

\end{enumerate}


\end{document}
