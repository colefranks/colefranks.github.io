\documentclass[12pt]{article} % Cross-references for handout numbers.
\usepackage{amsfonts}
%\usepackage{amsthm}
\usepackage{hyperref}
\usepackage{amssymb}
%\usepackage[capitalize]{cleveref}
\usepackage{xcolor}

%\input{handouts}

\newcounter{chapnum}

\newtheorem{definition}{Definition}[chapnum]
\newtheorem{remark}{Remark}[chapnum]
\newtheorem{theorem}{Theorem}[chapnum]
\newtheorem{lemma}[theorem]{Lemma}
\newtheorem{corollary}[theorem]{Corollary}
\newtheorem{proposition}[theorem]{Proposition}
\newtheorem{claim}[theorem]{Claim}
\newtheorem{observation}{Observation}[chapnum]

\renewcommand{\thesection}{\arabic{chapnum}.\arabic{section}}
\renewcommand{\thefigure}{\arabic{chapnum}.\arabic{figure}}


\newenvironment{proof}{\noindent{\bf Proof:} \hspace*{1em}}{
        \hspace*{\fill} $\triangle$ }
\newenvironment{proof_of}[1]{\noindent {\bf Proof of #1:}
        \hspace*{1em} }{\hspace*{\fill} $\triangle$ }
\newenvironment{proof_claim}{\begin{quotation} \noindent}{
        \hspace*{\fill} $\diamond$ \end{quotation}}
\newenvironment{solution}{\noindent{\bf Solution:} \hspace*{1em}}{
        \hspace*{\fill} $\triangle$ }


\newcommand{\R}{{\mathbb R}}
\newcommand{\Z}{{\mathbb Z}}
\newcommand{\Q}{{\mathbb Q}}
\newcommand{\C}{{\mathbb C}}
\newcommand{\N}{{\mathbb N}}
\newcommand{\lin}{\operatorname{lin}}
\newcommand{\aff}{\operatorname{aff}}
\newcommand{\cone}{\operatorname{cone}}
\newcommand{\conv}{\operatorname{conv}}
\newcommand{\vol}{\operatorname{vol}}
\newcommand{\poly}{\operatorname{poly}}




\newcommand{\CF}[1]{{\color{purple}[CF: #1]}}


\newlength{\toppush}
\setlength{\toppush}{2\headheight}
\addtolength{\toppush}{\headsep}

\newcommand{\htitle}[2]{\noindent\vspace*{-\toppush}\newline\parbox{6.5in}
{Massachusetts Institute of Technology \hfill 18.453: Combinatorial Optimization 
\newline
\textbf{Instructor:} Cole Franks \quad \textbf{Notes: }Michel Goemans and Zeb Brady \hfill#2\newline
\mbox{}\hrulefill\mbox{}}\vspace*{1ex}\mbox{}\newline
\begin{center}{\Large\bf #1}\end{center}}

\newcommand{\handout}[2]{\thispagestyle{empty}
 \markboth{ #1 \hfil #2}{ #1 \hfil #2}
 \pagestyle{myheadings}\htitle{#1}{#2}}


\setlength{\oddsidemargin}{0pt}
\setlength{\evensidemargin}{0pt}
\setlength{\textwidth}{6.5in}
\setlength{\topmargin}{0in}
\setlength{\textheight}{8.5in}


\newcounter{exercisenum}
\newcounter{exercisetot}
\setcounter{exercisetot}{0}



\newenvironment{exercises}{
	\begin{list}{{\bf Exercise \arabic{chapnum}-\arabic{exercisenum}. \hspace*{0.5em}}}
	{\setlength{\leftmargin}{0em}
	 \setlength{\rightmargin}{0em}
	 \setlength{\labelwidth}{0em}
	 \setlength{\labelsep}{0em}
	\usecounter{exercisenum}
      \setcounter{exercisenum}{\theexercisetot}}}{\setcounter{exercisetot}{\theexercisenum}\end{list}}


\newenvironment{pseudocode}{
    \begin{list}{}{
        \renewcommand{\makelabel}{$\triangleright$}
        \setlength{\topsep}{0pt}
        \setlength{\leftmargin}{32pt}
        \setlength{\labelwidth}{14pt}
        \setlength{\labelsep}{0mm}
        \setlength{\itemindent}{0mm}
        \setlength{\itemsep}{-3pt}
        \setlength{\itemsep}{0mm}
        \setlength{\parsep}{0pt}%
        \setlength{\listparindent}{0pt}
    }
}
{
    \end{list}
}

\usepackage{graphicx,../lp,amsmath} 


\begin{document}



\handout{Problem set 1}{Feb 18, 2021}

\medskip
This problem set is due at 11:00 pm on Mar 04, 2021. You are encouraged to collaborate with other students in this class (especially your pset group). But you {\bf must} write up
your answers in your own words. Do not look at the solutions of others while writing your
own. You are required to list and identify clearly all sources and collaborators. (``Wikipedia''
is too vague.) On the other hand, you don't need to list as sources the instructor, TA or lecture notes. Do list classmates, tutors and any other
source, animate or inanimate.  Failure to disclose sources may result in a grade of zero for the assignment, and referral to the Committee on Discipline.  %Your grade will not count unless you submit this information.

To submit your homework, upload it in PDF format using the Gradescope tool in Canvas before the deadline.
\textbf{You will have to select the page(s) containing each problem part when you upload, the grader
for that problem will only see these page(s). To make grading easier please try to put each problem part on a separate page, or, for shorter parts, visually separate them from each other.}
\begin{itemize}
\item your name,
\item sources and collaborators: As discussed above this should be the list of all texts, web
sites and people consulted. For example ``Consulted:
Jane Doe, Wikipedia Bernoulli differential equation''. If no sources other than the
course materials or instructors were consulted, then write ``Consulted: none''.
\end{itemize}
%1-4/1-5, 1-6, 1-7?, 1-8?, 1-10, 1-11, 1-15, 1-18, 1-20*
Unless otherwise indicated, the problems are graded out of 4 points. The graduate questions are worth 2 bonus points each for undergraduates.
\begin{enumerate}
\item 
%Exercise 1-2 of the bipartite matching notes.
Exercise 1-4 of the bipartite matching notes.
\item 
%Exercise 1-4 of the bipartite matching notes. 
Exercise 1-6 of the bipartite matching notes.
\item 
Exercise 1-7 of the bipartite matching notes.
\item 
(3+3 = 6 pts) Exercise 1-10 of the bipartite matching notes. 
%Exercise 1-15 of the bipartite matching notes.
\item
Exercise 1-18 of the bipartite matching notes. 
\item
%For graduate students, exercises 1-11 and 1-15. 
For graduate students, exercises 1-11 and 1-20.
\end{enumerate}


\end{document}
