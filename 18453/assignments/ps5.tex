\documentclass[12pt]{article} % Cross-references for handout numbers.
\usepackage{amsfonts}
%\usepackage{amsthm}
\usepackage{hyperref}
\usepackage{amssymb}
%\usepackage[capitalize]{cleveref}
\usepackage{xcolor}

%\input{handouts}

\newcounter{chapnum}

\newtheorem{definition}{Definition}[chapnum]
\newtheorem{remark}{Remark}[chapnum]
\newtheorem{theorem}{Theorem}[chapnum]
\newtheorem{lemma}[theorem]{Lemma}
\newtheorem{corollary}[theorem]{Corollary}
\newtheorem{proposition}[theorem]{Proposition}
\newtheorem{claim}[theorem]{Claim}
\newtheorem{observation}{Observation}[chapnum]

\renewcommand{\thesection}{\arabic{chapnum}.\arabic{section}}
\renewcommand{\thefigure}{\arabic{chapnum}.\arabic{figure}}


\newenvironment{proof}{\noindent{\bf Proof:} \hspace*{1em}}{
        \hspace*{\fill} $\triangle$ }
\newenvironment{proof_of}[1]{\noindent {\bf Proof of #1:}
        \hspace*{1em} }{\hspace*{\fill} $\triangle$ }
\newenvironment{proof_claim}{\begin{quotation} \noindent}{
        \hspace*{\fill} $\diamond$ \end{quotation}}
\newenvironment{solution}{\noindent{\bf Solution:} \hspace*{1em}}{
        \hspace*{\fill} $\triangle$ }


\newcommand{\R}{{\mathbb R}}
\newcommand{\Z}{{\mathbb Z}}
\newcommand{\Q}{{\mathbb Q}}
\newcommand{\C}{{\mathbb C}}
\newcommand{\N}{{\mathbb N}}
\newcommand{\lin}{\operatorname{lin}}
\newcommand{\aff}{\operatorname{aff}}
\newcommand{\cone}{\operatorname{cone}}
\newcommand{\conv}{\operatorname{conv}}
\newcommand{\vol}{\operatorname{vol}}
\newcommand{\poly}{\operatorname{poly}}




\newcommand{\CF}[1]{{\color{purple}[CF: #1]}}


\newlength{\toppush}
\setlength{\toppush}{2\headheight}
\addtolength{\toppush}{\headsep}

\newcommand{\htitle}[2]{\noindent\vspace*{-\toppush}\newline\parbox{6.5in}
{Massachusetts Institute of Technology \hfill 18.453: Combinatorial Optimization 
\newline
\textbf{Instructor:} Cole Franks \quad \textbf{Notes: }Michel Goemans and Zeb Brady \hfill#2\newline
\mbox{}\hrulefill\mbox{}}\vspace*{1ex}\mbox{}\newline
\begin{center}{\Large\bf #1}\end{center}}

\newcommand{\handout}[2]{\thispagestyle{empty}
 \markboth{ #1 \hfil #2}{ #1 \hfil #2}
 \pagestyle{myheadings}\htitle{#1}{#2}}


\setlength{\oddsidemargin}{0pt}
\setlength{\evensidemargin}{0pt}
\setlength{\textwidth}{6.5in}
\setlength{\topmargin}{0in}
\setlength{\textheight}{8.5in}


\newcounter{exercisenum}
\newcounter{exercisetot}
\setcounter{exercisetot}{0}



\newenvironment{exercises}{
	\begin{list}{{\bf Exercise \arabic{chapnum}-\arabic{exercisenum}. \hspace*{0.5em}}}
	{\setlength{\leftmargin}{0em}
	 \setlength{\rightmargin}{0em}
	 \setlength{\labelwidth}{0em}
	 \setlength{\labelsep}{0em}
	\usecounter{exercisenum}
      \setcounter{exercisenum}{\theexercisetot}}}{\setcounter{exercisetot}{\theexercisenum}\end{list}}


\newenvironment{pseudocode}{
    \begin{list}{}{
        \renewcommand{\makelabel}{$\triangleright$}
        \setlength{\topsep}{0pt}
        \setlength{\leftmargin}{32pt}
        \setlength{\labelwidth}{14pt}
        \setlength{\labelsep}{0mm}
        \setlength{\itemindent}{0mm}
        \setlength{\itemsep}{-3pt}
        \setlength{\itemsep}{0mm}
        \setlength{\parsep}{0pt}%
        \setlength{\listparindent}{0pt}
    }
}
{
    \end{list}
}

\usepackage{graphicx,../lp,amsmath} 


\begin{document}



\handout{Problem set 5}{April 22, 2021}

\medskip
This problem set is due on \textbf{Thursday May 13, 2021}. Instructions are the same as the first pset; some key points: collaboration is encouraged but you {\bf must} write up
your answers in your own words. You are required to list and identify clearly all sources and collaborators except instructors, TA or lecture notes. Each question is worth 4 points and each extra credit question is worth 2 points.

\begin{enumerate}
\item
%Exercise 5-4 of the notes on matroids. 
Exercise 5-5 of the notes on matroids.
\item
%Exercise 5-5 of the notes on matroids. 
Exercise 5-7 of the notes on matroids.
\item
%Exercise 5-8 of the notes on matroids. 
Exercise 5-8 of the notes on matroids. 
\item
%Suppose we are given a matroid $M=(E,{\cal I})$, a weight function $w: E \rightarrow \R$, and we are interested in finding a base $B$ of $M$ of maximum total weight $w(B)$. We could use the greedy algorithm, but suppose instead we use the following algorithm. Start from any base $B_1$ of $M$. At iteration $i$, if there exists $e\notin B_i$ and $f\in B_i$ with $w(e)>w(f)$ such that $B_i+e-f$ is a base of the matroid then let $B_{i+1}=B_i+e-f$ and continue; otherwise (if there is not such $e$ and $f$) stop and output $B_{i}$.
%\begin{enumerate}
%	\item Prove that this algorithm correctly outputs a base of maximum total weight. 
%	\item Suppose now that among all possible $e$ and $f$ satisfying the condition above, we should the one for which $w(e)-w(f)$ is maximum. Prove that $w(B_{i+1})-w(B_i)\geq \frac{1}{r} (w(B^*)-w(B_i))$, where $r=r(E)$ is the size of any base, and $B^*$ is an optimum base. (This result can be used to bound the number of iterations of such an algorithm.)
%\end{enumerate}
%\item
%Exercise 6-1 of the notes on matroid intersection. 

%Show the derivation of Theorem 6.2 from Theorem 6.1, from the notes on matroid intersection.  
Show the derivation of Theorem 6.3 from Theorem 6.1, from the notes on matroid intersection.
\item ({\bf Extra Credit}) Exercise 5-12 of the notes on matroids. 
\item
({\bf Extra Credit}) Use Theorem 6.8 from the notes on matroid intersection to show that if $G = (V,E)$ is a graph with $|E| \ge 2|V|-2$, such that for every nontrivial subset $S \subsetneq V$ the number of edges of $G$ with both endpoints in $S$ is at most $2|S|-2$, then $G$ has two edge-disjoint spanning trees.

\end{enumerate}


\end{document}
