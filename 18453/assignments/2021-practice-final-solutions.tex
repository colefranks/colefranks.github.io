\documentclass[12pt]{article}

\usepackage{../lp,amsmath}
% Cross-references for handout numbers.
\usepackage{amsfonts}
%\usepackage{amsthm}
\usepackage{hyperref}
\usepackage{amssymb}
%\usepackage[capitalize]{cleveref}
\usepackage{xcolor}

%\input{handouts}

\newcounter{chapnum}

\newtheorem{definition}{Definition}[chapnum]
\newtheorem{remark}{Remark}[chapnum]
\newtheorem{theorem}{Theorem}[chapnum]
\newtheorem{lemma}[theorem]{Lemma}
\newtheorem{corollary}[theorem]{Corollary}
\newtheorem{proposition}[theorem]{Proposition}
\newtheorem{claim}[theorem]{Claim}
\newtheorem{observation}{Observation}[chapnum]

\renewcommand{\thesection}{\arabic{chapnum}.\arabic{section}}
\renewcommand{\thefigure}{\arabic{chapnum}.\arabic{figure}}


\newenvironment{proof}{\noindent{\bf Proof:} \hspace*{1em}}{
        \hspace*{\fill} $\triangle$ }
\newenvironment{proof_of}[1]{\noindent {\bf Proof of #1:}
        \hspace*{1em} }{\hspace*{\fill} $\triangle$ }
\newenvironment{proof_claim}{\begin{quotation} \noindent}{
        \hspace*{\fill} $\diamond$ \end{quotation}}
\newenvironment{solution}{\noindent{\bf Solution:} \hspace*{1em}}{
        \hspace*{\fill} $\triangle$ }


\newcommand{\R}{{\mathbb R}}
\newcommand{\Z}{{\mathbb Z}}
\newcommand{\Q}{{\mathbb Q}}
\newcommand{\C}{{\mathbb C}}
\newcommand{\N}{{\mathbb N}}
\newcommand{\lin}{\operatorname{lin}}
\newcommand{\aff}{\operatorname{aff}}
\newcommand{\cone}{\operatorname{cone}}
\newcommand{\conv}{\operatorname{conv}}
\newcommand{\vol}{\operatorname{vol}}
\newcommand{\poly}{\operatorname{poly}}




\newcommand{\CF}[1]{{\color{purple}[CF: #1]}}


\newlength{\toppush}
\setlength{\toppush}{2\headheight}
\addtolength{\toppush}{\headsep}

\newcommand{\htitle}[2]{\noindent\vspace*{-\toppush}\newline\parbox{6.5in}
{Massachusetts Institute of Technology \hfill 18.453: Combinatorial Optimization 
\newline
\textbf{Instructor:} Cole Franks \quad \textbf{Notes: }Michel Goemans and Zeb Brady \hfill#2\newline
\mbox{}\hrulefill\mbox{}}\vspace*{1ex}\mbox{}\newline
\begin{center}{\Large\bf #1}\end{center}}

\newcommand{\handout}[2]{\thispagestyle{empty}
 \markboth{ #1 \hfil #2}{ #1 \hfil #2}
 \pagestyle{myheadings}\htitle{#1}{#2}}


\setlength{\oddsidemargin}{0pt}
\setlength{\evensidemargin}{0pt}
\setlength{\textwidth}{6.5in}
\setlength{\topmargin}{0in}
\setlength{\textheight}{8.5in}


\newcounter{exercisenum}
\newcounter{exercisetot}
\setcounter{exercisetot}{0}



\newenvironment{exercises}{
	\begin{list}{{\bf Exercise \arabic{chapnum}-\arabic{exercisenum}. \hspace*{0.5em}}}
	{\setlength{\leftmargin}{0em}
	 \setlength{\rightmargin}{0em}
	 \setlength{\labelwidth}{0em}
	 \setlength{\labelsep}{0em}
	\usecounter{exercisenum}
      \setcounter{exercisenum}{\theexercisetot}}}{\setcounter{exercisetot}{\theexercisenum}\end{list}}


\newenvironment{pseudocode}{
    \begin{list}{}{
        \renewcommand{\makelabel}{$\triangleright$}
        \setlength{\topsep}{0pt}
        \setlength{\leftmargin}{32pt}
        \setlength{\labelwidth}{14pt}
        \setlength{\labelsep}{0mm}
        \setlength{\itemindent}{0mm}
        \setlength{\itemsep}{-3pt}
        \setlength{\itemsep}{0mm}
        \setlength{\parsep}{0pt}%
        \setlength{\listparindent}{0pt}
    }
}
{
    \end{list}
}

\usepackage{graphicx}
\setlength{\topmargin}{-1.0in}
\setlength{\textheight}{9.5in}
\begin{document}

% \handout{Final}{May 18th, 2015}

\noindent {\Large 18.453 Practice Final} \\
~\\

\paragraph{Instructions.} This is practice for a {\bf timed} final. This is meant to be done in \textbf{3} hours with access to notes and course material, but no access to collaborators. For best practice I suggest trying to complete it under these conditions. Afterwards please tell me if 3 hours felt like enough.
 \vspace*{0.1in}

%\vspace*{0.1in}

%{\Large {\bf Your Name:}}



\begin{enumerate}

\item
Answer true or false. For items \textbf{not} marked with *, if true, provide a concise reason (no rigor necessary) and if false, exhibit a counterexample.
\begin{enumerate}
\item Every matching that is not maximum in a graph $G$ has an augmenting path.*
\item If $A, b$ are integral, then the linear program $\max\{c^T x: Ax \leq b\}$ has an integral maximizer.
\item The set of matchings in a bipartite graph forms a matroid.
\item Given a bipartite graph, the set of subgraphs of degree at most two is the intersection of two matroids.
\item Given a separation oracle for a polyhedron $P\subset [0,1]^n$, it is always possible to test feasibility of $P$ with polynomially many calls to the separation oracle.
\end{enumerate}

\textbf{Answers.}

\begin{enumerate}
\item Yes.
\item No. For the following program the only maximizer is fractional:
\[
\max \left\{ x_1 ~\middle\vert~
\begin{pmatrix}
  1 & 1 \\
  1 & -1
\end{pmatrix}
\begin{pmatrix}
  x_1 \\
  x_2 
\end{pmatrix}\le
\begin{pmatrix}
  1 \\
  0
\end{pmatrix}
\right\}.
\]  

\item No. For the complete graph with parts $\{a_1, a_2\}, \{b_1, b_2\}$ the exchange axiom is violated for the independent sets $X = \{\{a_1, b_1\}\}$, $Y = \{\{a_1, b_2\}, \{a_2, b_1\}\}$.
    
\item Yes. For a graph $G = (V,E)$ with bipartition $V = A \sqcup B$, the set of subgraphs of degree at most two is the intersection of two partition matroids $(E, \mathcal{I}_A)$ and $(E, \mathcal{I}_B)$ defined as follows. Let $E = \bigsqcup_{a\in A} E_a$ be the partition of edges according to their left endpoint; then $\mathcal{I}_A = \{E' \subset E ~\vert~ |E' \cap E_a| \le 2 ~\forall a \in A\}$. Similarly one defines $\mathcal{I}_B$.

\item No. It is impossible to tell a single point from the empty set, because the separation oracle might output the same.
\end{enumerate}

\newpage
\item
% optimal permutation with restrictions

For $k \le n$ an integer, define a $k$-\emph{bounded permutation} on $\{1, ..., n\}$ to be a permutation $\sigma$ such that $|\sigma(i) - i| \le k$ for all $i \in \{1, ..., n\}$.

Suppose we are given an integer $k \le n$ and costs $c(i)$ for $i\in\{1,...,n\}$, and our goal is to find a $k$-bounded permutation $\sigma$ on $\{1, ..., n\}$ minimizing $\sum_{i=1}^n c(i) \sigma(i)$. Give a polynomial-time algorithm for this problem (there is no need to give the most efficient algorithm, but the algorithm should be polynomial in $n$ and $k$). (You can refer to any algorithm we have seen in class.)\\

\textbf{Solution: } This can be cast as an instance of the minimum cost perfect matching problem. A permutation $\sigma$ is none other than a matching in the complete bipartite graph with bipartition $A \cup B$ equal to two disjoint copies of $[n]$, where there is an edge between $i$ and $\sigma(i)$. The restriction that the permutation is $k$-bounded requires that the permutaiton correspond to a matching in the subgraph $G = (A \cup B, E)$ where $E = \{(i,j): i \in A, j \in B, |i - j| \leq k\}$. If we set the cost of the edge $(i,j)$ to be $c_{i,j} = c(i) j$, then the minimum cost perfect matching in $G$ yields the minimum cost $k$-bounded permutation. This can be solved in (strongly) polynomial time using the Hungarian algorithm.



\newpage

%\newpage
%~
%%%%%%%%%%%%%%%%
%\newpage
\item
\begin{enumerate}
	\item
Consider a directed graph $G=(V,E)$ with nonnegative (upper) capacities $u: E \rightarrow {\mathbb R}$ (and no lower capacities). For any two vertices $s, t\in V$, define $\lambda_{st}\in {\mathbb R}$ to be the maximum flow value from $s$ to $t$. Given any 3 vertices $s, t, u\in V$, show that $\lambda_{su} \geq \min(\lambda_{st},\lambda_{tu})$.

%\newpage
\item If the graph is undirected, the previous result still holds: $\lambda_{su} \geq \min(\lambda_{st},\lambda_{tu})$ for all $s, t, u$. Furthermore, $\lambda_{st}=\lambda_{ts}$. Now, consider the complete graph $K_V$ on the vertex set $V$ with weight $\lambda_{uv}$ on edge $\{u,v\}$ for all $u,v$. Let $T$ be a {\it maximum weight} spanning tree on $K_V$ with respect to these weights $\lambda_{uv}$. Argue that for every $\{s,t\}\notin T$, we have $$\lambda_{st}=\min_{\{u,v\}\in P_{st}} \lambda_{uv}$$ where $P_{st}$ denotes (the edges of $K_V$ of) the unique path in $T$ between $s$ and $t$. (This implies the somewhat surprising result that, over all pairs $(s,t)$, $\lambda_{st}$ can take at most $|V|-1$ values (those along the edges of $T$).)
\end{enumerate}

\textbf{Solution.}

\begin{enumerate}
	\item
Consider a minimal cut separating $s$ and $u$, that is, a partition $V = S \sqcup U$ such that $s \in S$, $u \in U$, $\lambda_{su} = \sum\limits_{e \in \delta^+(S)} u(e)$. If $t \in S$, then the same cut separates $t$ and $u$, and so $\lambda_{su} \ge \lambda_{tu}$. Otherwise $t \in U$, the cut separates $s$ and $t$, and $\lambda_{su} \ge \lambda_{st}$. In either case, $\lambda_{su} \geq \min(\lambda_{st},\lambda_{tu})$.

\item Using the previous part, one shows by induction that $\lambda_{st} \ge \min\limits_{\{u,v\}\in P_{st}} \lambda_{uv}$. Let $\{u,v\}$ be a specific edge delivering the minimum in the right-hand side. We'd like to argue that $\lambda_{st} \le \lambda_{uv}$. Indeed, if not, then $\lambda_{st} > \lambda_{uv}$, and one can replace the edge $\{u,v\}$ in the tree $T$ by the edge $\{s,t\}$. The resulting subgraph is still a spanning tree, whereas its weight gets increased, which contradicts the choice of $T$. Therefore, $\lambda_{st} = \lambda_{uv}$.
\end{enumerate}

\newpage

%%%%%%%%%%%%%%%%%%%
%\newpage

\item

% matching matroid

Consider a bipartite graph $G = (A,B,E)$ with parts $A,B$ and edges $E \subseteq A\times B$. Suppose we have a matroid $M_A = (A,\mathcal{I}_A)$ on $A$ with rank function $r_A$. Define a family of sets $\mathcal{I}_B$ to be the collection of sets $T \subseteq B$ such that there exists a matching $M$ of $G$ with vertex set $V(M) = S\cup T$, such that $S \subseteq A$ and $S \in \mathcal{I}_A$.

 Prove that $M_B = (B,\mathcal{I}_B)$ is a matroid. (For {\bf half credit}, you can do this in the special case where every vertex of $A$ has degree $1$, so that $G$ is the graph of a function from $A$ to $B$.)\\
 
 \textbf{Solution: } We need to show that $\mathcal{I}_B$ is downward-closed and satisfies the exchange property. Downward closure follows because if $T$ is matched to independent set $S \in \mathcal{I}_A$, then $T' \subset T$ is matched to a subset of $S' \subset S$ which is also independent.
 
 For the exchange property, we must show that given $T', T \in \mathcal{I}_B$ with $|T'| > |T|$, there is some element $t \in T'\setminus T$ such that $T + t \in \mathcal{I}_B$. That is, $t \in T'\setminus T$ such that $T + t$ can be matched to an independent set in $\mathcal{I}_A$. Let $M$ be a matching between $T, S$ for $S\in \mathcal{I}_A,$ and let $M'$ be a matching between $T', S'$ for $S'\in \mathcal{I}_A.$ We'll use alternating paths in $M \cup M'$ to gradually modify $M$ until we get the matching we want.
 %As $|M'| > |M|$, there is an augmenting path for $M$ in $M' \cup M$. The matching obtained by augmenting $M$ along this path will cover a new element $t \in T'$, but might not match the set $T + t$ to an independent set in $A$. 
 
As $|S'| > |S|$, the basis exchange property for $I_A$ says there is an element $a$ in $S' \setminus S$ such that $S +a$ is independent. The element $a$ is covered by $M'$ but not $M$, so has degree $1$ in $M' \cup M$. As $M' \cup M$ has maximum degree two, the connected component containing $a$ is a path $P$. Moreover, it must be alternating in $M$. If this path ends in $B$, then we may augment $M$ along $P$ to cover an element $t$ of $T' \setminus T$, and we are done. If instead the path ends at some element $a' \in A$, then $a \in S \setminus S'$. The symmetric difference of $M\Delta P$ matches $T$ to the independent set $S + a - a'$. Replace $S \leftarrow S + a - a'$ and $M$ by $M \Delta P$. The size of the intersection $S \cap S'$ has increased by one. Continue this process until we are done (we have covered an element of $T' \setminus T$) or $S \subset S'$. Once $S \subset S'$, we can take $t$ to be any element of $T'\setminus T$.\\ \textbf{Note: } In retrospect this problem was too hard. It's still good practice, but don't let it scare you.


% network flows


\newpage

%%%%%%%%%%%%%%%
%\newpage
\item Let $x \in [0,1]^n$ be an unknown vector, and we suppose have access to a separation oracle for the set $S = [x_1, x_1 + 0.1] \times  \dots \times [x_n, x_n + 0.1] \subset \R^n$. Can we find a point in $S$ in time polynomial in $n$, and if so, how? (You can refer to any algorithm we have seen in class).
    
\textbf{Solution.} 

We exploit the ellipsoid method. The starting ellipsoid $E_0$ is the ball of radius $\sqrt{n}/2$ centered at $(1/2, \ldots, 1/2)$. Recall Claim~7.2 from the notes: the volumes of successive ellipsoids $E_k$ in the method decay exponentially,
\[
\mbox{Vol}(E_k) \le \mbox{Vol}(E_0) \exp \left( -\frac{k}{2(n+1)}\right).
\]
If by $k\textsuperscript{th}$ step a point in $S$ is not found, then $S \subset E_k$, and
\[
0.1^n = \mbox{Vol}(S) < \mbox{Vol}(E_k) \le \mbox{Vol}(E_0) \exp \left( -\frac{k}{2(n+1)}\right).
\]
It follows that $k < 2 n(n+1) \log 10$. Hence, after $2 n(n+1) \log 10$ iterations of the ellipsoid method a point in $S$ will be found.
% matching matroid


%%%%%%%%%%%%%%%%


% Commoner's criterion





\end{enumerate}

%%%%%%%%%%%%%%%%%%%%%%%%%%%%%%

\end{document}
