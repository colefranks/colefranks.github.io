\documentclass[12pt]{article} % Cross-references for handout numbers.
\usepackage{amsfonts}
%\usepackage{amsthm}
\usepackage{hyperref}
\usepackage{amssymb}
%\usepackage[capitalize]{cleveref}
\usepackage{xcolor}

%\input{handouts}

\newcounter{chapnum}

\newtheorem{definition}{Definition}[chapnum]
\newtheorem{remark}{Remark}[chapnum]
\newtheorem{theorem}{Theorem}[chapnum]
\newtheorem{lemma}[theorem]{Lemma}
\newtheorem{corollary}[theorem]{Corollary}
\newtheorem{proposition}[theorem]{Proposition}
\newtheorem{claim}[theorem]{Claim}
\newtheorem{observation}{Observation}[chapnum]

\renewcommand{\thesection}{\arabic{chapnum}.\arabic{section}}
\renewcommand{\thefigure}{\arabic{chapnum}.\arabic{figure}}


\newenvironment{proof}{\noindent{\bf Proof:} \hspace*{1em}}{
        \hspace*{\fill} $\triangle$ }
\newenvironment{proof_of}[1]{\noindent {\bf Proof of #1:}
        \hspace*{1em} }{\hspace*{\fill} $\triangle$ }
\newenvironment{proof_claim}{\begin{quotation} \noindent}{
        \hspace*{\fill} $\diamond$ \end{quotation}}
\newenvironment{solution}{\noindent{\bf Solution:} \hspace*{1em}}{
        \hspace*{\fill} $\triangle$ }


\newcommand{\R}{{\mathbb R}}
\newcommand{\Z}{{\mathbb Z}}
\newcommand{\Q}{{\mathbb Q}}
\newcommand{\C}{{\mathbb C}}
\newcommand{\N}{{\mathbb N}}
\newcommand{\lin}{\operatorname{lin}}
\newcommand{\aff}{\operatorname{aff}}
\newcommand{\cone}{\operatorname{cone}}
\newcommand{\conv}{\operatorname{conv}}
\newcommand{\vol}{\operatorname{vol}}
\newcommand{\poly}{\operatorname{poly}}




\newcommand{\CF}[1]{{\color{purple}[CF: #1]}}


\newlength{\toppush}
\setlength{\toppush}{2\headheight}
\addtolength{\toppush}{\headsep}

\newcommand{\htitle}[2]{\noindent\vspace*{-\toppush}\newline\parbox{6.5in}
{Massachusetts Institute of Technology \hfill 18.453: Combinatorial Optimization 
\newline
\textbf{Instructor:} Cole Franks \quad \textbf{Notes: }Michel Goemans and Zeb Brady \hfill#2\newline
\mbox{}\hrulefill\mbox{}}\vspace*{1ex}\mbox{}\newline
\begin{center}{\Large\bf #1}\end{center}}

\newcommand{\handout}[2]{\thispagestyle{empty}
 \markboth{ #1 \hfil #2}{ #1 \hfil #2}
 \pagestyle{myheadings}\htitle{#1}{#2}}


\setlength{\oddsidemargin}{0pt}
\setlength{\evensidemargin}{0pt}
\setlength{\textwidth}{6.5in}
\setlength{\topmargin}{0in}
\setlength{\textheight}{8.5in}


\newcounter{exercisenum}
\newcounter{exercisetot}
\setcounter{exercisetot}{0}



\newenvironment{exercises}{
	\begin{list}{{\bf Exercise \arabic{chapnum}-\arabic{exercisenum}. \hspace*{0.5em}}}
	{\setlength{\leftmargin}{0em}
	 \setlength{\rightmargin}{0em}
	 \setlength{\labelwidth}{0em}
	 \setlength{\labelsep}{0em}
	\usecounter{exercisenum}
      \setcounter{exercisenum}{\theexercisetot}}}{\setcounter{exercisetot}{\theexercisenum}\end{list}}


\newenvironment{pseudocode}{
    \begin{list}{}{
        \renewcommand{\makelabel}{$\triangleright$}
        \setlength{\topsep}{0pt}
        \setlength{\leftmargin}{32pt}
        \setlength{\labelwidth}{14pt}
        \setlength{\labelsep}{0mm}
        \setlength{\itemindent}{0mm}
        \setlength{\itemsep}{-3pt}
        \setlength{\itemsep}{0mm}
        \setlength{\parsep}{0pt}%
        \setlength{\listparindent}{0pt}
    }
}
{
    \end{list}
}

\usepackage{graphicx,../lp,amsmath} 


\begin{document}



\handout{Matroid intersection activity}{April 29, 2021}

\medskip
Collaborate on these with your breakout room in explain.mit.edu (or using whatever method you find convenient).

\begin{enumerate}
\item Let $G$ be a bipartite graph with bipartition $A, B$.
%Exercise 5-4 of the notes on matroids. 
\begin{enumerate}
\item Given an example showing that the set of matchings does not form the independent sets of a matroid.
\item Show that the set of matchings is the intersection of two matroids. \textbf{ Hint: }\footnote{It is the intersection of two partition matroids. }
\end{enumerate}
\newpage

\item Recall a problem from Pset 4: given an undirected graph $G$ and an assignment $p$ of numbers to the vertices, we'd like to direct the edges in $G$ so that every vertex has at most $p(v)$ incoming edges. 
\begin{enumerate}
\item Describe a pair of matroids whose largest common independent set has size $|E|$ if and only if $G$ has a direction satisfying the above condition. \textbf{Hint: }\footnote{Again, two partition matroids will suffice.}
\end{enumerate}
\newpage

\item Suppose $G$ is an undirected graph and the edge set $E$ of $G$ has been ``colored,"
 %i.e. partitioned into disjoint sets $E_1, \dots, E_t$. 
 Show that the set of \emph{colorful spanning trees} (spanning trees whose edges are all different colors) is the set of common bases of two matroids. \textbf{ Hint: }\footnote{This time you can use a graphic matroid and a partition matroid.}
\newpage

%Exercise 5-5 of the notes on matroids. 
\item For a directed graph $D$ and a ``root'' vertex $r \in V(D)$ such that $r$ has no incoming edges, define an \emph{arborescence} to be a spanning tree of $D$ directed away from $r$.\footnote{Here spanning tree just means that it's a spanning tree in the underlying undirected graph $G$.} 

\begin{enumerate}
\item  Let $G$ be the underlying undirected graph of $D$ obtained by forgetting the directions of the edges.\footnote{$G$ may have multi-edges if both directions $(u,v)$ and $(v,u)$ of an edge were present in $E(D)$.} Show that any subgraph of $D$ which corresponds to a spanning tree in $G$ and has at most one edge entering each vertex is an arborescence.
\item Show that the set of arborescences of $D, r$ is the set of common bases of two matroids. \textbf{Hint: } \footnote{Again you can use the intersection of a graphic matroid and a partition matroid.}
\end{enumerate}
\newpage
\item Consider an undirected graph $G$. We'd like to decide if $G$ is the union of two edge-disjoint spanning trees. Given a matroid $M = (E, I)$, define its \emph{dual matroid} $M^* $ to be $(E, I^*)$ where $I^*$ is the set of subsets of $E$ whose complements contain a base of $M$. 
\begin{enumerate}
\item Describe a pair of matroids whose largest common independent set has size $|V|-1$ if and only if $G$ has two edge-disjoint spanning trees. You may use that the dual matroid is indeed a matroid.
\item \textbf{Bonus: } prove that the dual matroid is a matroid.
\end{enumerate}
%Exercise 5-8 of the notes on matroids. 
\item 



\end{enumerate}


\end{document}
